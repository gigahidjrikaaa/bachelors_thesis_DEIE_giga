Praise be to Allah SWT for His abundant blessings, grace, and guidance, enabling the completion of this thesis. The long journey of completing this research has been filled with twists and turns, challenges, and invaluable lessons. Throughout the preparation of this thesis, I have received tremendous guidance, assistance, and support from various parties. Therefore, I would like to express my sincere gratitude to:

\begin{enumerate}
	\item Prof. Ir. Hanung Adi Nugroho, S.T., M.E., Ph.D., IPM., SMIEEE., as the Head of the Department of Electrical and Information Engineering, for their guidance, strategic vision, and for cultivating an academic atmosphere that enabled this research to flourish.
    
	\item Ir. Lesnanto Multa Putranto, S.T., M.Eng., Ph.D., IPM., SMIEEE., as the Vice Head of the Department of Electrical and Information Engineering, for their encouragement, administrative support, and dedication to student development, all of which greatly enriched my academic experience.

	\item Dr. Bimo Sunarfri Hantono, S.T., M.Eng., as my first thesis advisor. Thank you for the direction, guidance, and patience in steering this research to completion. Every discussion and feedback has shaped a clearer research direction amidst the complexity of ever-evolving innovation.

	\item Dr. Ir. Guntur Dharma Putra, S.T., M.Sc., as my second thesis advisor. Thank you for the guidance, support, and valuable insights throughout the completion of this thesis.
	
	\item My beloved parents, who have provided financial support and endless prayers throughout my education at Universitas Gadjah Mada. Without their sacrifices and trust, this achievement would never have been realized.
	
	\item My partner, Virna Amrita who has always accompanied me through Discord during long nights of struggle, providing encouragement when motivation began to fade, and being a source of comfort in times of joy and hardship. Your presence has been a light in the darkness, especially when facing challenges from the volatility of the crypto world that influenced this research journey.
	
	\item My siblings, Wisda and Ria, who have consistently prayed for and supported every step of my academic journey.
	
	\item My close friends--Azfar, Ariq, Zakong, Diamond, Arif, Ditya, Nando, Aufa, Difta, Akhdan, Evan, Aji and others who have been companions in arms during college, sharing joys and sorrows, and serving as an incredible support system. You are my second family who made my days on campus more meaningful.
	
	\item PT INA17, who has shown interest in and provided support for this project, validating that the research conducted has real applicative value in the industry.
	
	\item The Sumbu Labs team--Maulana, Dzikran, Farhan, Azfar, and Virna--who have helped me in working through one of the biggest projects of my life while doing this thesis in parallel. Our collaboration in developing CAR-dano (now Ototentik) has been an unforgettable experience. You are not just colleagues, but partners in making dreams come true.
	
	\item All parties involved in the EDU Chain Global Hackathon: Semester 3, where the UGM-AICare project successfully won 6000 USD. This achievement is proof that hard work and innovation can be rewarded, even though it sometimes became a beautiful "distraction" from the focus of thesis writing.

\end{enumerate}

The journey of completing this thesis has taught me that innovation does not always follow a straight path. There are times when we are tempted to branch out, explore new ideas, and even get "lost" in hackathon after hackathon. However, each of these experiences has enriched my understanding and broadened my perspective on how technology can make a real impact on society. There were days when I felt lonely working from home, but the support from my loved ones made every challenge feel lighter.

The motivation behind choosing AI agents as the focus of my bachelor's thesis stems from a deeply personal mission: to elevate the standard of mental health services at UGM. Throughout my time as a student, I witnessed firsthand, both in myself and in my peers, how difficult it is to seek help for mental health concerns. We are often too busy, or we simply fail to prioritize our mental wellbeing until it becomes critical. Many students struggle in silence, not because help isn't available, but because the barriers to access feel too high. This realization drove me to create Aika, the AI agent in UGM-AICare, designed to provide proactive interventions and regular check-ups that meet students where they are, when they need it most.

This vision was significant enough for me to embrace the ambitious scope of this work, even knowing it would take longer to complete than a typical bachelor's thesis. I only wish the best for UGM, just as my parents and friends have always wished the best for me. This university has been the place where I met remarkable people who humbled me, challenged my perspectives, and grounded me in reality. It shaped not just my academic journey, but my character and values. If this research can contribute to making mental health support more accessible and effective for future generations of UGM students, then every late night, every challenge, and every moment of uncertainty will have been worth it.

Finally, I hope that this thesis can contribute to the advancement of knowledge, particularly in the fields of artificial intelligence and healthcare technology, and can serve as inspiration for future research. May this work bring benefits to us all, aamiin.

\begin{flushright}
	\begin{tabular}{c}
		Yogyakarta, November 12, 2025 \\
		\vspace{1cm} \\
		Giga Hidjrika Aura Adkhy
	\end{tabular}
\end{flushright}