\chapter{Conclusion and Future Work}
\label{chap:conclusion}

This final chapter synthesizes the findings of the research, drawing conclusions based on the design, implementation, and evaluation of the proposed agentic AI framework. It revisits the research questions to assess the extent to which the project's objectives were met. Finally, it outlines the limitations of the current work and proposes concrete directions for future research.

\section{Conclusion}

This thesis confronted the systemic inefficiencies of the traditional, reactive mental health support paradigm prevalent in higher education. The core problem identified was the "insight-to-action gap," where institutions fail to act on potential indicators of student distress, placing the full burden of help-seeking on the students themselves—often the very individuals least capable of initiating it. To address this, this research undertook a Design Science approach to construct and validate a novel solution: a proactive, multi-agent framework named the \textbf{Safety Agent Suite}, prototyped within the UGM-AICare project.

The evaluation conducted in Chapter~\ref{chap:evaluation} provides empirical evidence that the designed artifact successfully achieves its primary objectives. The key conclusions, mapped directly to the research questions, are as follows:

\begin{enumerate}
    \item \textbf{Proactive Safety is Technically Feasible (RQ1):} The evaluation of the Safety Triage Agent (STA) demonstrated its capability to accurately and rapidly classify crisis situations from conversational text. The achievement of a low False Negative Rate (FNR) confirms that the agent can reliably identify at-risk students, even when their distress is not explicitly stated. This finding represents a crucial first step in shifting the support paradigm, as it provides a mechanism for system-initiated intervention, directly addressing the core failure of reactive models.

    \item \textbf{Reliable Agentic Orchestration Closes the Insight-to-Action Gap (RQ2):} The validation of the LangGraph-based orchestration confirmed that the multi-agent system can execute complex, stateful workflows with high reliability. The high success rate of tool calls and correct state transitions demonstrated that the framework can autonomously move from insight (e.g., a crisis classification from the STA) to action (e.g., the Case Management Agent creating a formal case file for human review). This automated workflow is the practical mechanism that closes the insight-to-action gap, a central goal of this thesis.

    \item \textbf{High-Quality, Privacy-Preserving Support is Achievable (RQ3):} The evaluation confirmed that the framework can deliver valuable outputs without compromising user privacy. The Therapeutic Coach Agent (TCA) was shown to generate empathetic and contextually appropriate guidance, meeting the quality standards defined by the evaluation rubric. Simultaneously, the successful code and unit test validation of the Insights Agent's (IA) k-anonymity implementation proves that it is possible to derive strategic, population-level insights for data-driven decision-making while rigorously protecting individual student identities.
\end{enumerate}

In summary, this thesis successfully designed, built, and validated a proof-of-concept for a proactive mental health support framework. The results indicate that the agentic architecture is not merely a theoretical construct but a viable and effective model for transforming institutional support systems, making them more scalable, responsive, and, most importantly, proactive.

\section{Suggestions for Future Work}

While this research successfully demonstrated the technical feasibility of the proposed framework, its scope as a bachelor's thesis necessitates acknowledging its limitations and outlining avenues for future inquiry. The following suggestions are offered to researchers and practitioners seeking to build upon this work:

\begin{enumerate}
    \item \textbf{Clinical Validation and Efficacy Studies:} The current evaluation was focused on technical performance and functional correctness. The most critical next step is to conduct formal clinical trials under the supervision of an ethics review board and mental health professionals. Such studies would be needed to measure the framework's actual impact on student well-being outcomes (e.g., reduction in anxiety symptoms) and to validate its safety and efficacy in a live, real-world environment.

    \item \textbf{Enhancing Cultural and Linguistic Nuance:} The prototype was developed primarily for the Indonesian-speaking UGM context. Future research should focus on enhancing the agents' understanding of cultural nuances, slang, and indirect expressions of distress specific to different student populations. This could involve fine-tuning the underlying language models on localized datasets and conducting qualitative studies with diverse user groups to improve the agents' conversational appropriateness.

    \item \textbf{Longitudinal and Multi-Modal Data Integration:} The current system primarily analyzes textual data from a single interaction. A more advanced implementation could integrate data from multiple sources over time (with user consent) to build a more holistic understanding of student well-being. This could include integrating data from the Learning Management System (LMS) or other university platforms to identify long-term behavioral patterns, though this would require a significant investigation into the associated ethical and privacy challenges.

    \item \textbf{Exploration of Advanced Agentic Behaviors:} The current agents follow a relatively fixed orchestration. Future work could explore more advanced agentic concepts, such as dynamic goal formulation, automated strategy planning, and self-healing capabilities where the agent system can autonomously adapt its own workflows in response to repeated failures or changing environmental conditions.
\end{enumerate}

These directions for future work highlight the significant potential for further innovation in the field of AI-driven mental health support, building upon the foundational agentic framework established in this thesis.
