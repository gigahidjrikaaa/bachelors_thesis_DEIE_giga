\chapter{Conclusion and Future Work}
\label{chap:conclusion}

This final chapter synthesizes the findings of the research, drawing conclusions based on the design, implementation, and evaluation of the proposed agentic AI framework. It revisits the research questions to assess the extent to which the project's objectives were met. Finally, it outlines the limitations of the current work and proposes concrete directions for future research.

\section{Conclusion}

This thesis confronted the systemic inefficiencies of the traditional, reactive mental health support paradigm prevalent in higher education. The core problem identified was the "insight-to-action gap," where institutions fail to act on potential indicators of student distress, placing the full burden of help-seeking on the students themselves, who are often the very individuals least capable of initiating it. To address this, this research undertook a Design Science approach to construct and validate a novel solution: a proactive, multi-agent framework named the \textbf{Safety Agent Suite}, prototyped within the UGM-AICare project.

The evaluation conducted in Chapter~\ref{chap:evaluation} provides empirical evidence regarding the designed artifact's capabilities and limitations. The key conclusions, mapped directly to the research questions, are as follows:

\begin{enumerate}
    \item \textbf{Proactive Safety via Two-Tier Detection (RQ1):} The evaluation revealed the value of the defense-in-depth architecture. While Aika's real-time triage (Tier 1) achieved 72\% sensitivity, the retrospective Safety Triage Agent (Tier 2) achieved 100\% sensitivity. The combined system thus achieves a \textbf{0\% False Negative Rate}, meeting the critical safety target. This +28\% Safety Net Improvement validates the architectural hypothesis that asynchronous, conversation-level analysis can compensate for real-time limitations. The system can reliably identify at-risk students without requiring explicit disclosure.

    \item \textbf{Orchestration Requires Refinement; Intervention Quality Exceeds Target (RQ2):} The orchestration evaluation yielded a \textbf{64.71\% state transition accuracy}, below the 95\% target. Failure analysis identified three primary categories: context degradation in multi-turn crises, over-escalation of ambiguous passive ideation, and inadequate handling of third-party danger scenarios. However, most failures were conservative (over-escalation), which is preferable in a safety-critical domain. In contrast, the Therapeutic Coach Agent achieved a \textbf{mean quality score of 4.08/5.0}, exceeding the 3.5 target. This confirms that LLM-based therapeutic guidance can achieve clinically acceptable quality, even as the orchestration logic requires further refinement.

    \item \textbf{Strategic Proactivity through Privacy-Preserving Insights (RQ3):} The k-anonymity implementation was successfully validated. The Insights Agent correctly suppressed aggregations below the k=5 threshold (e.g., a "Critical" severity group with n=3 was suppressed), while correctly reporting larger groups (e.g., "High" severity with n=7). This confirms that privacy-preserving institutional analytics are technically feasible.
\end{enumerate}

In summary, this thesis successfully designed, built, and validated a proof-of-concept for a proactive mental health support framework. The two-tier safety architecture achieves zero false negatives, and intervention quality exceeds baseline targets. However, the orchestration accuracy results indicate that the system is not yet production-ready and requires continued refinement, particularly for multi-turn context management and edge-case handling. The artifact demonstrates technical feasibility while honestly acknowledging areas for improvement.

\section{Suggestions for Future Work}

While this research successfully demonstrated technical feasibility, the evaluation results reveal specific areas requiring further development. The following suggestions are prioritized based on the empirical findings:

\begin{enumerate}
    \item \textbf{Orchestration Accuracy Improvement (High Priority):} The 64.71\% state transition accuracy is the most pressing limitation. Future work should focus on:
    \begin{itemize}
        \item \textbf{Enhanced Context Management:} Implementing explicit conversation state persistence to prevent context degradation in multi-turn crisis scenarios.
        \item \textbf{Third-Party Danger Handling:} Extending the intent classification system to explicitly recognize scenarios involving danger to others, not just the user.
        \item \textbf{Prompt Engineering Refinement:} Iterative tuning of the risk assessment thresholds to reduce over-escalation of ambiguous passive ideation while maintaining sensitivity.
    \end{itemize}

    \item \textbf{Clinical Validation and Efficacy Studies:} The current evaluation was focused on technical performance. The most critical next step is to conduct formal clinical trials under ethics board supervision to measure the framework's actual impact on student well-being outcomes and to validate its safety in a live environment.

    \item \textbf{Enhancing Cultural and Linguistic Nuance:} The evaluation revealed challenges with code-switching (mixed Indonesian/English) and local slang. Future research should focus on fine-tuning the underlying language models on localized datasets and conducting qualitative studies with diverse Indonesian student populations.

    \item \textbf{Real-Time Triage Optimization:} Aika's 72\% sensitivity, while complemented by the STA safety net, represents an opportunity for improvement. Techniques such as few-shot prompting with crisis-specific examples or lightweight fine-tuning could improve real-time detection without sacrificing latency.

    \item \textbf{Exploration of Advanced Privacy Models:} While k-anonymity proved functional, future iterations could explore Differential Privacy for stronger mathematical guarantees, particularly as the system scales to larger populations with more granular analytics needs.
\end{enumerate}

These directions for future work are directly informed by the evaluation results, ensuring that subsequent research addresses the demonstrated limitations while building upon the validated capabilities.
