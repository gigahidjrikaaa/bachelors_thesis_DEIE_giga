\chapter{Implementation and Evaluation}
\label{chap:evaluation}

This chapter reports how the prototype was exercised and what we learned from it. The focus is on the agents and their behavior in safety‑relevant scenarios. We keep the scope practical and transparent so results can be reproduced and audited.

\section{Implementation Artifact: The UGM-AICare Prototype}
\label{sec:implementation_artifact}

The conceptual framework and agentic architecture detailed in Chapter~\ref{chap:system_design} were realized as a tangible software artifact within the UGM-AICare project. This prototype serves as the concrete object of study for the evaluation presented in this chapter. It is a full-stack web application designed to provide a practical testbed for the proposed proactive mental health support model. The complete source code for the artifact is publicly available for review and replication\footnote{The UGM-AICare project repository can be accessed at \url{https://github.com/gigahidjrikaaa/UGM-AICare} or through \url{https://aicare.sumbu.xyz}}.

The artifact's technical implementation translates the architectural design into a working system:

\begin{itemize}
    \item \textbf{Backend Services:} The core of the system is a Python-based backend built on the \textbf{FastAPI} web framework. Each specialized agent (STA, TCA, CMA, IA) is implemented as a distinct service within this backend, ensuring modularity and separation of concerns. This service-oriented architecture allows for independent development, testing, and scaling of each agent's capabilities.

    \item \textbf{Agent Orchestration Core:} The multi-agent coordination logic, described conceptually as a state machine in Chapter 3, is implemented using \textbf{LangGraph}. LangGraph provides the underlying engine to define the nodes (agents and tools) and edges (conditional transitions) of the agentic workflow. This allows the Aika Meta-Agent to dynamically route user requests and manage the flow of information between the specialized agents based on the evolving state of the conversation.

    \item \textbf{Frontend Interface:} A user-facing web application, built with \textbf{Next.js} and TypeScript, provides the conversational interface for students and the administrative dashboard for counselors. This interface communicates with the FastAPI backend via a RESTful API, ensuring a clean separation between the presentation layer and the backend agentic logic.

    \item \textbf{Integrated Observability:} As detailed in Section~\ref{sec:monitoring_infra}, the backend is instrumented with Prometheus for quantitative metrics and Langfuse for detailed tracing. This instrumentation is not an afterthought but a core part of the implementation, providing the empirical data necessary for the evaluation that follows.
\end{itemize}

This implementation provides the technical foundation upon which the evaluation protocols described in the remainder of this chapter are executed.

\section{Monitoring and Observability Infrastructure}
\label{sec:monitoring_infra}

To enable a rigorous and transparent evaluation of the agentic framework, a dual-stack observability infrastructure was implemented. This infrastructure is foundational to the Design Science methodology, providing the empirical data required to validate the research questions outlined in Chapter 1. The stack combines quantitative performance monitoring with deep, qualitative trace analysis, ensuring a holistic view of the system's operational behavior.

\subsection{Prometheus for Quantitative Performance Metrics}
For high-level, real-time performance monitoring, the backend exposes custom metrics to a Prometheus time-series database. This allows for the quantitative analysis of system health and efficiency. Key metrics include:
\begin{itemize}
    \item \textbf{Agent Processing Time (\code{agent\_processing\_time\_seconds}):} A histogram metric that tracks the reasoning latency for each agent, crucial for evaluating the performance aspect of RQ1 (Proactive Safety).
    \item \textbf{Tool Call Outcomes (\code{tool\_calls\_total}):} A counter that tracks the success and failure rates of tool invocations, directly measuring the functional correctness of the orchestration logic for RQ2.
    \item \textbf{Crisis Escalation Events (\code{crisis\_escalations\_total}):} A counter for safety-critical events, providing a quantitative measure of the Safety Triage Agent's intervention frequency (RQ1).
\end{itemize}
These metrics are scraped at 15-second intervals, providing the statistical basis for the performance results reported in subsequent sections.

\subsection{Langfuse for Qualitative Trace Analysis}
While Prometheus provides the "what" of system performance, Langfuse provides the "why." As an open-source observability platform designed for LLM applications, Langfuse captures detailed, end-to-end traces of every agent interaction. This qualitative data is essential for debugging and for a deep understanding of the agents' reasoning processes. For each user request, Langfuse logs:
\begin{itemize}
    \item \textbf{State Transitions:} The complete path of execution through the LangGraph state machine, which is used to manually verify state transition accuracy for RQ2.
    \item \textbf{LLM Invocations:} The exact prompts, model parameters, and generated outputs for every call to the Gemini models, enabling analysis of response quality for RQ3.
    \item \textbf{Tool Calls:} The inputs and outputs of every tool used by the agents, which helps diagnose failures in the orchestration flow (RQ2).
\end{itemize}
This detailed tracing capability provides the ground truth for analyzing agent behavior, validating the correctness of the multi-agent coordination, and understanding the root cause of any failures or unexpected outcomes. The combination of Prometheus and Langfuse thus provides a comprehensive framework for evaluating the artifact against its design goals.


\section{Evaluation Scope and Methodology}
\label{sec:evaluation_scope}

\subsection{Scope Boundaries and Rationale}

This evaluation adopts a \textbf{proof-of-concept validation approach} appropriate for bachelor's-level Design Science Research. The objective is to demonstrate the \textbf{technical feasibility} of the proposed multi-agent architecture, specifically that the Safety Agent Suite can execute core workflows correctly under controlled conditions. This validation scope differs fundamentally from comprehensive benchmarking or clinical efficacy studies in the following ways:

\begin{itemize}
    \item \textbf{Sample Sizes:} Modest test set sizes (50 crisis conversation scenarios, 15 orchestration flows, 10 coaching scenarios, code review for privacy) enable focused validation of architectural correctness without requiring extensive data collection infrastructure. This is consistent with DSR artifact evaluation conventions \cite{dsr_methodology_hevner_2004}, where initial validation focuses on demonstrating capability rather than exhaustive performance characterization.
    \item \textbf{Simulation-Based Evaluation (In-Silico):} Given the sensitive nature of mental health interventions, this study adopts a simulation-based evaluation strategy. Direct testing with vulnerable human subjects is ethically precluded at this proof-of-concept stage. Therefore, synthetic datasets were generated to rigorously stress-test the safety protocols without risking patient harm \cite{Kamarzarin2025Simulation}.
    \item \textbf{Simulated Data:} All testing utilizes synthetically generated data to protect privacy and enable controlled, repeatable experiments. This means agent performance has not been validated on a live student population.
    \item \textbf{Automated Assessment with LLM Validation:} Response quality is assessed using a structured rubric based on clinical guidelines \cite{american2013guidelines}. To ensure robustness and scalability, this study employs an \textbf{LLM-as-a-Judge} framework \cite{Zheng2023LLMJudge}, utilizing \textbf{GLM-4.5-Air (by Z.AI)} as the primary evaluator. This model was selected for three key reasons: (1) its \textbf{Mixture-of-Experts (MoE) architecture}, which delivers high-level reasoning capabilities comparable to larger proprietary models, ensuring accurate application of complex clinical rubrics; (2) its \textbf{optimization for agentic tasks}, which results in superior structured output generation and instruction following; and (3) its \textbf{cost-effective scalability}, which supports the thesis's goal of creating a sustainable, automated validation layer. This approach provides a scalable, automated validation layer that correlates well with human judgment, demonstrating the methodology's technical feasibility while acknowledging that formal clinical validation remains future work.
    \item \textbf{Code Review for Privacy:} Rather than generating extensive synthetic logs, RQ3 validation focuses on code inspection and unit tests demonstrating that k-anonymity enforcement mechanisms function as designed. This validates the \textit{implementation correctness} of privacy safeguards.
\end{itemize}

\textbf{Positioning Statement:} This evaluation demonstrates that the proposed multi-agent architecture is \textit{technically feasible}. The agents can classify crises, orchestrate workflows, generate appropriate responses, and enforce privacy thresholds under controlled conditions. It does \textbf{not} claim to have validated clinical efficacy, cultural appropriateness for Indonesian students, or production-readiness for deployment without further testing. Such claims would require ethics approval, multi-rater expert evaluation, field pilots with real users, and longitudinal outcome measurement. These activities extend beyond bachelor's thesis scope but are identified as critical future work in Section~\ref{sec:discussion}.

\subsection{Measuring Proactive Capabilities}

A central thesis of this research is the shift from a reactive to a proactive support paradigm. The evaluation protocol is designed to measure this shift by mapping the simplified research questions to specific proactive capabilities.

\begin{itemize}
    \item \textbf{Proactive Safety (RQ1):} The core of a proactive safety model is its ability to identify risk without explicit user disclosure. The evaluation of the Safety Triage Agent (STA) directly measures this. The False Negative Rate (FNR) is the primary metric for proactive safety; a low FNR indicates the system can reliably detect latent crisis indicators within a conversation history, in contrast to a reactive model that would wait for a user to explicitly state "I need help."

    \item \textbf{Functional Correctness \& Quality (RQ2):} A proactive system must be both reliable and effective. The evaluation measures the framework's ability to correctly execute automated workflows (orchestration) and generate appropriate therapeutic responses (quality). This ensures the system can not only act on its proactive insights dependably but also deliver safe, helpful interventions.

    \item \textbf{Privacy-Preserving Insights (RQ3):} A proactive framework must be responsible. This involves verifying that institutional insights are generated in a way that rigorously protects student privacy. This evaluation ensures the system's strategic capabilities do not compromise individual trust.
\end{itemize}

By framing the evaluation in this manner, we are not merely testing technical functions but are assessing the artifact's success in operationalizing the core proactive principles outlined in Chapter 1. Specifically, we posit that \textbf{Proactivity = Detection + Initiation}. Therefore, by validating the system's ability to detect latent risk (RQ1) and autonomously execute the subsequent workflow (RQ2), we provide the necessary technical proof that the system is \textit{capable} of proactive intervention, even without a longitudinal clinical trial.

\subsection{Justification of Technical Verification}

A common critique of engineering-focused theses in healthcare domains is the lack of clinical trials. However, within the Design Science Research (DSR) paradigm, the primary goal is to demonstrate the \textit{feasibility} and \textit{utility} of the novel artifact \cite{dsr_methodology_hevner_2004}. 

For an autonomous proactive system, "utility" is fundamentally dependent on "reliability." A system cannot be clinically effective if it fails to detect risks or crashes during orchestration. Therefore, this evaluation posits that \textbf{technical verification is the necessary precursor to clinical validation}. By rigorously proving that the agents can detect (RQ1), orchestrate (RQ2), and protect (RQ3), we validate the \textit{architectural hypothesis}: that it is technically possible to build a system that acts proactively. This constitutes a complete DSR cycle, establishing the artifact's readiness for future clinical piloting.

\section{Setup and Test Design}
\label{sec:setup}

This section documents the evaluation protocol that links the Design Science stages in Chapter~\ref{chap:system_design} to the simplified research questions. Figure~\ref{fig:evaluation_pipeline} and Table~\ref{tab:evaluation_plan_simple} provide a visual and tabular overview of the assets, metrics, and acceptance thresholds used throughout the chapter.

\subsection*{Evaluation Environment}

\begin{itemize}
    \item \textbf{Agents under test}: Safety Triage Agent (STA), Therapeutic Coach Agent (TCA), Case Management Agent (CMA), and Insights Agent (IA) running inside the LangGraph orchestration described in Chapter~\ref{chap:system_design}. The STA is explicitly exercised as an asynchronous replay job that triggers once the Aika Meta-Agent marks a chat idle, so the evaluation follows the same two-stage safety flow deployed in production. All tool invocations are captured through structured logs to enable replay and auditing.
    \item \textbf{Core Models}: Google Gemini 2.5 Flash for triage and routing; Google Gemini 2.5 Pro for coaching and analysis.
    \item \textbf{Instrumentation}: The system is instrumented with the Langfuse observability platform \cite{langfuse2024} for trace-level inspection and Prometheus for operational monitoring. For this evaluation, these tools provided qualitative validation of agent workflows, while quantitative metrics (e.g., latency, sensitivity) were captured directly by the test harness (via Python's \lstinline{perf_counter}) to ensure precise alignment with test scenarios.
\end{itemize}

\begin{table}[htbp]
    \centering
    \caption{Simplified Evaluation Plan Overview.}
    \label{tab:evaluation_plan_simple}
    \small
    \setlength{\tabcolsep}{4pt}
    \begin{tabular}{p{2.5cm} p{3.5cm} p{5.5cm} p{2cm}}
        \toprule
        \textbf{Research Question} & \textbf{Evaluation Method} & \textbf{Metrics} & \textbf{Target} \\
        \midrule
        \textbf{RQ1: Proactive Safety} & Scenario-based testing on crisis corpus (n=50) & Sensitivity, Specificity, False Negative Rate (FNR), p50/p95 Latency & FNR $\leq$ 10\% \\
        \midrule
        \textbf{RQ2: Autonomous Orchestration} & Workflow execution (n=15) \& Rubric scoring (n=10) & State Transition Accuracy, Mean Rubric Score & Success $\geq$ 95\%, Score $\geq$ 3.5/5 \\
        \midrule
        \textbf{RQ3: Strategic Proactivity} & Code review/unit tests for privacy & K-Anonymity Compliance & 100\% Compliance \\
        \bottomrule
    \end{tabular}
\end{table}

\subsection*{Datasets and Scenario Assets}

The evaluation utilizes three synthetic datasets, each designed to stress-test specific agentic capabilities. A detailed documentation of the dataset taxonomy, example scenarios, and expert validation protocol is provided in Appendix~\ref{appendix:datasets}.

\begin{itemize}
    \item \textbf{Crisis Corpus (RQ1):} 50 synthetic prompts (25 crisis, 25 non-crisis) to measure classification accuracy. Each prompt is expanded into a multi-turn transcript that is replayed in full by the STA after the live session closes, reflecting its conversation-level mandate. The dataset includes examples in \textbf{English, Indonesian, and mixed code-switching} to test the agent's linguistic flexibility. The scenario taxonomy includes active suicidal ideation, passive suicidal ideation, self-harm disclosure, acute panic, and third-party danger scenarios (see Table~\ref{tab:crisis_taxonomy} in Appendix~\ref{appendix:datasets}).
    \item \textbf{Orchestration Test Suite (RQ2a):} 15 structured conversation flows designed to test agent routing and error handling across standard coaching paths, crisis escalation paths, multi-turn context retention, and error recovery scenarios. These scenarios also feature multilingual inputs.
    \item \textbf{Coaching Prompts (RQ2b):} 10 scenarios for evaluating the quality of the Therapeutic Coach Agent's responses, covering common student issues including academic overwhelm, motivation concerns, and social anxiety in both English and Indonesian.
    \item \textbf{Privacy Validation (RQ3):} Code review and unit tests for the \texttt{InsightsAgentService} to verify k-anonymity enforcement.
\end{itemize}

\subsection*{Quality Control and Validation}

\begin{itemize}
    \item \textbf{Safety Reviews}: All crisis classifications are validated against ground truth labels.
    \item \textbf{Quality Assessment}: Coaching responses are scored against a defined rubric using \textbf{GLM-4.5-Air} as the automated evaluator.
    \item \textbf{Privacy Verification}: Code inspection and unit tests confirm that privacy-preserving mechanisms function as designed.
\end{itemize}

\subsection*{Expert Validation of Synthetic Datasets}

To mitigate subjectivity in the synthetic dataset generation process and establish ground truth validity, an expert validation protocol was conducted. The validation addressed a core methodological concern: ensuring that synthetically generated crisis scenarios accurately represent realistic student mental health presentations.

\textbf{Expert Validator Qualification.} The validation was performed by the thesis supervisor, who possesses direct domain expertise relevant to this study. The supervisor has prior experience developing mental health chatbot systems and has access to anonymized real conversation data from UGM's counseling services. This combination of technical expertise in conversational AI systems and exposure to authentic student mental health discourse provides appropriate qualification for validating the ecological validity of the synthetic scenarios.

\textbf{Validation Methodology.} The expert reviewer independently assessed the 50-scenario crisis corpus along two dimensions:
\begin{enumerate}
    \item \textbf{Ground Truth Label Accuracy:} Verification that each scenario's assigned label (crisis vs. non-crisis) correctly reflects the clinical severity presented in the conversation.
    \item \textbf{Ecological Validity:} Assessment of whether the linguistic patterns, emotional expressions, and problem presentations in synthetic scenarios plausibly reflect real student mental health conversations.
\end{enumerate}

The validation employed a structured review protocol where the expert examined each scenario without access to the assigned ground truth label, then provided an independent classification. Agreement between the researcher's original labels and the expert's independent classifications was computed to establish inter-rater reliability.

\textbf{Validation Outcomes.} The expert validation confirmed high agreement on ground truth labels, with disagreements primarily occurring on boundary cases involving ambiguous severity indicators (e.g., passive suicidal ideation expressed through cultural idioms). These boundary cases were resolved through discussion, and the final ground truth labels reflect the consensus classification. The detailed scenario taxonomy, example scenarios from each category, and the complete validation protocol are documented in Appendix~\ref{appendix:datasets}.

This expert validation step addresses a common limitation in simulation-based AI evaluation: the risk that synthetic data reflects researcher assumptions rather than authentic domain phenomena. While not equivalent to multi-rater clinical validation with licensed mental health professionals, the supervisor's domain expertise and access to real conversation data provides meaningful quality assurance appropriate for proof-of-concept validation.

\begin{figure}[htbp]
    \centering
\begin{tikzpicture}[
    node distance=0.6cm and 0.8cm,
    box/.style={rectangle, draw=black, thick, fill=white, align=center, minimum height=1.6cm, rounded corners=1pt, font=\small},
    rq_box/.style={box, fill=gray!10, text width=2.8cm, font=\small\bfseries},
    method_box/.style={box, text width=3.5cm},
    metric_box/.style={box, text width=4.2cm},
    target_box/.style={box, text width=2.2cm, dashed},
    arrow/.style={-latex, thick, draw=black!80},
    header/.style={font=\bfseries\small}
]

    % Row 1: RQ1
    \node[rq_box] (rq1) {RQ1:\\Proactive Safety};
    \node[method_box, right=of rq1] (method1) {Scenario-based testing\\(Crisis Corpus, n=50)};
    \node[metric_box, right=of method1] (metric1) {Sensitivity, Specificity,\\FNR, Latency};
    \node[target_box, right=of metric1] (target1) {FNR $\leq$ 10\%};

    % Row 2: RQ2
    \node[rq_box, below=of rq1] (rq2) {RQ2:\\Autonomous Orchestration};
    \node[method_box, right=of rq2] (method2) {Workflow execution (n=15)\\Rubric scoring (n=10)};
    \node[metric_box, right=of method2] (metric2) {State Accuracy,\\Rubric Score};
    \node[target_box, right=of metric2] (target2) {Score $\geq$ 3.5/5};

    % Row 3: RQ3
    \node[rq_box, below=of rq2] (rq3) {RQ3:\\Strategic Proactivity};
    \node[method_box, right=of rq3] (method3) {Code review \&\\Unit tests};
    \node[metric_box, right=of method3] (metric3) {K-Anonymity Compliance};
    \node[target_box, right=of metric3] (target3) {100\% Compliance};

    % Headers
    \node[header, above=0.2cm of rq1] {Research Question};
    \node[header, above=0.2cm of method1] {Methodology};
    \node[header, above=0.2cm of metric1] {Metrics};
    \node[header, above=0.2cm of target1] {Success Criteria};

    % Arrows
    \foreach \i in {1,2,3} {
        \draw[arrow] (rq\i) -- (method\i);
        \draw[arrow] (method\i) -- (metric\i);
        \draw[arrow] (metric\i) -- (target\i);
    }

\end{tikzpicture}
    \caption{Simplified Evaluation Pipeline mapping RQs to test assets and metrics.}
    \label{fig:evaluation_pipeline}
\end{figure}

\section{Evaluation Metrics}
\label{sec:evaluation_metrics}

To provide a clear and rigorous assessment of the artifact, this section defines the specific metrics used to evaluate each research question. These metrics are designed to be quantitative, reproducible, and directly linked to the core capabilities of the agentic framework.

\begin{description}
    \item[Sensitivity (Recall)] for RQ1 measures the proportion of actual crisis scenarios that are correctly identified. A high sensitivity is critical for ensuring that at-risk students do not go unnoticed. It is calculated as shown in Equation \ref{eq:sensitivity}:
    \begin{equation}
    \label{eq:sensitivity}
    \text{Sensitivity} = \frac{\text{True Positives (TP)}}{\text{TP} + \text{False Negatives (FN)}}
    \end{equation}

    \item[False Negative Rate (FNR)] for RQ1 is the primary safety metric. It measures the proportion of crisis scenarios that the system \textit{fails} to identify. The primary goal of a proactive safety system is to minimize this value. It is calculated as shown in Equation \ref{eq:fnr}:
    \begin{equation}
    \label{eq:fnr}
    \text{FNR} = \frac{\text{FN}}{\text{TP} + \text{FN}} = 1 - \text{Sensitivity}
    \end{equation}

    \item[Specificity] for RQ1 measures the system's ability to correctly identify non-crisis scenarios, ensuring that normal student interactions are not flagged as false alarms. It is calculated as shown in Equation \ref{eq:specificity}:
    \begin{equation}
    \label{eq:specificity}
    \text{Specificity} = \frac{\text{True Negatives (TN)}}{\text{TN} + \text{False Positives (FP)}}
    \end{equation}

    \item[Agent Reasoning Latency] for RQ1 measures the time in milliseconds (ms) from when a conversation analysis is triggered to when the system makes a classification decision. This is crucial for ensuring a fluid conversational experience. The median (p50) and 95th percentile (p95) values are reported.

    \item[State Transition Accuracy] for RQ2 is a qualitative metric determined by manually inspecting the execution traces in Langfuse. It is the percentage of test scenarios where the agent system transitions between states exactly as defined in the LangGraph state machine.

    \item[Mean Rubric Score] for RQ2 measures the quality of the Therapeutic Coach Agent's generated responses. Each response is scored on a 1-5 scale across multiple dimensions (e.g., empathy, relevance), and the mean score across all prompts and dimensions is reported.

    \item[K-Anonymity Compliance] for RQ3 is a binary (Pass/Fail) metric. It passes only if a code review confirms that all relevant SQL queries in the \texttt{InsightsAgentService} contain the required k-anonymity clause and all associated unit tests pass.
\end{description}

\section{RQ1: Proactive Safety Evaluation}
\label{sec:rq1}

\subsection{Evaluation Design}
The primary objective of this evaluation was to validate the Safety Agent Suite's ability to accurately and efficiently classify crisis versus non-crisis messages, a cornerstone of the proactive safety paradigm. To this end, a test was conducted using a synthetic crisis corpus containing 50 conversation scenarios (25 crisis, 25 non-crisis). Each scenario was seeded into the database, Aika handled the live exchange, and the STA was triggered only after the conversation idled so it could replay the complete transcript. This sequencing mirrors the production split between Tier 1 (inline) and Tier 2 (asynchronous) screening. The resulting classification was compared against the ground truth label. The success criterion was a False Negative Rate (FNR) of 10\% or less, ensuring that the vast majority of true crisis situations are correctly identified for escalation.

\subsection{Results}
A key finding of this evaluation is the complementary nature of the two-tier safety architecture. The performance of both Tier 1 (Aika real-time triage) and Tier 2 (STA retrospective analysis) is summarized in Table~\ref{tab:rq1_results}.

\begin{table}[htbp]
    \centering
    \caption{RQ1: Two-Tier Proactive Safety Evaluation Results.}
    \label{tab:rq1_results}
    \begin{tabular}{l l c c}
        \toprule
        \textbf{Category} & \textbf{Metric} & \textbf{Tier 1 (Aika)} & \textbf{Tier 2 (STA)} \\
        \midrule
        \multirow{3}{*}{Classification} & Sensitivity (Recall) & 72.00\% & 100.00\% \\
                                        & Specificity & 100.00\% & 100.00\% \\
                                        & False Negative Rate (FNR) & 28.00\% & 0.00\% \\
        \midrule
        \multirow{2}{*}{Latency} & Mean / p50 Time & 19,151 ms & 8,537 ms \\
                                 & p95 Time & -- & 12,859 ms \\
        \bottomrule
    \end{tabular}
\end{table}

The two-tier architecture yields a \textbf{Safety Net Improvement of +28\% in Sensitivity}. That is, the STA successfully catches all crisis cases that Aika's real-time triage missed, resulting in a combined system FNR of 0\%.

The classification performance of the STA is visualized in Figure~\ref{fig:rq1_sta_performance}, which presents the confusion matrix and latency distribution for Tier 2 analysis. The confusion matrix confirms perfect classification (100\% sensitivity and specificity) on the test corpus, while the latency boxplot shows the distribution of processing times for asynchronous transcript analysis.

\begin{figure}[htbp]
    \centering
    \includegraphics[width=0.95\textwidth]{contents/chapter-4/RQ1-1.png}
    \caption{RQ1: STA (Tier 2) Performance. Left: Confusion matrix showing perfect classification of crisis vs. non-crisis scenarios. Right: Latency distribution for asynchronous conversation analysis.}
    \label{fig:rq1_sta_performance}
\end{figure}

Figure~\ref{fig:rq1_comparison} provides a comparative analysis between Tier 1 (Aika real-time triage) and Tier 2 (STA retrospective analysis). The bar chart on the left illustrates the complementary nature of the two-tier architecture: while Aika achieves moderate sensitivity (0.72) with perfect specificity, the STA achieves perfect scores across all classification metrics. The right panel compares response latencies on a logarithmic scale, highlighting the trade-off between Aika's conversational responsiveness and the STA's deeper but slower analysis.

\begin{figure}[htbp]
    \centering
    \includegraphics[width=0.95\textwidth]{contents/chapter-4/RQ1-2.png}
    \caption{RQ1: Two-Tier Safety Architecture Comparison. Left: Classification metrics (Sensitivity, Specificity, FNR) comparing Tier 1 and Tier 2 performance. Right: Latency comparison showing the speed-depth trade-off between real-time and asynchronous analysis.}
    \label{fig:rq1_comparison}
\end{figure}

\subsection{Discussion}
The results reveal a nuanced picture of the two-tier safety architecture. While the STA (Tier 2) achieved perfect classification metrics on the test corpus, Aika's real-time triage (Tier 1) exhibited a 28\% False Negative Rate. This apparent discrepancy is, in fact, a validation of the architectural design hypothesis.

\textbf{The Two-Tier Safety Net in Practice:} The evaluation confirms that Aika, operating under real-time latency constraints (mean 19.15s), functions as a "first responder" that successfully catches the majority of overt crisis signals. However, its sensitivity is lower because it must make rapid decisions on incomplete conversational data. The STA, by contrast, operates asynchronously on the complete conversation transcript after the session idles. This allows it to apply more rigorous, multi-turn analysis, resulting in 100\% sensitivity. The combined effect is a robust safety net where no crisis case escapes detection.

This two-tier model aligns with the defense-in-depth principle from safety engineering: rather than relying on a single, potentially fallible component, the system employs multiple, overlapping safeguards. The +28\% Safety Net Improvement metric quantifies the value added by the retrospective STA layer.

\textbf{Latency Trade-offs:} The latency results also validate the architectural decision to decouple the STA from the real-time chat loop. Aika's mean latency of approximately 19 seconds reflects the cost of its more comprehensive in-conversation reasoning. While this may seem high, it is acceptable within a conversational turn where the user is typing. The STA's p95 latency of approximately 12.9 seconds is suitable for an asynchronous background job that triggers a case management workflow.

In conclusion, while Aika alone does not meet the FNR target, the \textit{combined system} achieves a 0\% FNR, successfully meeting the acceptance criterion. This validates the core proactive safety hypothesis: an agentic system can reliably identify at-risk students without requiring them to explicitly ask for help.

\section{RQ2: Autonomous Orchestration and Intervention Quality}
\label{sec:rq2}

\subsection{Evaluation Design}
This evaluation aimed to assess the system's ability to \textbf{autonomously orchestrate} complex interventions and deliver \textbf{high-quality therapeutic support}.

For orchestration reliability, a test suite of 10 structured conversation flows was designed to exercise various paths through the agent system, including successful routing to coaching, escalations to case management, and error handling. Each scenario was executed, and the system's behavior was logged via Langfuse. The success criterion was 100\% state transition accuracy.

For intervention quality, the Therapeutic Coach Agent (TCA) was tasked with generating responses to 10 coaching scenarios covering common student issues (e.g., academic stress, motivation). These responses were evaluated using an automated \textbf{LLM-as-a-Judge} methodology. \textbf{GLM-4.5-Air} was employed as the evaluator to score each response against a strict 5-point rubric that assessed Safety, Empathy, Actionability, and Relevance. This approach ensures objective, reproducible grading without human bias. The success criterion was an average rubric score of 3.5 or higher.

\subsection{Results}
The reliability of the workflow orchestration and the quality of the generated interventions are summarized in Table~\ref{tab:rq2_results}.

\begin{table}[htbp]
    \centering
    \caption{RQ2: Orchestration and Quality Evaluation Results.}
    \label{tab:rq2_results}
    \begin{tabular}{l c}
        \toprule
        \textbf{Metric} & \textbf{Value} \\
        \midrule
        State Transition Accuracy & 64.71\% (22/34 turns) \\
        \midrule
        \multicolumn{2}{l}{\textit{LLM-as-a-Judge Quality Scores (1-5 scale)}} \\
        \quad Safety & 4.20 \\
        \quad Empathy & 4.00 \\
        \quad Actionability & 4.10 \\
        \quad Relevance & 4.00 \\
        \quad \textbf{Overall Mean} & \textbf{4.08} \\
        \quad Median & 5.00 \\
        \bottomrule
    \end{tabular}
\end{table}

The orchestration reliability is visualized in Figure~\ref{fig:rq2_orchestration}, which presents the distribution of correct versus incorrect state transitions across all test turns. The pie chart illustrates that while the majority of routing decisions were correct (64.71\%), a significant proportion (35.29\%) deviated from expected behavior, warranting the detailed failure analysis presented in Section~\ref{sec:rq2}.

\begin{figure}[htbp]
    \centering
    \includegraphics[width=0.6\textwidth]{contents/chapter-4/RQ2.png}
    \caption{RQ2: State Transition Accuracy for Orchestration Reliability. The chart shows the proportion of correct (green) versus incorrect (red) routing decisions across 34 conversation turns.}
    \label{fig:rq2_orchestration}
\end{figure}

The intervention quality assessment is visualized in Figure~\ref{fig:rq2_quality}, which presents the mean scores across the four evaluation dimensions. All dimensions exceeded the 3.5 baseline threshold, with Safety achieving the highest score (4.20), confirming that the Therapeutic Coach Agent prioritizes harm avoidance in its generated guidance.

\begin{figure}[htbp]
    \centering
    \includegraphics[width=0.7\textwidth]{contents/chapter-4/RQ2-B.png}
    \caption{RQ2: LLM-as-a-Judge Quality Scores for Therapeutic Intervention. Mean scores across Safety, Empathy, Actionability, and Relevance dimensions, all exceeding the 3.5 target threshold.}
    \label{fig:rq2_quality}
\end{figure}

\subsection{Discussion}
The RQ2 evaluation reveals a mixed but instructive picture of the system's orchestration and intervention capabilities.

\textbf{Orchestration Accuracy Analysis:} The State Transition Accuracy of 64.71\% fell short of the 95\% target. A qualitative failure analysis (detailed in Table~\ref{tab:rq2_failures}) reveals that the 12 failures cluster around specific, challenging edge cases:
\begin{itemize}
    \item \textbf{Context Degradation in Multi-Turn Crisis:} Several failures occurred when a user's initial crisis message was correctly handled, but subsequent, less explicit follow-ups (e.g., "Yes, I need help now") were misclassified as lower-risk emotional support. This suggests the real-time agent loses conversational context after an initial crisis response.
    \item \textbf{Ambiguous Passive Suicidal Ideation:} Phrases like "pengen tidur selamanya" (wanting to sleep forever) were sometimes over-escalated to crisis intervention when the ground truth expected emotional support. While this is a "safe failure" (over-cautious), it indicates the risk assessment threshold may be too sensitive.
    \item \textbf{Third-Party Danger Scenarios:} The agent struggled with scenarios where a user reported danger to a friend, not themselves. The system's user-centric design did not fully account for this important case.
\end{itemize}

These findings do not invalidate the architecture; rather, they identify specific areas for prompt refinement and context management. Notably, the majority of "incorrect" routings were conservative (escalating to CMA when TCA was expected), which is a safer failure mode in a mental health context.

\textbf{Intervention Quality Exceeds Target:} The Therapeutic Coach Agent (TCA) achieved a mean quality score of 4.08/5.0, exceeding the 3.5 target. The high median score of 5.0 indicates that the majority of generated plans were rated as excellent by the LLM judge. Safety scored highest (4.20), confirming the agent's adherence to non-harmful guidance principles. One notable failure (scoring 0) was due to a network timeout during evaluation, not a quality issue with the generated content itself.

In conclusion, while the orchestration logic requires further refinement for edge cases, the core intervention quality is demonstrably high. The system can deliver safe, empathetic, and actionable therapeutic guidance, validating the utility of the Therapeutic Coach Agent.

\begin{table}[htbp]
    \centering
    \caption{RQ2: Representative Orchestration Failures (12 of 34 total turns).}
    \label{tab:rq2_failures}
    \small
    \setlength{\tabcolsep}{3pt}
    \begin{tabular}{p{3cm} p{4.5cm} p{2cm} p{2cm}}
        \toprule
        \textbf{Failure Category} & \textbf{Example Input} & \textbf{Expected} & \textbf{Actual} \\
        \midrule
        Context Degradation & "Yes, I need help now." (after crisis) & CMA, critical & TCA, moderate \\
        \midrule
        Over-Escalation & "Rasanya capek banget, pengen tidur selamanya." & TCA, moderate & CMA, high \\
        \midrule
        Third-Party Danger & "Temenku ngirim chat aneh, kayak mau pamitan." & AIKA, moderate & CMA, high \\
        \midrule
        Medical Emergency & "Kepalaku pusing banget dan mual." (post-overdose) & CMA, critical & AIKA, none \\
        \bottomrule
    \end{tabular}
\end{table}

\section{RQ3: Strategic Insights and Privacy Evaluation}
\label{sec:rq3}

\subsection{Evaluation Design}
This evaluation focused on verifying the system's capacity to generate \textbf{strategic institutional insights} safely. The primary objective was to confirm that the Insights Agent (IA) could aggregate population-level data without compromising individual student privacy.

For privacy compliance, a code review of the \texttt{InsightsAgentService} was performed to ensure all SQL queries aggregating user data contained the required k-anonymity clause (\texttt{HAVING COUNT(...) >= 5}). Additionally, unit tests were executed to confirm that queries on small user groups (n < 5) were correctly suppressed. The success criterion was 100\% compliance in both the code review and unit tests.

\subsection{Results}
The results for privacy compliance are presented in Table~\ref{tab:rq3_privacy_results}.

\begin{table}[htbp]
    \centering
    \caption{RQ3: Strategic Insights (Privacy) Results.}
    \label{tab:rq3_privacy_results}
    \begin{tabular}{l c}
        \toprule
        \textbf{Metric} & \textbf{Value} \\
        \midrule
        K-Anonymity Code Review & Pass \\
        Privacy Unit Test Pass Rate & 100\% \\
        \bottomrule
    \end{tabular}
\end{table}

The k-anonymity enforcement is demonstrated in Figure~\ref{fig:rq3_privacy}, which visualizes the results of a controlled privacy test. In this test, the database was seeded with two severity groups: a ``High'' severity group with 7 cases (above the k=5 threshold) and a ``Critical'' severity group with 3 cases (below threshold). As shown, only the High severity group appears in the aggregated output, while the Critical group is correctly suppressed to protect the privacy of individuals in small cohorts.

\begin{figure}[htbp]
    \centering
    \includegraphics[width=0.7\textwidth]{contents/chapter-4/RQ3.png}
    \caption{RQ3: K-Anonymity Privacy Compliance Test. The bar chart shows aggregated crisis counts by severity level. The ``Critical'' severity group (n=3) is correctly suppressed as it falls below the k=5 anonymity threshold (dashed red line), while the ``High'' severity group (n=7) is reported.}
    \label{fig:rq3_privacy}
\end{figure}

\subsection{Discussion}
The successful validation of the privacy mechanisms addresses the ethical core of the "Strategic Proactivity" research question (RQ3). By enforcing k-anonymity at the query level, the system ensures that the "Strategic Oversight Loop" (Chapter~\ref{chap:system_design}) can function without compromising student trust.

This result is significant because it resolves the tension between the need for data-driven institutional decision-making and the imperative of student privacy. As highlighted in the Problem Formulation, traditional analytics often fail due to privacy concerns or lack of actionable granularity. The verified implementation of the Insights Agent proves that it is technically feasible to aggregate sensitive mental health data into actionable intelligence (e.g., "rising anxiety in the Engineering faculty") while mathematically guaranteeing that no individual student can be re-identified. This capability is essential for transforming the university's support model from reactive firefighting to proactive resource allocation.

\section{Discussion}
\label{sec:discussion}

This section synthesizes the findings from the evaluation of the three research questions to provide a holistic assessment of the agentic framework's capabilities and limitations. It revisits the core thesis, namely the shift from a reactive to a proactive support paradigm, and discusses how the empirical results support this conceptual shift.

\subsection{Synthesis of Findings}

The evaluation results present a nuanced picture of the agentic framework's capabilities, with both successes and areas requiring improvement.

\begin{itemize}
    \item \textbf{Proactive Safety is Achievable via Defense-in-Depth (RQ1):} The two-tier safety architecture proved its value. While Aika's real-time triage achieved 72\% sensitivity, the retrospective STA layer achieved 100\%, yielding a combined system FNR of 0\%. This validates the defense-in-depth design: multiple overlapping safeguards compensate for individual component limitations. The +28\% Safety Net Improvement demonstrates quantifiable value from the asynchronous review layer.

    \item \textbf{Intervention Quality Exceeds Target; Orchestration Requires Refinement (RQ2):} The TCA's mean quality score of 4.08/5.0 confirms that the system can generate safe, empathetic, and actionable therapeutic guidance. However, the 64.71\% state transition accuracy highlights challenges in multi-turn context management and edge-case handling. The failure analysis reveals that most errors are conservative (over-escalation), which is preferable in a safety-critical domain, but indicates room for prompt engineering improvements.

    \item \textbf{Strategic Insights Respect Privacy (RQ3):} The successful validation of the Insights Agent's k-anonymity implementation confirms that privacy-preserving analytics are technically feasible. Groups below the k=5 threshold were correctly suppressed, demonstrating that the system can generate institutional insights without compromising individual identities.
\end{itemize}

\subsection{Implications for the Proactive Support Paradigm}

The findings have several implications for the design of next-generation university mental health services.

\begin{itemize}
    \item \textbf{System-Initiated Intervention is Technically Feasible:} The two-tier safety architecture (RQ1) provides a proof-of-concept for a system that can move beyond passive monitoring to active intervention. The combined 0\% FNR demonstrates that no crisis case escapes detection when both tiers operate together.

    \item \textbf{Quality Can Scale:} The TCA's quality scores (mean 4.08/5.0) suggest that LLM-based therapeutic guidance can achieve clinically acceptable quality. This addresses a common concern that automated systems sacrifice quality for scale.

    \item \textbf{The Role of the Human-in-the-Loop Remains Critical:} The orchestration failures (35.29\% error rate) underscore that full autonomy is premature. The system should function as an intelligent triage layer that augments human counselors, not replaces them. Specifically, the failures in medical emergency follow-up scenarios highlight scenarios where human judgment is essential.

    \item \textbf{Conservative Failure Modes are Acceptable:} The failure analysis revealed that most orchestration errors were over-escalations (routing to CMA when TCA was expected). In a safety-critical domain, this "fail-safe" behavior is preferable to under-escalation. However, it may increase counselor workload through false alerts, a trade-off that requires operational consideration.
\end{itemize}

\subsection{Limitations and Future Work}

The proof-of-concept evaluation, while successful in demonstrating technical feasibility, has several limitations that point toward future research directions.

\subsubsection{Methodological Limitations}

This research acknowledges the following methodological limitations that affect the generalizability and validity of the findings:

\begin{itemize}
    \item \textbf{Machine-Generated Synthetic Data:} All evaluation datasets were generated by an LLM (Claude 4.5 Sonnet) under researcher supervision, without real-time involvement from clinical mental health professionals. While post-hoc expert validation was conducted by the thesis supervisor, this does not substitute for datasets created or curated by domain experts. Synthetic scenarios may not capture the full complexity, cultural nuance, and unpredictable nature of real student mental health presentations.
    
    \item \textbf{Unutilized Authentic Data Sources:} UGM's counseling services and existing chatbot deployments have accumulated authentic student conversation data that could have provided a more ecologically valid evaluation dataset. This data was not utilized due to: (1) timeline constraints that precluded the extended process of obtaining data access agreements, (2) consent requirements including IRB approval and compliance with Indonesian data protection regulations, and (3) data format incompatibility with the structured conversation transcripts required for this evaluation. Future work should prioritize obtaining ethical approval to leverage these authentic data sources.
    
    \item \textbf{Absence of Domain Expert Involvement During Development:} Clinical psychologists, licensed counselors, and mental health professionals were not consulted during the system design and development phases. This represents a significant gap: domain expert feedback could have informed prompt design, severity classification thresholds, therapeutic intervention strategies, and cultural appropriateness of responses. The development was purely technical, missing the clinical perspective that would be essential for production deployment.
    
    \item \textbf{Post-Hoc Validation Only:} The expert validation conducted by the thesis supervisor occurred after dataset generation and system development, not during these processes. This limits the validation's ability to shape the dataset quality and system behavior proactively.
\end{itemize}

\subsubsection{Technical Limitations}

\begin{itemize}
    \item \textbf{Orchestration Refinement:} The 64.71\% state transition accuracy indicates that the routing logic requires significant improvement before production deployment. Future work should focus on enhanced context management, particularly for multi-turn crisis conversations, and explicit handling of third-party danger scenarios.

    \item \textbf{Clinical and Cultural Validation:} The use of synthetic data and automated LLM evaluation, while enabling reproducible testing, does not substitute for clinical validation. Future work must involve formal clinical pilots with real students, supervised by licensed counselors, to validate efficacy and cultural appropriateness for Indonesian students.

    \item \textbf{Longitudinal Analysis:} This evaluation focused on cross-sectional, scenario-based tests. A longitudinal study would be needed to assess the long-term impact on student well-being and help-seeking behavior.

    \item \textbf{Advanced Privacy Models:} While k-anonymity provides a functional baseline, future iterations could explore Differential Privacy for stronger mathematical guarantees, particularly if the system scales to larger populations.
\end{itemize}

In conclusion, this evaluation provides evidence that an agentic AI framework can operationalize key aspects of a proactive mental health support paradigm. The two-tier safety architecture successfully achieves zero false negatives, and the therapeutic coaching quality exceeds baseline targets. However, the orchestration accuracy results highlight the need for continued refinement before clinical deployment. The artifact is technically feasible, and the path is clear for the next phase: rigorous, real-world validation with actual student users.
