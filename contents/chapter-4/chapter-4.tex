\chapter{Implementation and Evaluation (Hasil dan Pembahasan)}

This chapter reports how the prototype was exercised and what we learned from it. The focus is on the agents and their behavior in safety‑relevant scenarios. We keep the scope practical and transparent so results can be reproduced and audited.

\section{Setup and Test Design (Rancangan Pengujian)}
\label{sec:setup}

\begin{itemize}
  \item \textbf{Agen dan Orkestrasi}: Safety Triage Agent (STA), Support Coach Agent (SCA), Service Desk Agent (SDA), Insights Agent (IA); diorkestrasi dengan pengendali berbasis graf dan pemanggilan alat terstruktur.
  \item \textbf{Data Uji}: \textit{Synthetic} untuk skenario krisis (indikasi self‑harm, bunuh diri, kekerasan), set percakapan untuk coaching (N prompt), dan data log teranonymisasi/sintetik untuk agregasi.
  \item \textbf{Metrik Utama}: sensitivitas/spesifisitas triase, waktu ke eskalasi, tingkat keberhasilan pemanggilan fungsi dan \textit{workflow}, latensi p50/p95, ketahanan terhadap \textit{prompt injection}, dan skor kualitas coaching (rubrik CBT).
  \item \textbf{Pengawasan Manusia}: semua eskalasi berisiko tinggi ditinjau oleh manusia; audit log diaktifkan.
\end{itemize}

\section{RQ1 — Safety: Can STA detect crises promptly?}
\label{sec:rq1}

\textbf{Desain}. Uji pada set krisis sintetis dengan label. Ukur sensitivitas, spesifisitas, dan waktu ke eskalasi dari deteksi awal hingga pembuatan tiket/alert. Analisis khusus pada kegagalan berisiko (\textit{false negatives}).

\textbf{Hasil}. Ringkas angka utama (mis. sensitivitas, spesifisitas, p50/p95 latensi). Tampilkan contoh sukses dan kegagalan yang representatif.

\textbf{Bahasan}. Kompromi antara kecepatan dan kehati‑hatian; peran \textit{guardrail} dan \textit{fallback} manusia.

\section{RQ2 — Reliability: Does orchestration run reliably?}
\label{sec:rq2}

\textbf{Desain}. Telusuri rasio keberhasilan pemanggilan fungsi, validasi skema, \textit{retry/backoff}, dan penyelesaian \textit{workflow} ujung‑ke‑ujung.

\textbf{Hasil}. Laporkan tingkat keberhasilan, tingkat kegagalan yang pulih, dan kasus terhenti (jika ada). Sertakan latensi p50/p95 per langkah.

\textbf{Bahasan}. Pola kegagalan yang paling sering dan perbaikan yang mudah diterapkan.

\section{RQ3 — Quality: Are SCA responses reasonable and CBT‑informed?}
\label{sec:rq3}

\textbf{Desain}. Penilaian buta oleh evaluator pada sampel percakapan (kecil namun beragam). Rubrik menilai kepatuhan CBT dasar, keamanan saran, dan empati.

\textbf{Hasil}. Skor ringkas dan contoh tanggapan baik/kurang baik. Catat penolakan yang tepat pada topik di luar batas.

\textbf{Bahasan}. Pola perbaikan prompt/alat yang berdampak nyata.

\section{RQ4 — Insights (minimal): Can IA produce safe aggregate views?}
\label{sec:rq4}

\textbf{Desain}. Jalur agregasi sederhana: ambang privasi (mis. k‑anonymity) dan cek kestabilan jumlah/topik. Tidak ada klaim level individu.

\textbf{Hasil}. Laporkan hanya \textit{sanity check} agregat (mis. stabil/variatif) dan kepatuhan terhadap ambang privasi.

\textbf{Bahasan}. Keterbatasan desain saat ini dan langkah aman untuk perluasan.

\section{Discussion and Limitations (Diskusi dan Keterbatasan)}
\label{sec:discussion}

\begin{itemize}
  \item \textbf{Temuan utama}. Soroti apa yang berjalan baik (mis. orkestrasi stabil, latensi terkendali) dan apa yang perlu diperkuat (mis. penanganan tepi kasus tertentu).
  \item \textbf{Batasan}. Prototipe, data sintetis/anonym, tidak ada klaim efek klinis; model dapat bias/\textit{hallucinate} meski ada \textit{guardrail}.
  \item \textbf{Implikasi}. Perbaikan sederhana yang memberi dampak besar; rencana evaluasi lanjutan (\textit{field} kecil) dengan pengawasan etik yang memadai.
\end{itemize}

