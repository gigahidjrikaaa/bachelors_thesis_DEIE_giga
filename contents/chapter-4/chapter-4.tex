\chapter{Hasil dan Pembahasan}

Berikut ini adalah yang perlu diperhatikan untuk mengisi bab hasil dan pembahasan:

\begin{enumerate}
	\item Setiap rumusan masalah boleh memiliki lebih dari 1 tujuan.
	\item Setiap subbab harus spesifik menjawab setiap tujuan yang dituliskan.
	\item Setiap rumusan masalah boleh dijawab dengan 1 subbab atau lebih.
\end{enumerate}

Berikut ini adalah contoh sub bab untuk menjelaskan tujuan penelitian.

\section{Pembahasan Tujuan 1 dengan Hasil Penelitian 1 (Ubah Judul Sesuai dengan Hal yang Hendak dibahas)}

Sub bab pertama adalah membahas tujuan penelitian pertama dengan hasil penelitian ke-1. 
Dapat ditambahkan beberapa sub bab jika diperlukan.

\section{Pembahasan Tujuan 1 dengan Hasil Penelitian 2 (Ubah Judul Sesuai dengan Hal yang Hendak dibahas)}

Sub bab kedua adalah membahas tujuan penelitian pertama dengan hasil penelitian ke-2. Sub bab ini merupakan contoh tambahan sub bab pertama.

\section{Pembahasan Tujuan 2 dengan Hasil Penelitian 3 (Ubah Judul Sesuai dengan Hal yang Hendak dibahas)}

Sub bab ketiga adalah membahas tujuan penelitian kedua. Dapat ditambahkan beberapa sub bab jika diperlukan.

\section{Perbandingan Hasil Penelitian dengan Hasil Terdahulu}

Pembahasan penutup dapat menjelaskan mengenai kelebihan hasil pengembangan / 
penelitian dan kekurangan dibandingkan dengan skripsi atau penelitian terdahulu atau
perbandingan terhadap produk lain yang ada di pasaran. Penulis dapat menggunakan tabel untuk membandingkan secara gamblang dan menjelaskannya.