\chapter{System Design and Architecture}
\label{chap:system_design}

%%%%%%%%%%%%%%%%%%%%%%%%%%%%%%%%%%%%%%%%%%%%%%%%%%%%%%
%%% SECTION 3.1 - DSR METHODOLOGY %%%
%%%%%%%%%%%%%%%%%%%%%%%%%%%%%%%%%%%%%%%%%%%%%%%%%%%%%%

\section{Research Methodology: Design Science Research (DSR)}
\label{sec:dsr_methodology}

The research presented in this thesis is constructive in nature, aimed not merely at describing or explaining a phenomenon, but at creating a novel and useful artifact to solve a real-world problem. To provide a rigorous and systematic structure for this endeavor, this study adopts the \textbf{Design Science Research (DSR)} methodology. DSR is a well-established paradigm in Information Systems research focused on the creation and evaluation of innovative IT artifacts intended to solve identified organizational problems \cite{dsr_methodology_hevner_2004}. The primary goal of DSR is to generate prescriptive design knowledge through the building and evaluation of these artifacts.

The DSR process model, as outlined by Peffers et al., provides an iterative framework that guides the research from problem identification to the communication of results \cite{dsr_methodology_peers_2006}. This thesis follows these stages, mapping them directly to its structure to ensure a logical and transparent research process. The complete workflow of this research is visualized in Figure \ref{fig:dsr_flowchart}. This diagram illustrates the iterative path from problem formulation through to the final conclusions and recommendations.

\definecolor{ugmBlue}{RGB}{0,73,144}
\definecolor{ugmGold}{RGB}{217,160,33}
\definecolor{ugmLightBlue}{RGB}{230,240,250}

\begin{figure}[htbp]
    \centering
    \resizebox{\textwidth}{!}{%
    \begin{tikzpicture}[
        node distance=0.8cm,
        % Stage box style
        stagebox/.style={
            rectangle, 
            draw=ugmBlue, 
            line width=1pt, 
            rounded corners=4pt, 
            fill=ugmLightBlue, 
            minimum width=2.4cm, 
            minimum height=2.8cm, 
            align=center,
            font=\footnotesize\bfseries
        },
        % Stage number circle
        stagenum/.style={
            circle, 
            draw=ugmBlue, 
            line width=1pt, 
            fill=ugmBlue, 
            text=white, 
            font=\footnotesize\bfseries,
            minimum size=0.6cm,
            inner sep=0pt
        },
        % Chapter mapping label
        chapterlabel/.style={
            font=\scriptsize\itshape,
            text=ugmBlue!70
        },
        % Main flow arrow
        mainarrow/.style={-{Stealth[length=3mm]}, line width=1.2pt, ugmBlue},
        % Iteration arrow
        iterarrow/.style={-{Stealth[length=2.5mm]}, line width=1pt, ugmGold, dashed},
        % Entry point styles
        entrybox/.style={
            rectangle,
            draw=ugmBlue!60,
            line width=0.8pt,
            rounded corners=2pt,
            fill=white,
            minimum width=1.8cm,
            minimum height=0.5cm,
            align=center,
            font=\tiny
        }
    ]
        % ===== STAGE BOXES =====
        % Stage 1: Problem Identification
        \node[stagebox] (s1) {Problem\\Identification\\[2pt]\&\\Motivation};
        \node[stagenum, above=0.15cm of s1] (n1) {1};
        \node[chapterlabel, below=0.1cm of s1] {Chapter 1};
        
        % Stage 2: Objectives
        \node[stagebox, right=1.2cm of s1] (s2) {
            Define\\Objectives\\[2pt]of a\\Solution
        };
        \node[stagenum, above=0.15cm of s2] (n2) {2};
        \node[chapterlabel, below=0.1cm of s2] {Chapter 1};
        
        % Stage 3: Design & Development
        \node[stagebox, right=1.2cm of s2] (s3) {
            Design\\[2pt]\&\\[2pt]Development
        };
        \node[stagenum, above=0.15cm of s3] (n3) {3};
        \node[chapterlabel, below=0.1cm of s3] {Chapter 3};
        
        % Stage 4: Demonstration
        \node[stagebox, right=1.2cm of s3] (s4) {
            Demonstration
        };
        \node[stagenum, above=0.15cm of s4] (n4) {4};
        \node[chapterlabel, below=0.1cm of s4] {Chapter 4};
        
        % Stage 5: Evaluation
        \node[stagebox, right=1.2cm of s4] (s5) {
            Evaluation
        };
        \node[stagenum, above=0.15cm of s5] (n5) {5};
        \node[chapterlabel, below=0.1cm of s5] {Chapter 4};
        
        % Stage 6: Communication
        \node[stagebox, right=1.2cm of s5] (s6) {
            Communication
        };
        \node[stagenum, above=0.15cm of s6] (n6) {6};
        \node[chapterlabel, below=0.1cm of s6] {Chapter 5};
        
        % ===== MAIN PROCESS FLOW =====
        \draw[mainarrow] (s1) -- (s2);
        \draw[mainarrow] (s2) -- (s3);
        \draw[mainarrow] (s3) -- (s4);
        \draw[mainarrow] (s4) -- (s5);
        \draw[mainarrow] (s5) -- (s6);
        
        % ===== ITERATION ARROWS =====
        % Design iteration back to objectives
        \draw[iterarrow] (s3.south) -- ++(0,-0.6) -| node[pos=0.25, below, font=\tiny, text=ugmGold] {Iterate} (s2.south);
        
        % Evaluation feedback to design
        \draw[iterarrow] (s5.north) -- ++(0,0.8) -| (s3.north);
        
        % ===== ENTRY POINTS (below stages) =====
        \node[entrybox, below=1.0cm of s1] (e1) {Problem-\\Centered};
        \node[entrybox, below=1.0cm of s2] (e2) {Objective-\\Centered};
        \node[entrybox, below=1.0cm of s3] (e3) {Design \&\\Development};
        \node[entrybox, below=1.0cm of s4] (e4) {Client/\\Context};
        
        % Entry point arrows
        \draw[-{Stealth[length=2mm]}, ugmBlue!50, line width=0.6pt] (e1) -- (s1);
        \draw[-{Stealth[length=2mm]}, ugmBlue!50, line width=0.6pt] (e2) -- (s2);
        \draw[-{Stealth[length=2mm]}, ugmBlue!50, line width=0.6pt] (e3) -- (s3);
        \draw[-{Stealth[length=2mm]}, ugmBlue!50, line width=0.6pt] (e4) -- (s4);
        
        % Entry points label
        \node[font=\scriptsize, text=ugmBlue!70, left=0.3cm of e1] {Possible Entry Points:};
        
        % ===== DISCIPLINE KNOWLEDGE ANNOTATIONS =====
        % Top annotation bar
        \draw[ugmBlue!40, line width=0.5pt] ([yshift=1.5cm]s1.north west) -- ([yshift=1.5cm]s6.north east);
        \node[font=\tiny, text=ugmBlue!60, anchor=south] at ([yshift=1.6cm]s1.north) {Literature Review};
        \node[font=\tiny, text=ugmBlue!60, anchor=south] at ([yshift=1.6cm]s3.north) {Artifact Construction};
        \node[font=\tiny, text=ugmBlue!60, anchor=south] at ([yshift=1.6cm]s5.north) {Validation};
        
    \end{tikzpicture}%
    }% End resizebox
    \caption{The Design Science Research (DSR) process model as applied in this thesis, adapted from Peffers et al. \cite{dsr_methodology_peers_2006}. The six stages are shown in sequence with chapter mappings below each stage. Dashed arrows indicate iterative feedback loops between evaluation and design phases. Entry points indicate where different research motivations may initiate the DSR cycle.}
    \label{fig:dsr_flowchart}
\end{figure}

%%%%%%%%%%%%%%%%%%%%%%%%%%%%%%%%%%%%%%%%%%%%%%%%%%%%%%%
%%% SECTION 3.2 - SYSTEM OVERVIEW %%%
%%%%%%%%%%%%%%%%%%%%%%%%%%%%%%%%%%%%%%%%%%%%%%%%%%%%%%%

\section{System Overview and Conceptual Design}

The artifact proposed and developed in this research is a novel agentic AI framework designed to address the systemic inefficiencies of traditional, reactive mental health support models in Higher Education Institutions. The conceptual architecture is predicated on the principles of a Multi-Agent System (MAS), wherein a suite of collaborative, specialized intelligent agents, collectively termed the \textbf{Safety Agent Suite}, work in concert to create a proactive, scalable, and data-driven support ecosystem. This framework is designed not as a monolithic application, but as a dynamic, closed-loop system that operates on two interconnected levels: a micro-level loop for real-time, individual student support and a macro-level loop for strategic, institutional oversight and proactive intervention \cite{kashiv2025aidrivennetworks, nwoke2025insightautomation}.

The system's primary entities and their designated interaction points are illustrated in the conceptual context diagram in Figure \ref{fig:context_diagram}. This diagram shows how all users interact with a single, unified \textbf{Aika Meta-Agent}, which then coordinates the various specialist agents (STA, TCA, CMA, IA) that operate as background services.

\begin{figure}[htbp]
    \centering
    \resizebox{0.8\textwidth}{!}{%
    \begin{tikzpicture}[
        node distance=1.5cm and 1.5cm,
        actor/.style={circle, draw=black, thick, fill=gray!10, minimum size=1.2cm, align=center, font=\small},
        meta/.style={rectangle, draw=ugmGold, very thick, fill=ugmGold!10, rounded corners, minimum width=4cm, minimum height=1.5cm, align=center, font=\bfseries},
        agent/.style={rectangle, draw=ugmBlue, thick, fill=ugmBlue!5, rounded corners, minimum width=2.5cm, minimum height=1cm, align=center, font=\small},
        arrow/.style={-Latex, thick},
        biarrow/.style={Latex-Latex, thick}
    ]
        % Central Meta-Agent
        \node[meta] (aika) {Aika Meta-Agent \\ (Unified Interface)};

        % Users
        \node[actor, above left=of aika] (student) {Student};
        \node[actor, above right=of aika] (admin) {Admin / \\ Counselor};

        % Specialist Agents
        \node[agent, below left=of aika, xshift=-0.5cm] (sta) {Safety Triage \\ Agent (STA)};
        \node[agent, below right=of aika, xshift=0.5cm] (tca) {Therapeutic \\ Coach (TCA)};
        \node[agent, below=of sta] (cma) {Case Mgmt \\ Agent (CMA)};
        \node[agent, below=of tca] (ia) {Insights \\ Agent (IA)};

        % Connections
        \draw[biarrow] (student) -- (aika);
        \draw[biarrow] (admin) -- (aika);
        
        \draw[arrow] (aika) -- (sta);
        \draw[arrow] (aika) -- (tca);
        \draw[arrow] (aika) -- (cma);
        \draw[arrow] (aika) -- (ia);

    \end{tikzpicture}%
    }
    \caption{Conceptual Context Diagram: The Aika Meta-Agent acts as the unified interface for all user roles, orchestrating the background specialist agents.}
    \label{fig:context_diagram}
\end{figure}

\begin{table}[htbp]
    \centering
    \caption{Agent descriptions and their primary roles in the Safety Agent Suite.}
    \label{tab:agent_descriptions}
    \small
    \setlength{\tabcolsep}{4pt}
    \begin{tabular}{p{2.5cm} p{10.5cm}}
        \toprule
        \textbf{Agent} & \textbf{Primary Role} \\
        \midrule
        \textbf{Aika Meta-Agent} & The sole user-facing conversationalist and orchestrator. Manages all user interactions, performs initial risk assessment, and routes tasks to specialist agents. \\
    	\textbf{Safety Triage Agent (STA)} & A background conversation-level assessor. After a chat session goes idle or ends, it replays the full transcript to produce Tier 2 risk labels and compliance artifacts that corroborate (or override) Aika's inline Tier 1 judgment. \\
        \textbf{Therapeutic Coach Agent (TCA)} & A background agent that generates CBT-based intervention plans and recommends resources for the user's dashboard. Does not engage in direct conversation. \\
        \textbf{Case Management Agent (CMA)} & The procedural backbone. Manages administrative tasks like crisis case creation, appointment scheduling, and sending notifications to counselors. \\
        \textbf{Insights Agent (IA)} & The strategic analyst. Processes anonymized, aggregated data to provide population-level well-being trends and insights to administrators. \\
        \bottomrule
    \end{tabular}
\end{table}

Conceptually, the framework's architecture is best understood as two distinct but integrated operational loops:

\begin{enumerate}
    \item \textbf{The Real-Time Interaction Loop:} This loop handles immediate, synchronous interactions with individual students through a unified conversational interface. \textbf{Critically, the Aika Meta-Agent is the sole user-facing component}: students interact exclusively with Aika, never directly accessing the specialist agents. When a student sends a message, Aika processes it via a single Gemini API call that returns a structured JSON response containing immediate risk assessment (Tier 1) and intent classification.
    
    Based on this initial assessment, Aika acts as a supervisor to the background specialist agents:
    \begin{itemize}
        \item \textbf{Therapeutic Coach Agent (TCA):} Triggered directly by Aika when the risk is assessed as \texttt{moderate} or \texttt{low}. It works in the background to generate CBT-based intervention plans.
        \item \textbf{Case Management Agent (CMA):} Triggered directly by Aika when the user explicitly requests administrative actions (e.g., scheduling) or when a critical risk is detected.
        \item \textbf{Safety Triage Agent (STA):} Runs asynchronously in the background after the user becomes inactive or explicitly ends the chat. It reprocesses the entire conversation to confirm the final risk rating, generate the conversation-level STA assessment, and supply evidence for the compliance ledger. It can also be manually invoked by administrators for specific risk checks.
    \end{itemize}
    
    This loop is designed for high-availability and low-latency, ensuring students receive immediate support while complex reasoning occurs asynchronously in the background.
    
    \item \textbf{The Strategic Oversight Loop:} This loop operates on a longer, asynchronous timescale to enable proactive, institution-wide strategy. The \textbf{Insights Agent (IA)} works entirely in the background, periodically analyzing anonymized, aggregated data from all student interactions. However, administrators and counselors can invoke IA through Aika by requesting analytics queries (e.g., "show trending topics this week", "case statistics for November"), at which point Aika routes the request to IA and synthesizes the analytics report into a user-friendly response. IA generates reports on population-level well-being trends, sentiment analysis, and emerging topics of concern, delivered via both scheduled batch processing and on-demand queries through Aika's conversational interface. These insights provide empirical evidence for data-driven resource allocation, such as commissioning new workshops or adjusting counseling staff schedules. This loop directly addresses the "insight-to-action" gap that plagues current systems \cite{nwoke2025insightautomation, jorno2018actionableinsight}.
\end{enumerate}

The synergy between these two loops is the cornerstone of the framework's design. The real-time loop gathers the data that fuels the strategic loop, while the insights from the strategic loop can be used to configure and improve the proactive interventions delivered by the real-time loop, creating a continuously learning and adaptive support ecosystem. This dual-loop architecture is visualized in Figure \ref{fig:two_loops_diagram}, which also highlights that Aika's inline responses and the STA/TCA/CMA queue are deliberately decoupled to keep asynchronous reasoning off the critical path for students.

\begin{figure}[htbp]
    \centering
    \resizebox{0.95\textwidth}{!}{%
    \begin{tikzpicture}[
        node distance=1.5cm and 2cm,
        font=\small,
        % Styles
        loop_box/.style={
            rectangle, draw=black!60, very thick, rounded corners=5pt,
            fill=gray!5, inner sep=15pt,
            label={[font=\bfseries\large, anchor=north west, text=ugmBlue]north west:#1}
        },
        actor/.style={
            circle, draw=black, thick, fill=white,
            minimum size=1.2cm, align=center, drop shadow
        },
        agent/.style={
            rectangle, draw=ugmBlue, thick, fill=ugmBlue!10,
            rounded corners=3pt, minimum width=3cm, minimum height=1.2cm,
            align=center, drop shadow
        },
        data_store/.style={
            cylinder, shape border rotate=90, draw=black, thick,
            aspect=0.25, fill=white, minimum width=2.5cm, minimum height=1.5cm,
            align=center
        },
        arrow/.style={-Latex, thick, draw=black!80},
        async_arrow/.style={-Latex, thick, dashed, draw=black!80},
        feedback_arrow/.style={-Latex, very thick, draw=ugmGold, dashed}
    ]

        % --- Nodes ---
        % Real-Time Loop
        \node[agent] (aika) {Aika Meta-Agent};
        \node[actor, left=3.5cm of aika] (student) {Student};
        \node[agent, below=2.5cm of aika] (bg_agents) {Background Agents\\(STA, TCA, CMA)};

        % Data Store (Bridge)
        \node[data_store, below=2cm of bg_agents] (data) {Aggregated\\Data};

        % Strategic Loop
        \node[agent, below=2cm of data] (ia) {Insights Agent\\(IA)};
        \node[actor, right=3.5cm of ia] (admin) {Admin /\\Counselor};

        % --- Background Layers for Fit Boxes ---
        \begin{scope}[on background layer]
            \node[fit=(student) (aika) (bg_agents), loop_box={The Real-Time Interaction Loop}] (rt_loop) {};
            \node[fit=(ia) (admin), loop_box={The Strategic Oversight Loop}] (strat_loop) {};
        \end{scope}

        % --- Connections: Real-Time Loop ---
        % Student <-> Aika
        \draw[arrow] (student) to[bend left=15] node[midway, above] {Interaction} (aika);
        \draw[arrow] (aika) to[bend left=15] node[midway, below] {Reply} (student);

        % Aika <-> Background Agents
        \draw[async_arrow] (aika.250) -- node[midway, left, align=right, font=\footnotesize] {Async\\Dispatch} (bg_agents.110);
        \draw[async_arrow] (bg_agents.70) -- node[midway, right, align=left, font=\footnotesize] {Evidence\\\& Plans} (aika.290);

        % --- Connections: Strategic Loop ---
        % Admin <-> IA
        \draw[arrow] (admin) to[bend left=15] node[midway, below] {Queries} (ia);
        \draw[arrow] (ia) to[bend left=15] node[midway, above] {Reports} (admin);

        % --- Cross-Loop Connections ---
        % Background -> Data -> IA
        \draw[arrow] (bg_agents) -- node[midway, right] {Fuels} (data);
        \draw[arrow] (data) -- node[midway, right] {Informs} (ia);

        % Feedback: IA -> Aika (The "Improves" loop)
        % Route this carefully to avoid crossing the data store or looking messy
        % Go around the left side
        \draw[feedback_arrow] (ia.west) -- ++(-2,0) |- (aika.west) node[pos=0.25, left, align=right, text=ugmGold] {Strategic\\Insights\\(Improves)};

    \end{tikzpicture}%
    }
    \caption{The Two Proactive Loops: Aika handles synchronous dialogue alone, while the STA/TCA/CMA bundle runs asynchronously in the background to supply evidence and plans. Their outputs fuel the strategic loop, whose insights adapt the live experience.}
    \label{fig:two_loops_diagram}
\end{figure}

\subsection{Core Interaction: The Unified JSON Response Schema}

The architectural lynchpin of the real-time interaction loop is the system's reliance on a structured, unified JSON response schema. When a user sends a message, the Aika Meta-Agent does not engage in a multi-step reasoning process with other agents. Instead, it makes a single, optimized call to its underlying language model (Gemini 2.5 Flash), guided by a system prompt that instructs it to return a comprehensive JSON object. This design pattern ensures that conversational fluency, safety screening, and routing logic are handled in a single, atomic transaction.

The returned JSON object's schema is detailed in Table~\ref{tab:aika_json_schema}. Each field serves a distinct purpose in the agent's decision-making process, from generating an empathetic reply to providing a transparent audit trail for the assigned risk level.

\begin{table}[htbp]
    \centering
    \caption{The unified JSON response schema returned by the Aika Meta-Agent.}
    \label{tab:aika_json_schema}
    \small
    \setlength{\tabcolsep}{4pt}
    \begin{tabular}{p{3.5cm} p{1.5cm} p{8cm}}
        \toprule
        \textbf{Field} & \textbf{Type} & \textbf{Description} \\
        \midrule
        \code{suggested\_response} & string & The conversational response when no specialist agents are needed. If agents are invoked, this field is typically omitted or null. \\
        \code{immediate\_risk} & string & A five-level risk classification (\code{none}, \code{low}, \code{moderate}, \code{high}, \code{critical}) for the single message, enabling instantaneous safety screening. \\
        \code{crisis\_keywords} & array & A list of detected keywords from a predefined crisis lexicon (e.g., "bunuh diri," "menyakiti diri sendiri"). \\
        \code{risk\_reasoning} & string & A model-generated explanation for the assigned risk level, providing transparency for human oversight. \\
        \code{intent} & string & The classified user intent (e.g., \code{emotional\_support}, \code{crisis\_intervention}, \code{analytics\_query}), which dictates subsequent routing logic. \\
        \code{intent\_confidence} & float & A confidence score (0.0-1.0) indicating the model's certainty in the intent classification. \\
        \code{needs\_agents} & boolean & A flag indicating whether the query requires routing to a background specialist agent. \\
        \code{next\_step} & string & The specific downstream agent to route to (\code{tca}, \code{cma}, \code{ia}, \code{sta}, or \code{none}). \\
        \code{reasoning} & string & A brief explanation of why specialist agents are or are not needed, justifying the routing decision. \\
        \code{analytics\_params} & object & Optional parameters (e.g., \code{question\_id}, date range) captured when the intent is \code{analytics\_query}, enabling the Insights Agent to execute specific reports. \\
        \bottomrule
    \end{tabular}
\end{table}

This unified response schema yields several architectural benefits. First, it facilitates \textbf{latency optimization}; by consolidating response generation and risk assessment into one call, the system can achieve faster response times, which is critical for maintaining conversational fluidity. Second, it enables \textbf{embedded safety}, as risk assessment is an integral and non-negotiable part of every interaction loop. Third, the schema ensures \textbf{transparent oversight} by providing a clear audit trail for the system's reasoning. Finally, the \code{needs\_agents} flag allows for \textbf{conditional agent invocation}, an efficient resource management strategy that reduces backend compute costs by bypassing complex orchestration for simple queries.

An example of this schema in practice is shown in Figure~\ref{fig:json_moderate_stress}, where a user expresses moderate, non-imminent distress. In contrast, Figure~\ref{fig:json_greeting} shows the optimized response for a simple greeting. This structure operationalizes the principle that Aika is the sole user-facing component, synthesizing conversational intelligence and safety screening into a single, coherent interface layer.

\begin{figure}[htbp]
    \centering
\begin{lstlisting}[style=academicStyle, language=json]
{
  "suggested_response": null,
  "immediate_risk": "low",
  "crisis_keywords": [],
  "risk_reasoning": "User expresses anxiety about exams but no self-harm or severe distress indicators.",
  "intent": "emotional_support",
  "intent_confidence": 0.95,
  "needs_agents": true,
  "next_step": "tca",
  "reasoning": "Requires TCA for CBT coping strategies and intervention plan"
}
\end{lstlisting}
    \caption{Example Aika JSON response for moderate stress scenario. The response includes a supportive reply, a low risk assessment, and a routing decision to the Therapeutic Coach Agent (TCA).}
    \label{fig:json_moderate_stress}
\end{figure}

In contrast, a simple greeting would return:

\begin{figure}[htbp]
    \centering
\begin{lstlisting}[style=academicStyle, language=json]
{
  "suggested_response": "Halo! Saya Aika, asisten kesehatan mentalmu. Ada yang bisa saya bantu hari ini?",
  "immediate_risk": "none",
  "crisis_keywords": [],
  "risk_reasoning": "Standard greeting, no risk detected.",
  "intent": "casual_chat",
  "intent_confidence": 0.99,
  "needs_agents": false,
  "next_step": "none",
  "reasoning": "Simple greeting handled by meta-agent directly."
}
\end{lstlisting}
    \caption{Example Aika JSON response for casual greeting. The system detects no risk and handles the interaction directly without invoking specialist agents, optimizing latency.}
    \label{fig:json_greeting}
\end{figure}

\subsection{The Strategic Oversight Loop: Data-Driven Institutional Insight}

The Strategic Oversight Loop is designed to empower administrators with actionable insights derived from aggregated student interaction data. This loop addresses the systemic issues of delayed awareness and reactionary planning that currently plague mental health support services in higher education.

Key features of this loop include:

\begin{itemize}
    \item \textbf{Proactive Analytics:} The Insights Agent (IA) autonomously analyzes trends and generates reports on student well-being, identifying potential issues before they escalate into crises.
    \item \textbf{On-Demand Reporting:} Administrators can request custom reports or updates on specific metrics (e.g., "Show me the trend of moderate to high-risk cases over the past month"), which the IA fulfills by querying the latest data and synthesizing it into a clear, actionable format.
    \item \textbf{Scheduled Briefings:} The system can be configured to send regular, automated briefings to administrators, summarizing key metrics and highlighting any areas of concern that require attention.
\end{itemize}

This loop ensures that institutional leaders are not only reactive but also proactive, using real data to drive decisions and allocate resources where they are most needed.

%%%%%%%%%%%%%%%%%%%%%%%%%%%%%%%%%%%%%%%%%%%%%%%%%%%%%%%
%%% SECTION 3.3 - FUNCTIONAL ARCHITECTURE %%%
%%%%%%%%%%%%%%%%%%%%%%%%%%%%%%%%%%%%%%%%%%%%%%%%%%%%%%%

\section{Functional Architecture: The Agentic Core}
\label{chap:functional_architecture}

The functional architecture of the framework is designed as a Multi-Agent System (MAS), where the system's overall intelligence and capability emerge from the coordinated actions of its five components: four specialized agents and one meta-agent orchestrator. This section details the "what" of the system by defining the specific role, operational logic, and capabilities of each component within the \textbf{Safety Agent Suite}. Each specialist agent functions as a distinct component within the LangGraph state machine, perceiving its environment through the shared state, executing its logic, and updating the state with its results, while the Aika Meta-Agent coordinates their invocation and synthesizes their outputs.

\subsection{The Safety Triage Agent (STA): The Background Guardian}

The Safety Triage Agent (STA) serves as the system's comprehensive safety monitor. Unlike Aika's immediate Tier 1 screening, the STA performs deep-dive, asynchronous analysis (Tier 2). Once a conversation ends (either explicitly or because the user goes inactive), the Aika Meta-Agent enqueues the full transcript for STA review. The STA then reconstructs the exchange, applies richer temporal reasoning, and produces a signed conversation-level assessment that can confirm, refine, or escalate the risk rating recorded during the live chat. Its output feeds the compliance ledger and ensures that no concerning pattern slips through simply because the real-time classifier was overconfident or interrupted.

The agentic behavior of the STA can be understood through the BDI model:
\begin{itemize}
    \item \textbf{Beliefs:} The STA's beliefs are formed from the full conversation history and the structured output of its analysis model.
    \item \textbf{Desires:} Its fundamental desire is to ensure user safety by correctly identifying latent risks that may have been missed during real-time exchange.
    \item \textbf{Intentions:} If the STA identifies a \texttt{critical} risk during its analysis, its intention is to immediately trigger the \textbf{Case Management Agent (CMA)} for escalation. Otherwise, it updates the user's risk profile in the database.
\end{itemize}

\subsection{The Therapeutic Coach Agent (TCA): The Empathetic Guide}

The Therapeutic Coach Agent (TCA) acts as the background support engine for students in non-crisis situations. It is triggered by Aika when the initial risk assessment indicates \texttt{moderate} or \texttt{low} distress. The TCA does not converse directly with the user; instead, it generates structured therapeutic content (e.g., CBT exercises, coping strategies) that is delivered to the user's dashboard or via Aika.

Its agentic model is as follows:
\begin{itemize}
    \item \textbf{Beliefs:} The TCA's beliefs include the user's current message and Aika's risk assessment.
    \item \textbf{Desires:} Its core desire is to reduce user distress by providing actionable, evidence-based guidance.
    \item \textbf{Intentions:} Upon invocation, the TCA forms the intention to execute its \texttt{generate\_intervention\_plan} tool to create a personalized support plan.
\end{itemize}

\subsection{The Case Management Agent (CMA): The Procedural Coordinator}

The Case Management Agent (CMA) serves as the system's administrative backbone. It is activated under two distinct conditions: (1) by the \textbf{STA} (or Aika) following a critical risk detection, or (2) directly by \textbf{Aika} when a user explicitly requests an administrative action (e.g., "I want to book an appointment").

Its BDI breakdown is highly procedural:
\begin{itemize}
    \item \textbf{Beliefs:} The CMA believes the state of the world requires administrative action, triggered by a risk flag or a user intent.
    \item \textbf{Desires:} Its primary desire is to execute administrative workflows reliably and accurately.
    \item \textbf{Intentions:} When triggered by a crisis, it intends to execute \texttt{create\_crisis\_case}. When triggered by a user request, it intends to execute \texttt{schedule\_appointment}.
\end{itemize}

\subsection{The Insights Agent (IA): The Strategic Analyst}

The Insights Agent (IA) functions as the institution's automated well-being analyst, tasked with identifying population-level mental health trends from aggregated data. It is invoked exclusively by administrators to generate strategic reports.

Its agentic model is focused on data analysis and synthesis:
\begin{itemize}
    \item \textbf{Beliefs:} The IA's beliefs are derived from the administrator's query (e.g., "Show me crisis trends for October") and the aggregated, anonymized data it can access from the database.
    \item \textbf{Desires:} Its desire is to provide accurate, privacy-preserving, and actionable insights that help university leadership make data-driven decisions.
    \item \textbf{Intentions:} Based on the administrator's request, the IA forms an intention to run a specific, pre-defined SQL query against the database. It then forms a subsequent intention: to synthesize the numerical results from that query into a coherent, narrative summary for the administrator.
\end{itemize}

\subsection{The Aika Meta-Agent: Unified Orchestration Layer}
\label{sec:aika_meta_agent}

While the four specialized agents (STA, TCA, CMA, IA) provide the system's core intelligence, their coordination requires an orchestration layer. This layer must solve a fundamental challenge in multi-agent systems: how to present a unified, coherent interface to different user roles while dynamically routing requests based on intent, access rights, and context \cite{wooldridge2009introductionmas}. The Aika Meta-Agent is designed as this unified orchestration layer, acting as the single point of contact for all users and the master controller for the specialist agents operating in the background. Its primary responsibilities are to interpret user intent, manage conversational state, enforce role-based access control, and synthesize the outputs of the specialist agents into a coherent response.

The agentic behavior of the Aika Meta-Agent is defined as:
\begin{itemize}
    \item \textbf{Beliefs:} Aika believes the current state of the conversation, the user's authenticated role (Student/Admin), and the capabilities of the available specialist agents.
    \item \textbf{Desires:} Its primary desire is to maintain a seamless, empathetic user experience while strictly enforcing safety protocols and routing rules.
    \item \textbf{Intentions:} Upon receiving a message, Aika forms the intention to classify the user's intent (e.g., "greeting" vs. "crisis"), select the appropriate downstream agent (or handle it locally), and synthesize the final response.
\end{itemize}

\subsubsection{Dual-Mode Operation: Router vs. ReAct Agent}
Aika operates in two distinct cognitive modes to balance latency with capability:

\begin{enumerate}
    \item \textbf{Fast-Path Routing (Router Mode):} Upon receiving a message, Aika first acts as a semantic router (functioning as the architectural \textit{Supervisor}, see Section \ref{sec:supervisor_arch}). It utilizes a single-shot inference step to classify the user's intent into a structured JSON schema. This avoids the latency of a full reasoning loop for simple routing decisions.
    
    \item \textbf{Iterative Execution (ReAct Mode):} If the routing decision determines that Aika should handle the request directly (e.g., for appointment scheduling or general inquiries), the system transitions to a Reasoning and Acting (ReAct) loop. Defined formally as a trajectory $\tau = (o_1, a_1, o_2, a_2, \dots)$, Aika iteratively:
    \begin{itemize}
        \item \textbf{Reasons} about the current state and missing information.
        \item \textbf{Acts} by invoking specific tools (e.g., \texttt{check\_schedule}, \texttt{book\_slot}).
        \item \textbf{Observes} the tool output and refines its next action.
    \end{itemize}
\end{enumerate}

This hybrid approach ensures that the system remains responsive for high-level orchestration while retaining the depth required for complex task execution. This dual-mode logic is visualized in Figure \ref{fig:dual_mode_logic}.

\begin{figure}[htbp]
    \centering
    \resizebox{0.95\textwidth}{!}{%
    \begin{tikzpicture}[
        node distance=1.8cm and 2.5cm,
        auto,
        startstop/.style={circle, draw=black, thick, fill=black!5, minimum size=0.8cm, font=\footnotesize},
        process/.style={rectangle, draw=black, thick, fill=white, text width=2.5cm, align=center, rounded corners=2pt, minimum height=1cm, font=\small},
        decision/.style={diamond, draw=black, thick, fill=gray!10, aspect=2, text width=1.8cm, align=center, inner sep=0pt, font=\footnotesize},
        line/.style={draw, -latex, thick},
        label_node/.style={font=\scriptsize, fill=white, inner sep=1pt, align=center}
    ]
        % Nodes
        \node [startstop] (start) {Start};
        \node [process, below=of start] (router) {Aika Router\\(Single-Shot)};
        \node [decision, below=of router] (decide) {Needs\\Tools?};
        
        % Fast Path
        \node [process, right=of decide, xshift=1cm] (direct) {Direct\\Response};
        
        % ReAct Path
        \node [process, below=of decide, yshift=-0.5cm] (thought) {Thought};
        \node [process, below=of thought] (action) {Action\\(Tool Call)};
        \node [process, left=of action, xshift=-1cm] (observe) {Observation};
        
        \node [startstop, right=of direct] (end) {End};

        % Edges
        \path [line] (start) -- (router);
        \path [line] (router) -- (decide);
        
        % No Tools -> Direct
        \path [line] (decide) -- node [label_node] {No} (direct);
        \path [line] (direct) -- (end);
        
        % Yes Tools -> ReAct
        \path [line] (decide) -- node [label_node] {Yes} (thought);
        \path [line] (thought) -- (action);
        \path [line] (action) -- (observe);
        \path [line] (observe) |- (thought);
        
        % Exit ReAct
        \path [line] (thought.east) -| node [label_node, near start] {Done} (direct.south);

    \end{tikzpicture}%
    }
    \caption{Dual-Mode Operation Logic. The system first attempts a fast-path routing decision. Only if complex tool use is required does it enter the iterative ReAct loop.}
    \label{fig:dual_mode_logic}
\end{figure}

Collectively, these specialized agents operationalize the two proactive loops described in Section 3.2. The STA and TCA are the primary actors in the \textbf{Real-Time Interaction Loop}, enabling proactive individual support through immediate risk detection and the asynchronous delivery of therapeutic content. The IA is the engine of the \textbf{Strategic Oversight Loop}, providing the institution with proactive, population-level insights. The CMA acts as a crucial bridge between these loops, translating automated insights (from STA or IA) into concrete administrative actions, such as case creation or counselor notification. This functional separation ensures that each component is optimized for its specific role within the broader proactive ecosystem.

%%%%%%%%%%%%%%%%%%%%%%%%%%%%%%%%%%%%%%%%%%%%%%%%%%%%%%%
%%% SECTION 3.4 - TECHNICAL ARCHITECTURE %%%
%%%%%%%%%%%%%%%%%%%%%%%%%%%%%%%%%%%%%%%%%%%%%%%%%%%%%%%

\section{Technical Architecture}
\label{sec:technical_architecture}

This section details the technical blueprint of the Safety Agent Suite, translating the conceptual and functional designs into a concrete implementation strategy. The architecture is built upon a modern, cloud-native technology stack, selected to ensure modularity, scalability, and maintainability, which are critical for a system of this nature.

\subsection{Technology Stack}

The selection of technologies was guided by the need for asynchronous performance, robust data management, and stateful agent orchestration. The core components are:

\begin{itemize}
    \item \textbf{Backend Framework: FastAPI.} The backend is implemented in Python using FastAPI. This choice was motivated by FastAPI's high performance and its native support for asynchronous operations. For a conversational AI system where multiple I/O-bound tasks occur (e.g., database queries, external API calls to LLMs), asynchronous handling is paramount to prevent blocking and ensure a responsive user experience.

    \item \textbf{Agent Orchestration: LangGraph.} The complex, conditional logic of the multi-agent system is managed using LangGraph \cite{langgraph2024}. LangGraph provides a stateful, graph-based framework for composing agents. This is a significant improvement over stateless LLM calls, as it allows the system to maintain a coherent state across multiple turns of a conversation and multiple agent invocations. It directly enables the implementation of the agentic loops and decision points described in the functional architecture.

    \item \textbf{Data Persistence: PostgreSQL and SQLAlchemy.} A PostgreSQL database serves as the primary data store for all persistent information, including user profiles, conversation histories, and agent execution logs. Interaction with the database is managed through the SQLAlchemy Object-Relational Mapper (ORM). This combination provides a robust, transactional, and scalable foundation for data management, while the ORM simplifies data handling in the Python application code.

    \item \textbf{Containerization: Docker.} The entire application stack, including the FastAPI backend, database, and other services, is containerized using Docker. This ensures a consistent, reproducible, and isolated environment for development, testing, and potential deployment, simplifying dependency management and enhancing system reliability.
\end{itemize}

\subsection{Data Model and Persistence}

The system's data model is designed to support its core functions: tracking conversations, managing user data, and logging agent behavior for analysis and auditing. While a full database schema is extensive, the core entities include:

\begin{itemize}
    \item \textbf{User and Profile Tables:} Store essential user information, preferences, and consent status, forming the basis for personalized interaction.
    \item \textbf{Conversation and Message Tables:} Log every user interaction, providing the raw data for the Insights Agent and a history for contextual conversations.
    \item \textbf{Case Management Tables:} Store structured data for escalated cases, including risk level, summary, and assigned counselor, enabling the HITL workflow.
    \item \textbf{LangGraph Execution Logs:} A critical component for fulfilling RQ2, these tables (\texttt{LangGraphExecution} and \texttt{LangGraphNodeExecution}) capture detailed traces of every agent orchestration. They log which nodes (agents) were executed, the transitions between them, their inputs and outputs, and any errors encountered. This provides an invaluable audit trail for debugging and evaluating the orchestration logic.
\end{itemize}

\subsection{Stateful Orchestration with LangGraph}

The heart of the technical architecture is the LangGraph state machine, which operationalizes the agentic behavior. The orchestration follows a "supervisor" pattern where the Aika Meta-Agent serves as the central decision node, routing control to specialist agents only when specific conditions are met.

The process is as follows:
\begin{enumerate}
    \item A user message initializes the \texttt{AgentState}.
    \item The graph routes the state to the first node, the \textbf{Aika Meta-Agent}.
    \item Aika analyzes the input using its system prompt and updates the \texttt{AgentState} with a Tier 1 risk assessment and a routing decision (e.g., \texttt{needs\_agents: true}, \texttt{next\_step: "tca"}).
    \item A conditional edge reads the \texttt{next\_step} from the state and routes it to the appropriate next node:
    \begin{itemize}
        \item \textbf{Therapeutic Coach Agent (TCA):} For generating coping strategies (Moderate/Low risk).
        \item \textbf{Case Management Agent (CMA):} For immediate crisis escalation or administrative requests.
        \item \textbf{Safety Triage Agent (STA):} For manual risk analysis invoked by administrators (Tier 2 analysis for students runs as a background task).
        \item \textbf{Insights Agent (IA):} For population-level analytics and reporting (Admin only).
        \item \textbf{End:} For direct replies where no specialist agent is required.
    \end{itemize}
    \item Specialist agents, if invoked, execute their logic, update the shared state, and the flow converges to the end of the graph.
\end{enumerate}

This stateful, graph-based approach provides a robust and explicit way to manage the complex, non-deterministic nature of a multi-agent conversational system. A high-level visualization of this state machine is presented in Figure \ref{fig:langgraph_state_machine}.

\begin{figure}[htbp]
    \centering
    \resizebox{0.95\textwidth}{!}{%
    \begin{tikzpicture}[
        node distance=1.8cm and 2.0cm,
        auto,
        % Styles
        startstop/.style={circle, draw=black, thick, fill=black!5, minimum size=0.8cm, font=\footnotesize},
        agent/.style={rectangle, draw=black, thick, fill=white, text width=2.2cm, align=center, rounded corners=2pt, minimum height=1.2cm, font=\small\bfseries},
        decision/.style={diamond, draw=black, thick, fill=gray!10, aspect=2, text width=1.8cm, align=center, inner sep=0pt, font=\footnotesize},
        line/.style={draw, -latex, thick},
        label_node/.style={font=\scriptsize, fill=white, inner sep=1pt, align=center}
    ]

        % Nodes
        \node [startstop] (start) {Start};
        \node [agent, below=of start, text width=3cm] (aika) {Aika Meta-Agent\\(Supervisor)};
        
        % Specialist Agents arranged below Aika
        % Spreading 4 agents horizontally
        \node [agent, below=of aika, xshift=-4.5cm, yshift=-1cm] (sta) {Safety Triage\\(STA)};
        \node [agent, below=of aika, xshift=-1.5cm, yshift=-1cm] (tca) {Therapeutic Coach\\(TCA)};
        \node [agent, below=of aika, xshift=1.5cm, yshift=-1cm] (cma) {Case Mgmt\\(CMA)};
        \node [agent, below=of aika, xshift=4.5cm, yshift=-1cm] (ia) {Insights Agent\\(IA)};
        
        \node [startstop, below=of aika, yshift=-4cm] (end) {End};

        % Edges from Start
        \path [line] (start) -- (aika);

        % Routing from Aika
        \path [line] (aika) -| node [label_node, near start] {Safety Check} (sta);
        \path [line] (aika) -| node [label_node, near start, pos=0.3] {Coaching} (tca);
        \path [line] (aika) -| node [label_node, near start, pos=0.3] {Crisis/Admin} (cma);
        \path [line] (aika) -| node [label_node, near start] {Analytics} (ia);
        
        % Direct Reply path (curved to avoid crossing)
        % Drawing a path that goes around the agents
        \draw [line] (aika.east) -- ++(5.5,0) |- node [label_node, near start] {Direct / ReAct} (end);

        % Convergence to End
        \path [line] (sta) |- (end);
        \path [line] (tca) |- (end);
        \path [line] (cma) |- (end);
        \path [line] (ia) |- (end);

    \end{tikzpicture}%
    }
    \caption{LangGraph State Machine Visualization. Aika acts as the central supervisor, routing the conversation to specialist agents (STA, TCA, CMA, IA) or responding directly based on the context.}
    \label{fig:langgraph_state_machine}
\end{figure}

\subsubsection{Hierarchical Supervisor Architecture}
\label{sec:supervisor_arch}
The system implements a \textit{Supervisor} architectural pattern, modeled as a Hierarchical State Machine (HSM). In this topology, the Aika Meta-Agent functions as the root supervisor node, maintaining the global state of the conversation.

Unlike a flat multi-agent system where agents communicate directly with one another (mesh topology), the Supervisor architecture enforces a star topology:
\begin{itemize}
    \item \textbf{Centralized Control:} All state transitions must pass through the Aika node, ensuring a single source of truth for the conversation context.
    \item \textbf{Isolated Subgraphs:} Specialized agents (STA, TCA, CMA) are implemented as independent subgraphs. They process their specific tasks and return the updated state to the supervisor, rather than handing off control to other agents directly.
    \item \textbf{Conditional Routing:} The edges between the supervisor and the subgraphs are conditional, determined by the \texttt{needs\_agents} and \texttt{risk\_level} variables derived during the routing phase.
\end{itemize}

This structure minimizes the "infinite loop" hallucinations common in cyclic multi-agent graphs and provides a structured execution path for safety-critical mental health interventions. Note that while the graph topology is deterministic, the routing decisions themselves are made by the LLM-based supervisor, introducing inherent variability characteristic of language model inference. The hierarchical relationship is illustrated in Figure \ref{fig:supervisor_hierarchy}.

\begin{figure}[htbp]
    \centering
    \resizebox{0.8\textwidth}{!}{%
    \begin{tikzpicture}[
        node distance=3cm,
        supervisor/.style={circle, draw=black, very thick, fill=white, minimum size=2.5cm, align=center, font=\bfseries},
        worker/.style={rectangle, draw=black, thick, fill=gray!10, minimum width=2.5cm, minimum height=1.2cm, align=center, font=\small},
        arrow/.style={-latex, thick},
        label_text/.style={font=\scriptsize, align=center, midway, fill=white}
    ]
        % Nodes
        \node [supervisor] (aika) {Aika\\(Supervisor)};
        
        % Workers arranged in a row below
        \node [worker, below=of aika, xshift=-5cm] (sta) {STA\\(Worker)};
        \node [worker, below=of aika, xshift=-1.7cm] (tca) {TCA\\(Worker)};
        \node [worker, below=of aika, xshift=1.7cm] (cma) {CMA\\(Worker)};
        \node [worker, below=of aika, xshift=5cm] (ia) {IA\\(Worker)};
        
        % Edges
        % Aika -> STA
        \draw [arrow] (aika) -- node [label_text, sloped, above] {Delegate} (sta);
        \draw [arrow, dashed] (sta) to[bend left=15] node [label_text, sloped, below] {Return} (aika);
        
        % Aika -> TCA
        \draw [arrow] (aika) -- node [label_text, sloped, above] {Delegate} (tca);
        \draw [arrow, dashed] (tca) to[bend left=15] node [label_text, sloped, below] {Return} (aika);
        
        % Aika -> CMA
        \draw [arrow] (aika) -- node [label_text, sloped, above] {Delegate} (cma);
        \draw [arrow, dashed] (cma) to[bend right=15] node [label_text, sloped, below] {Return} (aika);

        % Aika -> IA
        \draw [arrow] (aika) -- node [label_text, sloped, above] {Delegate} (ia);
        \draw [arrow, dashed] (ia) to[bend right=15] node [label_text, sloped, below] {Return} (aika);

    \end{tikzpicture}%
    }
    \caption{Hierarchical Supervisor Architecture. Aika acts as the central supervisor, delegating tasks to worker agents (subgraphs) and receiving their state updates. Workers do not communicate directly with each other.}
    \label{fig:supervisor_hierarchy}
\end{figure}

%%%%%%%%%%%%%%%%%%%%%%%%%%%%%%%%%%%%%%%%%%%%%%%%%%%%%%%
%%% SECTION 3.5 - CROSS-CUTTING CONCERNS %%%
%%%%%%%%%%%%%%%%%%%%%%%%%%%%%%%%%%%%%%%%%%%%%%%%%%%%%%%

\section{Cross-Cutting Concerns}
\label{sec:cross_cutting_concerns}

Beyond the core functional and technical architecture, a production-worthy system, particularly in a sensitive domain like mental health, must address several system-wide, non-functional requirements. These "cross-cutting concerns" ensure the system is secure, responsive, and safe.

\subsection{Security and Privacy by Design}

Security and privacy are not afterthoughts but are foundational to the system's design, earning user trust and ensuring ethical operation.

\begin{itemize}
    \item \textbf{Role-Based Access Control (RBAC):} The system enforces strict access control based on user roles (e.g., student, counselor, administrator). For instance, counselors can only view cases assigned to them, and administrators can access aggregated analytics from the Insights Agent but not individual, non-anonymized conversation logs. This is managed through authentication middleware in the FastAPI backend.
    
    \item \textbf{Data Encryption:} All data is encrypted both in-transit, using TLS for all API communications, and at-rest in the PostgreSQL database. This protects sensitive conversation data from unauthorized access even in the event of a direct infrastructure breach.

    \item \textbf{Privacy-Preserving Analytics:} The Insights Agent is architecturally constrained to protect student privacy. As stated in RQ3, its SQL queries are designed to enforce k-anonymity \cite{sweeney2002kanonymity} by including clauses that prevent data from being returned for any group smaller than a predefined threshold (k=5). This ensures that analytics can reveal population-level trends without ever exposing data that could be traced back to an individual student.
\end{itemize}

\subsection{Architectural Provisions for Responsiveness}

While formal performance benchmarking is outside the scope of this thesis, the architecture was explicitly designed to support a responsive, real-time conversational experience. This is a critical functional requirement for user engagement.

\begin{itemize}
    \item \textbf{Asynchronous Processing:} The choice of FastAPI was deliberate for its native \texttt{async/await} support. This allows the application to handle long-running I/O operations, such as calling the Gemini API or querying the database, without blocking the main execution thread. This ensures the system can manage multiple concurrent conversations smoothly.

    \item \textbf{Optimized Language Models:} The system employs a two-tier model strategy to balance capability with latency. For the initial, real-time safety screening performed by the \textbf{Aika Meta-Agent}, a low-latency model (Gemini 2.5 Flash) is used to ensure rapid response times. For more complex, asynchronous tasks like generating detailed narrative summaries in the Insights Agent, a more powerful model (Gemini 2.5 Pro) is used, as latency is less critical for these background tasks.
\end{itemize}

\subsection{Human-in-the-Loop (HITL) Workflow for Safety}

No fully automated system can or should replace human clinical judgment in crisis situations. The framework is designed with a robust Human-in-the-Loop (HITL) workflow as its ultimate safety net.

Once a crisis is detected, the subsequent escalation path is deterministic and auditable:
\begin{enumerate}
    \item The \textbf{Aika Meta-Agent} (immediate) or \textbf{STA} (background) detects a message with "Critical" risk.
    \item This immediately triggers the \textbf{Case Management Agent (CMA)}.
    \item The CMA executes its \texttt{create\_crisis\_case} tool, which creates a structured, high-priority ticket in the database.
    \item Simultaneously, the CMA invokes a notification service (e.g., via email or a secure messaging integration) that sends an alert to the on-call human counselor(s). This alert contains the case ID and a link to a secure dashboard where they can review the conversation.
    \item The system then presents the user with immediate, static help resources (e.g., emergency hotline numbers) while the human counselor takes over the case management.
\end{enumerate}

This HITL design ensures that the AI's role is to act as a high-speed, scalable detection and triage system, but the ultimate responsibility for crisis intervention remains with trained human professionals.

%%%%%%%%%%%%%%%%%%%%%%%%%%%%%%%%%%%%%%%%%%%%%%%%%%%%%%%%
%%% SECTION 3.4 - ETHICAL CONSIDERATIONS & LIMITATIONS %%%
%%%%%%%%%%%%%%%%%%%%%%%%%%%%%%%%%%%%%%%%%%%%%%%%%%%%%%%%

\section{Ethical Considerations and Research Limitations}
\label{sec:ethical_considerations}

The development of an AI-driven framework for mental health support necessitates thorough examination of ethical implications and transparent acknowledgment of research limitations. This section addresses the ethical design choices and defines the boundaries of the study's findings.

\subsection{Informed Consent and Transparency}

The UGM-AICare framework is designed with the principle that users must have clear understanding of the system's capabilities and limitations. The Aika Meta-Agent explicitly discloses its non-human nature in initial interactions, ensuring users engage with informed consent about the conversational context. This transparency is critical in healthcare applications where users may form therapeutic relationships with AI systems.

\subsection{Human-in-the-Loop for Safety and Ethical Safeguards}

The framework is explicitly designed as a tool that assists, but does not replace, human counselors. Every critical risk escalation from the \textbf{Aika Meta-Agent} or \textbf{Safety Triage Agent (STA)} creates a case that requires mandatory review and action by a qualified human professional. The system automates the detection and reporting, but the final clinical judgment and intervention remain firmly in human hands.

This human oversight is not merely procedural; it addresses the fundamental ethical limitation of LLMs in safety-critical contexts. While models like Gemini 2.5 Flash demonstrate strong performance in text understanding, they can still misinterpret nuanced emotional states or linguistic cues. The human-in-the-loop design ensures that no automated risk classification leads directly to intervention without expert clinical validation.

Given the high-stakes nature of mental health triage, the system is designed with explicit ethical safeguards:

\begin{itemize}
    \item \textbf{Conservative Risk Classification:} The agents employ a "safety-first" bias, erring on the side of escalation when ambiguous risk indicators are detected. This prevents false negatives in critical situations.
    \item \textbf{Human-in-the-Loop for Critical Cases:} All cases flagged as "critical" trigger immediate notifications to human counselors. The agents do not make autonomous decisions about crisis intervention; they serve as detection and escalation mechanisms only.
    \item \textbf{Transparency in Agent Responses:} The Aika Meta-Agent explicitly discloses its non-human nature and limitations in its initial greeting, ensuring users have informed consent about the conversational context.
\end{itemize}

Technology alone is insufficient to guarantee ethical operation. Therefore, the system is designed with procedural safeguards that ensure human oversight for all critical functions, ensuring the framework operates as a support tool rather than as an autonomous clinical actor.

\subsection{AI as Support Tool, Not Replacement for Therapy}

It is ethically imperative to clearly define the system's role. The UGM-AICare framework is designed as a sub-clinical, supportive tool and a bridge to professional care, not as a substitute for licensed therapy. The Therapeutic Coach Agent is programmed to explicitly state this boundary and to encourage users to seek professional help for serious or persistent issues, facilitated through the Case Management Agent's appointment booking functionality and clinical escalation workflows.

\subsection{Research Limitations and Scope Boundaries}

This study, as a work of Design Science Research focused on artifact creation and evaluation, is subject to several important limitations:

\begin{itemize}
    \item \textbf{Methodological Limitation - Scenario-Based Evaluation:} The evaluation of this framework (detailed in Chapter~\ref{chap:evaluation}) is based on controlled scenario testing with synthetic conversational data, not real-world user deployment. This thesis validates the \textit{technical feasibility} of the agentic workflows and the \textit{architectural integrity} of the multi-agent design. It does \textbf{not} measure long-term psychological outcomes or therapeutic efficacy on actual students. Such claims would require extensive ethics approval, medical supervision, and longitudinal clinical trials that exceed the scope of bachelor's-level research.
    
    \item \textbf{Technical Limitation - Inherent Risks of LLMs:} The framework relies on Google Gemini 2.5 Flash and Gemini 2.5 Flash Lite APIs for different agent tasks (routing, classification, plan generation). Like all LLMs, these models are subject to inherent limitations including potential biases from training data and the possibility of generating factually incorrect or nonsensical responses ("hallucinations"). While the system's use of structured tools, typed state schemas, and explicit agent prompts is designed to mitigate these risks, they cannot be eliminated entirely.
    
    \item \textbf{Data Limitation - Simulated Evaluation Data:} The evaluation is conducted using synthetically generated mental health scenarios and simulated conversational patterns, not real user data. This is necessary to protect privacy during the development phase and to enable controlled testing without requiring human subjects approval. However, it means that agent performance has not been validated on the specific linguistic diversity, cultural contexts, and edge cases of a live Indonesian student population.
    
    \item \textbf{Scope Limitation - Agent Architecture Focus:} This thesis evaluates the multi-agent architecture: the BDI-based specialist agents, Aika orchestration layer, and their collective behavior in safety-critical conversations. The full UGM-AICare implementation includes database design, user interface components, blockchain token systems, and deployment infrastructure, but \textbf{these system components are not subjects of formal evaluation in this work}. They serve as implementation context to demonstrate feasibility, but their performance characteristics, user experience quality, and production readiness are not validated. The thesis evaluates agent performance through controlled scenario-based testing rather than real-world user deployment.
\end{itemize}

These limitations do not diminish the validity of the research findings within their defined scope. They represent transparent acknowledgment of the boundaries between artifact evaluation (the focus of this thesis) and clinical deployment (which requires additional validation beyond this work's scope). The evaluation methodology in Chapter~\ref{chap:evaluation} is designed to rigorously assess the aspects that \textit{can} be measured through controlled testing: agent accuracy, orchestration correctness, response quality, and privacy preservation in aggregated analytics.