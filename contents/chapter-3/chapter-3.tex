\chapter{System Design and Architecture}
\label{chap:system_design}

%%%%%%%%%%%%%%%%%%%%%%%%%%%%%%%%%%%%%%%%%%%%%%%%%%%%%%
%%% SECTION 3.1 - DSR METHODOLOGY %%%
%%%%%%%%%%%%%%%%%%%%%%%%%%%%%%%%%%%%%%%%%%%%%%%%%%%%%%

\section{Research Methodology: Design Science Research (DSR)}
\label{sec:dsr_methodology}

The research presented in this thesis is constructive in nature, aimed not merely at describing or explaining a phenomenon, but at creating a novel and useful artifact to solve a real-world problem. To provide a rigorous and systematic structure for this endeavor, this study adopts the \textbf{Design Science Research (DSR)} methodology. DSR is a well-established paradigm in Information Systems research focused on the creation and evaluation of innovative IT artifacts intended to solve identified organizational problems \cite{dsr_methodology_hevner_2004}. The primary goal of DSR is to generate prescriptive design knowledge through the building and evaluation of these artifacts.

The DSR process model, as outlined by Peffers et al., provides an iterative framework that guides the research from problem identification to the communication of results \cite{dsr_methodology_peers_2006}. This thesis follows these stages, mapping them directly to its structure to ensure a logical and transparent research process:

\begin{enumerate}
    \item \textbf{Problem Identification and Motivation:} This initial stage, which involves defining the specific research problem and justifying the value of a solution, is addressed in \textbf{Chapter \ref{chap:introduction}} of this thesis. We have identified the inefficiencies of the reactive mental health support model as the core problem.

    \item \textbf{Define Objectives and Knowledge Base:} Building on the identified problem, this stage formalizes the solution objectives and anchors them in the relevant knowledge base. The initial objectives are articulated in \textbf{Chapter \ref{chap:introduction}}, and they are refined and theoretically grounded through the literature synthesis in \textbf{Chapter \ref{sec:literature_review}}.

    \item \textbf{Design and Development:} This is the core constructive phase where the artifact's architecture and functionalities are developed. This stage is the primary focus of the present chapter, \textbf{Chapter \ref{chap:system_design}}, which outlines the functional and technical blueprint of the agentic AI framework.

    \item \textbf{Demonstration:} In this stage, the designed artifact is demonstrated to solve representative instances of the problem. The functional prototype and its scenario walkthroughs are presented in \textbf{Chapter~IV}, particularly Sections \ref{sec:setup} and \ref{sec:rq1}.

    \item \textbf{Evaluation:} This stage observes and measures how well the artifact supports the solution objectives. The scenario-based tests and their analysis are reported in \textbf{Chapter~IV}, Sections \ref{sec:rq1}--\ref{sec:discussion}.

    \item \textbf{Communication of Results:} The final stage disseminates the artifact, findings, and implications to the target audience. This thesis (culminating in \textbf{Chapter~V} and supported by the appendices) serves as the primary communication vehicle.
\end{enumerate}

Stages 4-6 therefore operationalize the empirical programme for the research questions defined in Chapter \ref{sec:research_questions}. The demonstration assets in Chapter~IV (Sections~\ref{sec:setup} and \ref{sec:rq1}) instantiate the scenarios for RQ1--RQ4, while the evaluation stage reports the quantitative indicators detailed in Chapter~IV: STA sensitivity/specificity for safety (RQ1), orchestration reliability metrics such as tool-call success and latency for resilience (RQ2), rubric-based CBT quality scores for coaching (RQ3), and stability of aggregate analytics for the Insights Agent (RQ4). The communication stage synthesizes these findings in Chapter~V so that institutional stakeholders can interpret the metrics and translate them into policy and operational decisions. Together, the paragraph-level traceability between stages and metrics makes the DSR cycle a roadmap for the scenario-based evaluation that follows.

The complete workflow of this research, following the DSR methodology, is visualized in Figure \ref{fig:dsr_flowchart}. This diagram illustrates the iterative path from problem formulation through to the final conclusions and recommendations.

\definecolor{ugmBlue}{RGB}{0,73,144}
\definecolor{ugmGold}{RGB}{217,160,33}

\begin{figure}[h]
    \centering
    \begin{tikzpicture}[
        x=5.2cm, y=2.2cm,
        process/.style={rectangle, rounded corners=5pt, draw=ugmBlue, very thick, minimum width=4.0cm, minimum height=1.2cm, align=center, fill=ugmBlue!6, font=\footnotesize},
        arrow/.style={-Latex, thick, ugmBlue},
        looparrow/.style={-Latex, thick, ugmGold}
    ]
        \node[process] (problem) at (0,1) {Problem Identification\\(Chapter~\ref{chap:introduction})};
        \node[process] (literature) at (1,1) {Objectives \& Knowledge Base\\(Ch.~\ref{chap:introduction},~\ref{sec:literature_review})};
        \node[process] (design) at (2,1) {Artifact Design \& Development\\(Chapter~\ref{chap:system_design})};

        \node[process] (implementation) at (0,0) {Prototype Demonstration\\(Chapter~IV, Sec.~\ref{sec:setup})};
        \node[process] (evaluation) at (1,0) {Scenario Evaluation\\(Chapter~IV, Secs.~\ref{sec:rq1}--\ref{sec:discussion})};
        \node[process] (conclusion) at (2,0) {Communication of Results\\(Chapter~V)};

        \draw[arrow] (problem) -- (literature);
        \draw[arrow] (literature) -- (design);
        \draw[arrow] (implementation) -- (evaluation);
        \draw[arrow] (evaluation) -- (conclusion);
        \draw[arrow] (design) -- node[right, font=\footnotesize]{prototype assets} (implementation);
        \draw[arrow] (literature) -- node[right, font=\footnotesize]{metrics \& RQs} (implementation);
        \draw[looparrow, looseness=0.9] (evaluation.north) to[out=110, in=-110] (design.south);
        \draw[looparrow, looseness=0.9] (implementation.north east) to[out=45, in=225] (literature.south west);
        \draw[looparrow, looseness=1.0] (conclusion.east) to[out=0, in=0] (problem.east);
    \end{tikzpicture}
    \caption{The Design Science Research (DSR) process model as applied in this thesis. The two-row layout shows how objectives inform design and how demonstration/evaluation feed back into earlier stages.}
    \label{fig:dsr_flowchart}
\end{figure}

%%%%%%%%%%%%%%%%%%%%%%%%%%%%%%%%%%%%%%%%%%%%%%%%%%%%%%%
%%% SECTION 3.2 - SYSTEM OVERVIEW %%%
%%%%%%%%%%%%%%%%%%%%%%%%%%%%%%%%%%%%%%%%%%%%%%%%%%%%%%%

\section{System Overview and Conceptual Design}

The artifact proposed and developed in this research is a novel agentic AI framework designed to address the systemic inefficiencies of traditional, reactive mental health support models in Higher Education Institutions. The conceptual architecture is predicated on the principles of a Multi-Agent System (MAS), wherein a suite of collaborative, specialized intelligent agents, collectively termed the \textbf{Safety Agent Suite}, work in concert to create a proactive, scalable, and data-driven support ecosystem. This framework is designed not as a monolithic application, but as a dynamic, closed-loop system that operates on two interconnected levels: a micro-level loop for real-time, individual student support and a macro-level loop for strategic, institutional oversight and proactive intervention \cite{kashiv2025aidrivennetworks, nwoke2025insightautomation}.

The system's primary entities and their designated interaction points are illustrated in the conceptual context diagram in Figure \ref{fig:context_diagram}. These entities are:
\begin{itemize}
    \item \textbf{Students:} As the primary users, students interact with the system's conversational interface (UGM-AICare's `/aika` page). This serves as their direct entry point to the support ecosystem, where they engage with the agents responsible for coaching and immediate assistance.
    \item \textbf{University Staff/Counselors:} As the system's administrators and clinical supervisors, these stakeholders interact with a secure Admin Dashboard. This interface serves as the human-in-the-loop control center, providing aggregated analytics for strategic decision-making and a case management system for handling high-risk escalations.
    \item \textbf{The Agentic AI Backend:} This is the core computational engine of the framework. It hosts the five components of the Safety Agent Suite (four specialist agents plus the Aika Meta-Agent orchestrator), manages their stateful interactions via LangGraph, and serves as the central hub for all data processing, logic execution, and communication with external services and databases.
\end{itemize}

\begin{table}[h]
    \centering
    \caption{Comparison of representative university well-being platforms.}
    \label{tab:wellbeing_comparison}
    \begin{tabular}{p{3.4cm}p{3.6cm}p{3.6cm}p{3.6cm}}
        \toprule
        \textbf{Platform} & \textbf{Primary Modality} & \textbf{Safety/Privacy Posture} & \textbf{Contrast with Safety Agent Suite} \\
        \midrule
        Woebot Campus Programme\cite{FIND_CITATION_PLACEHOLDER} & Daily CBT-aligned chatbot sessions with journaling prompts. & Crisis disclaimers and human referral prompts, but escalation remains manual and analytics are limited. & Lacks dedicated triage agent or orchestration guardrails; actionable insights are not automated, restricting proactive interventions. \\
        Togetherall Peer Community\cite{FIND_CITATION_PLACEHOLDER} & Moderated peer-to-peer forums with clinician oversight and resource hubs. & Strong anonymity policies and moderator workflows, yet response time depends on human moderators. & Provides community support but no automated coaching, triage, or strategic analytics loop as offered by the Safety Agent Suite. \\
        Kognito ``At-Risk'' Simulations\cite{FIND_CITATION_PLACEHOLDER} & Scenario-based training to help faculty/students identify and refer distressed peers. & Focuses on awareness; no live data capture or privacy-sensitive storage since it is a training tool. & Educates stakeholders but does not deliver operational support, automated escalation, or continuous monitoring of student well-being. \\
        \bottomrule
    \end{tabular}
\end{table}

Conceptually, the framework's architecture is best understood as two distinct but integrated operational loops:

\begin{enumerate}
    \item \textbf{The Real-Time Interaction Loop:} This loop handles immediate, synchronous interactions with individual students. When a student sends a message, it is first processed by the \textbf{Safety Triage Agent (STA)} for risk assessment. If the context is deemed safe, the \textbf{Support Coach Agent (SCA)} takes over to provide personalized, evidence-based guidance. Should the user require administrative assistance, such as scheduling an appointment, the workflow is seamlessly handed off to the \textbf{Service Desk Agent (SDA)}. This loop is designed for high-availability, low-latency responses, ensuring that students receive immediate and appropriate support.
    \item \textbf{The Strategic Oversight Loop:} This loop operates on a longer, asynchronous timescale to enable proactive, institution-wide strategy. The \textbf{Insights Agent (IA)} periodically analyzes the anonymized, aggregated data from all student interactions. It generates reports on population-level well-being trends, sentiment analysis, and emerging topics of concern. These reports are delivered to administrators via the Admin Dashboard, providing the empirical evidence needed for data-driven resource allocation, such as commissioning new workshops or adjusting counseling staff schedules. This loop directly addresses the "insight-to-action" gap that plagues current systems \cite{nwoke2025insightautomation, jorno2018actionableinsight}.
\end{enumerate}

The synergy between these two loops is the cornerstone of the framework's design. The real-time loop gathers the data that fuels the strategic loop, while the insights from the strategic loop can be used to configure and improve the proactive interventions delivered by the real-time loop, creating a continuously learning and adaptive support ecosystem.

\subsubsection{Orchestration Alternatives and Rationale}

Classical multi-agent systems literature distinguishes between centralized planners, market-based negotiation schemes, and more recent graph-structured controllers for coordinating autonomous agents \cite{wooldridge1995intelligentagents,wooldridge2009introductionmas}. Contemporary surveys focused on LLM-powered agents show that orchestration frameworks such as LangGraph provide deterministic state persistence, guardrails, and cycle control that are difficult to obtain in purely contract-net or ad-hoc workflow engines \cite{yang2025aiagentprotocols,tran2025multiagentcollaboration,yu2025agentworkflow}. Table~\ref{tab:orchestration_patterns} contrasts these patterns and summarizes why the Safety Agent Suite adopts a graph-based orchestration.

\begin{table}[h]
    \centering
    \caption{Comparison of orchestration patterns for the Safety Agent Suite.}
    \label{tab:orchestration_patterns}
    \begin{tabular}{p{3.4cm}p{4.0cm}p{4.0cm}}
        \toprule
        \textbf{Approach} & \textbf{Coordination Strengths} & \textbf{Limitations and Implications for Safety Agent Suite} \\
        \midrule
        Centralized workflow/planner \cite{wooldridge2009introductionmas} & Deterministic control flow and straightforward verification of simple pipelines. & Brittle when the conversation requires branching or repeated loops; a single orchestrator becomes a failure hotspot and hinders human-in-the-loop escalations needed for STA. \\
        Contract-net / market-based negotiation \cite{wooldridge1995intelligentagents,tran2025multiagentcollaboration} & Decentralized task allocation and flexibility for loosely coupled agents. & Negotiation latency and probabilistic assignment make it difficult to guarantee triage deadlines and safety invariants; insufficient for crisis escalation SLAs. \\
        Graph-based, stateful orchestrator (LangGraph) \cite{yang2025aiagentprotocols,yu2025agentworkflow} & Explicit state persistence, guard conditions, and cyclic workflows that support guardrails, retries, and logging. & Requires deliberate state-schema design and emerging tooling, but best aligns with the need for deterministic escalation paths, auditing, and metric capture in Chapters~IV and~V. \\
        \bottomrule
    \end{tabular}
\end{table}

In practice, LangGraph's stateful edges allow the Safety Triage Agent to enforce risk-score thresholds before delegating to the Support Coach Agent, while still preserving the ability to retry, log, and escalate without bespoke infrastructure. This orchestration choice therefore underpins the evaluation metrics reported later for tool-call reliability, latency, and auditability.

\begin{figure}[h]
    \centering
    \resizebox{0.95\textwidth}{!}{%
    \begin{tikzpicture}[
        entity/.style={rectangle, rounded corners=4pt, draw=ugmBlue, very thick, fill=ugmBlue!6, align=center, minimum width=3.6cm, minimum height=1.4cm},
        external/.style={entity, fill=white},
        datastore/.style={rectangle, draw=ugmBlue, thick, fill=ugmBlue!10, align=center, minimum width=4.4cm, minimum height=1.2cm},
        innerbox/.style={rectangle, rounded corners=3pt, draw=ugmBlue!70, fill=ugmBlue!12, minimum width=3.0cm, minimum height=0.85cm, font=\footnotesize, align=center},
        arrow/.style={-Latex, thick, ugmBlue},
        dashedarrow/.style={-Latex, thick, ugmBlue, dashed}
    ]
        \node[external] (student) {Student\\\footnotesize UGM-AICare App};
        \node[entity, right=3.8cm of student, minimum width=4.6cm, minimum height=5.3cm] (backend) {};
        \node at (backend.north) [yshift=-0.35cm, font=\bfseries\small, text=ugmBlue] {Safety Agent Suite};
        \node[innerbox, fill=ugmGold!15] (aika) at ([yshift=1.25cm]backend.center) {Aika Meta-Agent};
        \node[innerbox] (sta) at ([yshift=0.45cm]backend.center) {Safety Triage Agent};
        \node[innerbox, below=0.2cm of sta] (sca) {Support Coach Agent};
        \node[innerbox, below=0.2cm of sca] (sda) {Service Desk Agent};
        \node[innerbox, below=0.2cm of sda] (ia) {Insights Agent};
        \node[external, right=3.6cm of backend] (staff) {University Staff\\\footnotesize Admin Dashboard};
        \node[datastore, below=1.8cm of backend] (dataplatform) {Data Platform\\\footnotesize Encrypted storage \& analytics};
        \node[external, above=1.8cm of backend] (llm) {LLM Service\\\footnotesize (Gemini API)};

        \draw[arrow] (student) -- node[above, font=\footnotesize]{Messages, mood signal} (backend);
        \draw[arrow] (backend) -- node[above, font=\footnotesize]{Coach responses, resources} (student);
        \draw[arrow] (backend) -- node[above, font=\footnotesize]{Alerts, insights} (staff);
        \draw[dashedarrow] (staff) -- node[below, font=\footnotesize]{Escalation reviews, configuration} (backend);
        \draw[arrow] (backend) -- node[right, font=\footnotesize]{Anonymised logs, metrics} (dataplatform);
        \draw[dashedarrow] (dataplatform) -- node[left, font=\footnotesize]{Aggregated reports} (backend);
        \draw[arrow] (backend) -- node[right, font=\footnotesize]{Structured prompts, tool calls} (llm);
        \draw[dashedarrow] (llm) -- node[left, font=\footnotesize]{Generated responses} (backend);
    \end{tikzpicture}}
    \caption{High-level context of the Safety Agent Suite showing primary stakeholders, orchestration boundaries, and data exchanges. Dashed arrows denote supervisory or configuration interactions.}
    \label{fig:context_diagram}
\end{figure}

%%%%%%%%%%%%%%%%%%%%%%%%%%%%%%%%%%%%%%%%%%%%%%%%%%%%%%%
%%% SECTION 3.3 - FUNCTIONAL ARCHITECTURE %%%
%%%%%%%%%%%%%%%%%%%%%%%%%%%%%%%%%%%%%%%%%%%%%%%%%%%%%%%

\section{Functional Architecture: The Agentic Core}
\label{chap:functional_architecture}

The functional architecture of the framework is designed as a Multi-Agent System (MAS), where the system's overall intelligence and capability emerge from the coordinated actions of its five components: four specialized agents and one meta-agent orchestrator. This section details the "what" of the system by defining the specific role, operational logic, and capabilities of each component within the \textbf{Safety Agent Suite}. Each specialist agent functions as a distinct component within the LangGraph state machine, perceiving its environment through the shared state, executing its logic, and updating the state with its results, while the Aika Meta-Agent coordinates their invocation and synthesizes their outputs.

\subsection{The Safety Triage Agent (STA): The Real-Time Guardian}

\subsubsection{Goal}
The primary objective of the STA is to function as a real-time, automated safety monitor for every user interaction. Its goal is to assess the immediate risk level of a user's conversation to detect potential crises and trigger an appropriate escalation protocol without delay, ensuring that safety is the foremost priority of the system.

\subsubsection{Perception (Inputs)}
The STA perceives the conversational environment by intercepting each user message before it is processed by other agents. Its primary input is the raw text of the user's current utterance. Let $M_t$ be the user's message at time $t$. The STA's perception is solely focused on this message:
\begin{itemize}
    \item \textbf{Current User Message ($M_t$):} A string containing the user's latest input.
\end{itemize}

\subsubsection{Processing Logic}
The core logic of the STA is a high-speed classification task. Upon receiving the message $M_t$, the agent invokes a specialized function, powered by the Gemini 2.5 Pro model, to classify the message into one of several predefined risk categories. The classification function, $f_{STA}$, can be represented as:
$$ R_t = f_{STA}(M_t; \theta_{LLM}) $$
where $\theta_{LLM}$ represents the parameters of the underlying Large Language Model, and the output, $R_t$, is an element of the set of possible risk levels, $R \in \{\text{Low, Moderate, Critical}\}$. The prompt for this classification is highly optimized for speed and accuracy, instructing the model to evaluate the text for indicators of self-harm, severe distress, or explicit requests for urgent help.

\subsubsection{Action (Outputs)}
Based on the classification result $R_t$, the STA's action is to update the system's state, which in turn determines the next step in the LangGraph workflow.
\begin{itemize}
    \item \textbf{State Update:} The agent's primary output is an update to the shared state graph, setting the \texttt{risk\_level} variable to the value of $R_t$.
    \item \textbf{Trigger Escalation (if $R_t$ = Critical):} If a critical risk is detected, the agent's action triggers a conditional edge in the graph that invokes the \texttt{escalate\_crisis} tool. This tool flags the conversation on the Admin Dashboard, logs the event, and instructs the Service Desk Agent (SDA) to create a high-priority case. It also immediately presents the user with pre-defined emergency resources.
\end{itemize}

\begin{itemize}
    \item \textbf{Design Rationale:} Assumes each incoming $M_t$ is UTF-8 text already filtered for profanity/noise; applies a calibrated confidence threshold (LLM softmax score $\geq0.6$) before labelling critical risk, otherwise defers to a human counselor; prompt template and \texttt{escalate\_crisis} schema were validated on the synthetic crisis corpus and scenario metrics reported in Chapter~\ref{sec:rq1}.\cite{FIND_CITATION_PLACEHOLDER}
\end{itemize}

\subsection{The Support Coach Agent (SCA): The Empathetic Mentor}

\subsubsection{Goal}
The SCA is the primary user-facing conversational agent, designed to provide personalized, evidence-based mental health coaching. Its goal is to engage the student in a supportive, empathetic dialogue, guiding them through structured self-help modules based on established therapeutic principles like Cognitive Behavioral Therapy (CBT), and to sustain engagement through gentle progress tracking and timely check-ins.

\subsubsection{Perception (Inputs)}
The SCA operates on the history of the conversation and the user's profile. Its key inputs from the state graph are:
\begin{itemize}
    \item \textbf{Conversation History ($H_{t-1}$):} The full transcript of the conversation up to the previous turn.
    \item \textbf{User's Current Message ($M_t$):} The message deemed safe by the STA.
    \item \textbf{User State:} Information about the user's progress, including completed modules and outstanding check-ins.
\end{itemize}

\subsubsection{Processing Logic}
The SCA's logic is generative and context-aware. It uses the Gemini 2.5 Pro model to generate a conversational response, $A_t$, that is empathetic and relevant to the user's message and history.
$$ A_t = f_{SCA}(M_t, H_{t-1}; \theta_{LLM}) $$
This agent has access to a toolset that allows it to retrieve and present structured content. When a user's query or the conversation flow indicates a need for a specific skill (e.g., managing anxiety), the SCA can decide to invoke its \texttt{retrieve\_cbt\_module} tool to fetch and present the relevant exercise.

\subsubsection{Action (Outputs)}
\begin{itemize}
    \item \textbf{Conversational Response:} A human-like text response to be displayed to the user.
    \item \textbf{Tool Call (Content Delivery):} Invocation of tools to present CBT exercises or other self-help modules.
    \item \textbf{Tool Call (Progress Logging):} Upon completion of a module, the SCA can call the \texttt{record\_module\_completion} tool to update the student's progress log in the shared state and database.
\end{itemize}

\begin{itemize}
    \item \textbf{Design Rationale:} Assumes the STA has sanitised $M_t$ and that $H_{t-1}$ retains fewer than 50 dialogue turns to bound context length; enforces refusal policies for out-of-scope clinical topics and escalates administrative intents to the SDA; prompts and tool schemas were peer-reviewed with counselling staff and evaluated via the rubric-based assessment in Chapter~\ref{sec:rq2}.\cite{FIND_CITATION_PLACEHOLDER}
\end{itemize}

\subsection{The Service Desk Agent (SDA): The Administrative Orchestrator}

\subsubsection{Goal}
The SDA functions as the administrative backbone of the support system. Its primary goal is to automate the operational workflows related to clinical case management and resource scheduling, thereby reducing the manual burden on university staff and ensuring that escalations and requests are handled efficiently and reliably.

\subsubsection{Perception (Inputs)}
The SDA is primarily triggered by events from other agents or direct commands from the Admin Dashboard. Its inputs are structured data, not conversational text:
\begin{itemize}
    \item \textbf{Escalation Event:} A signal from the STA containing the conversation ID and risk level of a flagged case.
    \item \textbf{Scheduling Request:} A structured request from the SCA (initiated by a user) containing the user's ID and desired appointment times.
    \item \textbf{Admin Commands:} Directives from a human administrator via the dashboard (e.g., "close case," "add note").
\end{itemize}

\subsubsection{Processing Logic}
The SDA's logic is procedural and tool-based. It does not engage in open-ended conversation but rather executes a sequence of pre-defined actions based on its inputs. For example, upon receiving an escalation event, its logic is to execute the \texttt{create\_case} tool, followed by the \texttt{assign\_case\_status} tool with the "New" parameter.

\subsubsection{Action (Outputs)}
\begin{itemize}
    \item \textbf{Database Operations:} The SDA's primary actions are database mutations, such as creating, updating, or closing case records in the clinical management database.
    \item \textbf{API Calls to External Services:} It can interact with external calendar systems to check for counselor availability and book appointments.
    \item \textbf{Notifications:} It sends automated email or dashboard notifications to counselors when a new case is assigned to them or when a student books an appointment.
\end{itemize}

\begin{itemize}
    \item \textbf{Design Rationale:} Assumes structured events carry validated UUIDs, ISO-8601 timestamps, and pass schema validation upstream; enforces idempotent tool execution with exponential backoff and escalates to human staff after three failed retries; tool schemas were exercised in integration tests summarised in Chapter~\ref{sec:rq2}.\cite{FIND_CITATION_PLACEHOLDER}
\end{itemize}

\subsection{The Insights Agent (IA): The Strategic Analyst}

\subsubsection{Goal}
The IA is designed to function as the institution's automated well-being analyst. Its goal is to autonomously process anonymized, aggregated conversation data to identify population-level mental health trends, sentiment shifts, and emerging topics of concern. This provides the institution with actionable, data-driven intelligence to inform resource allocation and proactive strategy.

\subsubsection{Perception (Inputs)}
The IA is activated by a time-based trigger (e.g., a weekly Cron job) and its primary input is the entire corpus of anonymized conversation logs.
\begin{itemize}
    \item \textbf{Time-Based Trigger:} A signal from the APScheduler to begin its analysis.
    \item \textbf{Anonymized Database Access:} Read-only access to the \texttt{conversation\_logs} table, from which all personally identifiable information (PII) has been redacted.
\end{itemize}

\subsubsection{Processing Logic}
The IA's logic involves a pipeline of Natural Language Processing (NLP) tasks performed on the collected data. This includes:
\begin{itemize}
    \item \textbf{Topic Modeling:} Using algorithms like Latent Dirichlet Allocation (LDA) or modern transformer-based clustering to identify the most prevalent topics of discussion (e.g., "exam stress," "social isolation").
    \item \textbf{Sentiment Analysis:} Calculating the overall sentiment score for the student population over the given period and tracking its change over time.
    \item \textbf{Summarization:} Using the Gemini 2.5 Pro model to generate concise, human-readable summaries of the key findings from the topic and sentiment analysis.
\end{itemize}

\subsubsection{Action (Outputs)}
\begin{itemize}
    \item \textbf{Structured Report Generation:} The final output is a structured report (e.g., in JSON or PDF format) containing visualizations (e.g., charts of topic frequency over time) and the generated summaries.
    \item \textbf{Dashboard Update:} The agent pushes this report to the Admin Dashboard, updating the analytics view for university staff.
    \item \textbf{Email Notification:} It can be configured to automatically email the report to a list of stakeholders, such as the head of counseling services.
\end{itemize}

\begin{itemize}
    \item \textbf{Design Rationale:} Assumes access to anonymised logs aggregated with $k\geq50$ users and retained for no longer than 90 days; enforces differential privacy noise addition (placeholder $\epsilon=1.0$) and suppresses outputs below the threshold; analytical prompts and pipelines were stress-tested via the stability checks in Chapter~\ref{sec:rq4}.\cite{FIND_CITATION_PLACEHOLDER}
\end{itemize}

\subsection{The Aika Meta-Agent: Unified Orchestration Layer}
\label{sec:aika_meta_agent}

While the four specialized agents (STA, SCA, SDA, IA) constitute the core intelligence of the Safety Agent Suite, their effective coordination requires an additional orchestration layer that addresses a fundamental challenge in multi-agent systems: how to present a unified, coherent interface to heterogeneous user roles while dynamically routing requests to the appropriate specialist based on intent classification, role-based access control, and conversational context \cite{wooldridge2009introductionmas, burguillo2017multiagentsystems}.

\subsubsection{Motivation and Formal Problem Statement}

In traditional multi-agent architectures, users must explicitly select their target agent, introducing decision friction and potential misrouting. This becomes particularly problematic in mental health applications where a single user utterance may require sequential processing by multiple agents (e.g., crisis detection followed by therapeutic coaching). Furthermore, different stakeholder classes (students, counselors, administrators) require fundamentally different interaction paradigms with the same underlying agent infrastructure.

The Aika Meta-Agent addresses this orchestration challenge by functioning as a \textbf{context-aware dispatcher and personality synthesizer}. The name "Aika", combining the Japanese characters for "love/affection" and "excellence/beautiful", encapsulates the system's dual mandate: delivering compassionate, human-centered support while maintaining technical rigor and operational excellence.

Formally, the orchestration problem can be stated as follows. Let $\mathcal{U} = \{u_1, u_2, \ldots, u_n\}$ represent the set of user messages, $\mathcal{R} = \{\texttt{student}, \texttt{counselor}, \texttt{admin}\}$ denote the set of authenticated user roles, and $\mathcal{H}_t = \{(u_1, a_1), (u_2, a_2), \ldots, (u_{t-1}, a_{t-1})\}$ represent the conversation history up to time $t$, where each tuple $(u_i, a_i)$ pairs a user utterance with the system's response. The agent space is defined as $\mathcal{A} = \{A_{\text{STA}}, A_{\text{SCA}}, A_{\text{SDA}}, A_{\text{IA}}\}$.

The orchestration function $\Phi_{\text{Aika}}$ must satisfy:
\begin{equation}
\Phi_{\text{Aika}}: \mathcal{U} \times \mathcal{R} \times \mathcal{H} \rightarrow \mathcal{A}^* \times \mathcal{P}
\label{eq:orchestration_function}
\end{equation}
where $\mathcal{A}^*$ denotes a sequence of agent invocations (allowing for multi-agent workflows), and $\mathcal{P}$ is the personality space encoding role-appropriate linguistic register, tone, and domain-specific terminology.

The orchestration must satisfy several invariants:
\begin{enumerate}
    \item \textbf{Safety First (Crisis Routing):} $\forall u \in \mathcal{U}, r = \texttt{student} \Rightarrow A_{\text{STA}} \in \Phi_{\text{Aika}}(u, r, \mathcal{H})$
    \item \textbf{Role-Based Access Control:} Privileged operations (case management, analytics) are only accessible to authenticated staff: $\Phi_{\text{Aika}}(u, \texttt{student}, \mathcal{H}) \cap \{A_{\text{SDA}}, A_{\text{IA}}\} = \emptyset$ for administrative queries
    \item \textbf{Deterministic Safety Escalation:} High-risk classifications must deterministically route to SDA: $f_{\text{STA}}(u) \geq \theta_{\text{critical}} \Rightarrow A_{\text{SDA}} \in \Phi_{\text{Aika}}(u, r, \mathcal{H})$
\end{enumerate}

\subsubsection{Architectural Pattern: Hierarchical Meta-Agent Coordination}

Aika implements a \textbf{hierarchical coordinator-specialist architecture} \cite{durfee1999distributedsystems, stone2000multiagentcoordination}, distinct from flat peer-to-peer agent negotiation schemes (e.g., Contract Net Protocol \cite{smith1980contractnet}) or fully decentralized market-based approaches. In this pattern, a meta-controller maintains global state and routing policies while delegating task execution to domain specialists.

The architecture can be formalized as a two-level hierarchy:
\begin{equation}
\text{Level 1 (Meta-Layer)}: \quad \mathcal{M} = (\mathcal{S}_{\text{global}}, \pi_{\text{route}}, \psi_{\text{synthesize}})
\label{eq:meta_layer}
\end{equation}
where:
\begin{itemize}
    \item $\mathcal{S}_{\text{global}}$ is the global state space containing user role, session context, and agent invocation history
    \item $\pi_{\text{route}}: \mathcal{S}_{\text{global}} \times \mathcal{U} \rightarrow \mathcal{A}$ is the routing policy function
    \item $\psi_{\text{synthesize}}: \mathcal{A}^* \times \mathcal{R} \rightarrow \text{Response}$ is the response synthesis function that aggregates specialist outputs into a role-coherent reply
\end{itemize}

\begin{equation}
\text{Level 2 (Specialist Layer)}: \quad \mathcal{S} = \{(A_i, \mathcal{D}_i, f_i) \mid i \in \{\text{STA, SCA, SDA, IA}\}\}
\label{eq:specialist_layer}
\end{equation}
where each specialist agent $A_i$ operates over its domain $\mathcal{D}_i$ with processing function $f_i$.

The routing policy $\pi_{\text{route}}$ is implemented as a compositional function:
\begin{equation}
\pi_{\text{route}}(s, u) = \pi_{\text{role}}(r) \circ \pi_{\text{intent}}(u, \mathcal{H}) \circ \pi_{\text{safety}}(u)
\label{eq:routing_composition}
\end{equation}
where:
\begin{itemize}
    \item $\pi_{\text{safety}}(u) \rightarrow \{A_{\text{STA}}\}$ for all student messages (safety-first invariant)
    \item $\pi_{\text{intent}}(u, \mathcal{H}) \rightarrow \mathcal{I}$ classifies user intent into $\mathcal{I} = \{\texttt{crisis}, \texttt{support}, \texttt{scheduling}, \texttt{analytics}\}$
    \item $\pi_{\text{role}}(r)$ applies role-specific constraints: $\pi_{\text{role}}(\texttt{student}) \cap \{A_{\text{IA}}\} = \emptyset$
\end{itemize}

\subsubsection{Role-Based Orchestration Workflows}

The meta-agent implements distinct workflow graphs for each user role, formalized as finite state machines over the agent space.

\paragraph{Student Workflow ($r = \texttt{student}$)}
For student interactions, the workflow implements a safety-first pipeline:
\begin{equation}
\Gamma_{\text{student}}(u, \mathcal{H}) = \begin{cases}
A_{\text{STA}} \rightarrow A_{\text{SCA}} \rightarrow \text{END} & \text{if } R(u) \in [\text{Low}, \text{Moderate}] \\
A_{\text{STA}} \rightarrow A_{\text{SDA}} \rightarrow \text{END} & \text{if } R(u) \in [\text{High}, \text{Critical}]
\end{cases}
\label{eq:student_workflow}
\end{equation}
where $R(u)$ is the risk classification output by $f_{\text{STA}}(u; \theta_{\text{LLM}})$.

The personality function for students is defined as:
\begin{equation}
\psi_{\texttt{student}}(a_{\text{specialist}}) = \text{Transform}(a_{\text{specialist}}, \mathcal{T}_{\text{empathetic}}, \mathcal{L}_{\text{informal-ID}})
\label{eq:student_personality}
\end{equation}
where $\mathcal{T}_{\text{empathetic}}$ represents empathetic tone markers (e.g., "Aku mengerti kamu sedang...") and $\mathcal{L}_{\text{informal-ID}}$ denotes informal Indonesian linguistic register.

\paragraph{Administrator Workflow ($r = \texttt{admin}$)}
For administrative users, the workflow bifurcates based on query classification:
\begin{equation}
\Gamma_{\text{admin}}(u, \mathcal{H}) = \begin{cases}
A_{\text{IA}} \rightarrow \text{END} & \text{if } \texttt{classify\_intent}(u) = \texttt{analytics} \\
A_{\text{SDA}} \rightarrow \text{END} & \text{if } \texttt{classify\_intent}(u) = \texttt{operational}
\end{cases}
\label{eq:admin_workflow}
\end{equation}

The personality transform for administrators emphasizes data-driven professionalism:
\begin{equation}
\psi_{\texttt{admin}}(a_{\text{specialist}}) = \text{Transform}(a_{\text{specialist}}, \mathcal{T}_{\text{analytical}}, \mathcal{L}_{\text{formal-ID/EN}})
\label{eq:admin_personality}
\end{equation}

\paragraph{Counselor Workflow ($r = \texttt{counselor}$)}
For clinical staff, the workflow provides integrated case management and insights:
\begin{equation}
\Gamma_{\text{counselor}}(u, \mathcal{H}) = \begin{cases}
A_{\text{SDA}} \rightarrow A_{\text{IA}} \rightarrow \text{END} & \text{if } \texttt{classify\_intent}(u) = \texttt{case\_query} \\
A_{\text{SDA}} \rightarrow A_{\text{SCA}} \rightarrow \text{END} & \text{if } \texttt{classify\_intent}(u) = \texttt{intervention\_plan}
\end{cases}
\label{eq:counselor_workflow}
\end{equation}

The personality adopts clinical terminology and evidence-based framing:
\begin{equation}
\psi_{\texttt{counselor}}(a_{\text{specialist}}) = \text{Transform}(a_{\text{specialist}}, \mathcal{T}_{\text{clinical}}, \mathcal{V}_{\text{CBT/therapeutic}})
\label{eq:counselor_personality}
\end{equation}
where $\mathcal{V}_{\text{CBT/therapeutic}}$ denotes the specialized vocabulary space for therapeutic interventions.

\subsubsection{LangGraph StateGraph Implementation}

The Aika orchestrator is implemented as a LangGraph StateGraph with typed state management. The state schema extends the base \texttt{SafetyAgentState}:
\begin{equation}
\texttt{AikaState} = \texttt{SafetyAgentState} \cup \{\texttt{user\_role}, \texttt{intent\_class}, \texttt{agent\_sequence}, \texttt{routing\_metadata}\}
\label{eq:aika_state}
\end{equation}

The graph structure implements the routing composition from Equation~\ref{eq:routing_composition}:
\begin{align}
\texttt{workflow} &= \texttt{StateGraph(AikaState)} \nonumber \\
\texttt{nodes} &= \{\texttt{classify\_intent}, \texttt{route\_by\_role}, \texttt{invoke\_sta}, \texttt{invoke\_sca}, \nonumber \\
&\quad\quad \texttt{invoke\_sda}, \texttt{invoke\_ia}, \texttt{synthesize\_response}\} \label{eq:langgraph_nodes}
\end{align}

Conditional edges implement the role-based workflows from Equations~\ref{eq:student_workflow}--\ref{eq:counselor_workflow}:
\begin{equation}
\texttt{add\_conditional\_edges}(\texttt{classify\_intent}, \lambda s: \pi_{\text{role}}(s.\texttt{user\_role}), \mathcal{E}_{\text{role}})
\label{eq:conditional_routing}
\end{equation}
where $\mathcal{E}_{\text{role}}$ maps role states to specialist invocation nodes.

Figure~\ref{fig:aika_orchestration} illustrates the complete orchestration flow, showing how Aika coordinates role-based routing to the Safety Agent Suite specialists.

\begin{figure}[h]
    \centering
    \begin{tikzpicture}[
        node distance=1.2cm and 1.5cm,
        % Academic style definitions
        metanode/.style={
            rectangle, 
            draw=black!80, 
            line width=0.8pt,
            fill=black!5, 
            align=center, 
            minimum width=4.5cm, 
            minimum height=1.2cm, 
            font=\small\sffamily
        },
        rolenode/.style={
            rectangle, 
            draw=black!60, 
            line width=0.6pt,
            fill=white, 
            align=center, 
            minimum width=2.8cm, 
            minimum height=0.9cm, 
            font=\footnotesize\sffamily
        },
        agentnode/.style={
            rectangle, 
            draw=black!70, 
            line width=0.6pt,
            fill=black!3, 
            align=center, 
            minimum width=2.5cm, 
            minimum height=0.85cm, 
            font=\footnotesize\sffamily
        },
        % Arrow styles
        inputarrow/.style={
            -Stealth, 
            line width=0.7pt,
            black!70
        },
        routearrow/.style={
            -Stealth, 
            line width=0.6pt,
            black!60
        },
        backarrow/.style={
            -Stealth, 
            line width=0.5pt,
            black!50,
            dashed
        },
        % Label styles
        edgelabel/.style={
            font=\scriptsize,
            align=center,
            fill=white,
            inner sep=1pt
        },
        % Layer boxes
        layerbox/.style={
            rectangle,
            draw=black!30,
            line width=0.4pt,
            rounded corners=2pt,
            inner sep=8pt
        }
    ]
        % User Role Layer
        \node[rolenode] (student) {\textbf{Student}\\{\scriptsize$r = \texttt{student}$}};
        \node[rolenode, right=of student] (admin) {\textbf{Administrator}\\{\scriptsize$r = \texttt{admin}$}};
        \node[rolenode, right=of admin] (counselor) {\textbf{Counselor}\\{\scriptsize$r = \texttt{counselor}$}};
        
        % Layer label for user roles
        \node[left=0.3cm of student, font=\scriptsize, text width=1.5cm, align=right] (userlabel) {\textit{User\\Roles}};
        
        % Meta-Agent Layer
        \node[metanode, below=2cm of admin] (aika) {
            \textbf{Aika Meta-Agent}\\
            {\scriptsize Orchestration: $\Phi_{\text{Aika}}(u, r, \mathcal{H})$}
        };
        
        % Layer label for meta-agent
        \node[left=0.3cm of aika, font=\scriptsize, text width=1.5cm, align=right] (metalabel) {\textit{Meta-\\Layer}};
        
        % Specialist Agent Layer
        \node[agentnode, below left=2.2cm and 3.5cm of aika] (sta) {
            \textbf{STA}\\
            {\scriptsize $f_{\text{STA}}$}
        };
        \node[agentnode, right=0.8cm of sta] (sca) {
            \textbf{SCA}\\
            {\scriptsize $f_{\text{SCA}}$}
        };
        \node[agentnode, right=0.8cm of sca] (sda) {
            \textbf{SDA}\\
            {\scriptsize $f_{\text{SDA}}$}
        };
        \node[agentnode, right=0.8cm of sda] (ia) {
            \textbf{IA}\\
            {\scriptsize $f_{\text{IA}}$}
        };
        
        % Layer label for specialist agents
        \node[left=0.3cm of sta, font=\scriptsize, text width=1.5cm, align=right] (agentlabel) {\textit{Specialist\\Agents}};
        
        % Input arrows from user roles to meta-agent
        \draw[inputarrow] (student.south) -- node[edgelabel, left, pos=0.4] {$u, \mathcal{H}_t$} (aika.north -| student.south);
        \draw[inputarrow] (admin.south) -- node[edgelabel, above, pos=0.3] {$u, \mathcal{H}_t$} (aika.north);
        \draw[inputarrow] (counselor.south) -- node[edgelabel, right, pos=0.4] {$u, \mathcal{H}_t$} (aika.north -| counselor.south);
        
        % Routing arrows from meta-agent to specialists
        \draw[routearrow] (aika.south) -- ++(-3.5,-0.8) -- node[edgelabel, above left, pos=0.7] {$\pi_{\text{route}}$} (sta.north);
        \draw[routearrow] (aika.south) -- ++(-1.3,-0.8) -- node[edgelabel, above left, pos=0.6] {$\Gamma_{\texttt{student}}$} (sca.north);
        \draw[routearrow] (aika.south) -- ++(1.3,-0.8) -- node[edgelabel, above right, pos=0.6] {$\Gamma_{\texttt{admin}}$} (sda.north);
        \draw[routearrow] (aika.south) -- ++(3.5,-0.8) -- node[edgelabel, above right, pos=0.7] {$\Gamma_{\texttt{counselor}}$} (ia.north);
        
        % Response aggregation arrows (dashed)
        \draw[backarrow] (sta.north) -- ++(0,0.5) -| node[edgelabel, above, pos=0.12] {\scriptsize$\psi_{\text{synthesize}}$} (aika.south);
        \draw[backarrow] (sca.north) -- ++(0,0.3) -| (aika.south);
        \draw[backarrow] (sda.north) -- ++(0,0.3) -| (aika.south);
        \draw[backarrow] (ia.north) -- ++(0,0.5) -| (aika.south);
        
        % Agent descriptions below specialist layer
        \node[below=0.15cm of sta, font=\scriptsize, text width=2.3cm, align=center, text=black!60] {
            Crisis Detection\\$R(u) \in [0,3]$
        };
        \node[below=0.15cm of sca, font=\scriptsize, text width=2.3cm, align=center, text=black!60] {
            Empathetic\\Coaching
        };
        \node[below=0.15cm of sda, font=\scriptsize, text width=2.3cm, align=center, text=black!60] {
            Case Management\\Operations
        };
        \node[below=0.15cm of ia, font=\scriptsize, text width=2.3cm, align=center, text=black!60] {
            Privacy-Aware\\Analytics
        };
        
        % Academic-style legend
        \node[below=3.2cm of aika, font=\scriptsize, text width=12cm, align=left] (legend) {
            \textbf{Notation:} 
            $u$ = user message; 
            $r$ = user role; 
            $\mathcal{H}_t$ = conversation history at time $t$; 
            $\pi_{\text{route}}$ = routing policy function; 
            $\Gamma_r$ = role-specific workflow; 
            $\psi_{\text{synthesize}}$ = response synthesis function. 
            Solid arrows represent request routing; dashed arrows represent response aggregation.
        };
    \end{tikzpicture}
    \caption{Hierarchical architecture of the Aika Meta-Agent orchestration system. The meta-layer receives user inputs with role context ($r$) and conversation history ($\mathcal{H}_t$), applies routing policy $\pi_{\text{route}}$ to invoke appropriate specialist agents, and synthesizes responses via $\psi_{\text{synthesize}}$ with role-appropriate personality transforms. The architecture implements the orchestration function from Equation~\ref{eq:orchestration_function} with role-based workflows defined in Equations~\ref{eq:student_workflow}--\ref{eq:counselor_personality}.}
    \label{fig:aika_orchestration}
\end{figure}

\subsubsection{Complexity Analysis and Performance Characteristics}

The orchestration overhead can be analyzed through computational complexity and latency budgets.

\paragraph{Time Complexity}
The routing decision $\pi_{\text{route}}$ involves three sequential operations:
\begin{equation}
T_{\text{orchestration}} = T_{\texttt{auth}}(r) + T_{\texttt{intent}}(u) + T_{\texttt{lookup}}(\mathcal{A})
\label{eq:orchestration_time}
\end{equation}
where:
\begin{itemize}
    \item $T_{\texttt{auth}}(r) = O(1)$ for role verification via JWT token validation
    \item $T_{\texttt{intent}}(u) = O(|u| \cdot d_{\text{LLM}})$ for LLM-based intent classification, where $d_{\text{LLM}}$ is the model's computational depth
    \item $T_{\texttt{lookup}}(\mathcal{A}) = O(1)$ for hash-based agent registry lookup
\end{itemize}

Empirically, intent classification dominates: $T_{\texttt{intent}} \in [100, 200]$ ms (p95), contributing $\approx 10\%$ to the end-to-end latency budget of 1.5s defined in Section~\ref{chap:technical_architecture}.

\paragraph{Space Complexity}
The global state $\mathcal{S}_{\text{global}}$ maintains:
\begin{equation}
|\mathcal{S}_{\text{global}}| = O(|\mathcal{H}|) + O(|\mathcal{R}|) + O(|\mathcal{A}|) = O(k \cdot n + 3 + 4) \approx O(n)
\label{eq:state_complexity}
\end{equation}
where $k$ is the maximum conversation history length (bounded at 50 turns per Section~\ref{chap:functional_architecture}) and $n$ is the average message length. This scales linearly with conversation depth, making it tractable for real-time operation.

\subsubsection{Advantages, Trade-offs, and Design Rationale}

The Aika Meta-Agent architecture provides several formal guarantees and practical benefits:

\begin{enumerate}
    \item \textbf{Safety Invariant Preservation:} By enforcing $A_{\text{STA}} \in \Phi_{\text{Aika}}(u, \texttt{student}, \mathcal{H})$ for all student messages, the meta-layer ensures crisis detection cannot be bypassed through direct agent access.
    
    \item \textbf{Cognitive Load Reduction:} Users interface with a single coherent AI entity ($\Phi_{\text{Aika}}$) rather than selecting from $|\mathcal{A}| = 4$ specialist agents, reducing decision friction by $O(|\mathcal{A}|)$ choices per interaction.
    
    \item \textbf{Role-Based Access Control:} The routing constraints $\pi_{\text{role}}(r)$ enforce privilege separation: students cannot access administrative analytics ($A_{\text{IA}}$) or case management ($A_{\text{SDA}}$), preventing unauthorized data exposure.
    
    \item \textbf{Conversational Coherence:} The synthesis function $\psi_{\text{synthesize}}$ maintains personality consistency across multi-agent workflows, avoiding jarring tone shifts that would occur in naive agent chaining.
    
    \item \textbf{Modularity and Maintainability:} Changes to specialist agent logic (e.g., updating CBT prompts in $f_{\text{SCA}}$) do not require modifications to $\Phi_{\text{Aika}}$, as long as input/output schemas remain stable.
\end{enumerate}

However, this design introduces measurable trade-offs:

\begin{enumerate}
    \item \textbf{Latency Overhead:} Intent classification adds $T_{\texttt{intent}} \approx 150$ ms (median) before specialist invocation, increasing p95 end-to-end latency from $\approx 1.3$s (direct agent access) to $\approx 1.5$s (via Aika). This overhead is deemed acceptable given the $<2$s threshold for conversational UI responsiveness \cite{nielsen1993responsetimes}.
    
    \item \textbf{Single Point of Failure:} The centralized orchestration pattern makes $\Phi_{\text{Aika}}$ a critical component whose failure blocks all user interactions. This is mitigated through stateless implementation (enabling horizontal scaling) and circuit breaker patterns for LLM API failures.
    
    \item \textbf{Prompt Engineering Complexity:} Maintaining role-consistent personalities across $|\mathcal{R}| = 3$ roles and $|\mathcal{A}| = 4$ agents requires careful curation of $\psi_{\text{synthesize}}$ transforms, validated through user acceptance testing (not formalized in this prototype).
\end{enumerate}

\subsubsection{Integration with Evaluation Framework}

The Aika Meta-Agent's performance is evaluated indirectly through the metrics framework defined in Chapter~\ref{sec:rq1}--\ref{sec:rq4}:

\begin{enumerate}
    \item \textbf{Routing Accuracy (RQ2):} Let $\mathcal{E}_{\text{route}}$ denote routing errors (misclassified intents). The orchestrator's contribution to workflow reliability is measured as:
    \begin{equation}
    \text{Accuracy}_{\text{route}} = 1 - \frac{|\mathcal{E}_{\text{route}}|}{|\mathcal{U}_{\text{test}}|}
    \label{eq:routing_accuracy}
    \end{equation}
    Target: $\text{Accuracy}_{\text{route}} \geq 0.95$ (i.e., $<5\%$ misrouting rate).
    
    \item \textbf{Latency Contribution (RQ1, RQ2):} The orchestration overhead is incorporated into end-to-end measurements:
    \begin{equation}
    T_{\text{total}} = T_{\text{orchestration}} + \sum_{A_i \in \Phi_{\text{Aika}}(u,r,\mathcal{H})} T_{A_i}
    \label{eq:total_latency}
    \end{equation}
    Target: $T_{\text{orchestration}}$ contributes $<15\%$ to $T_{\text{total}}$ (measured via p95 latency breakdown).
    
    \item \textbf{Safety Escalation Preservation (RQ1):} The meta-agent must not introduce false negatives in crisis routing:
    \begin{equation}
    \forall u : f_{\text{STA}}(u) \geq \theta_{\text{critical}} \Rightarrow A_{\text{SDA}} \in \Phi_{\text{Aika}}(u, \texttt{student}, \mathcal{H})
    \label{eq:safety_preservation}
    \end{equation}
    This invariant is verified through crisis corpus testing (Section~\ref{sec:rq1}).
\end{enumerate}

\subsubsection{Positioning in Multi-Agent Systems Literature}

The Aika Meta-Agent instantiates a \textbf{mediator pattern} \cite{gamma1994designpatterns} within the multi-agent systems framework, combining elements of:
\begin{itemize}
    \item \textbf{Blackboard architectures} \cite{engelmore1988blackboard}, where $\mathcal{S}_{\text{global}}$ serves as shared knowledge space
    \item \textbf{Hierarchical task networks} \cite{erol1994htn}, where $\Gamma_r$ decomposes high-level goals into specialist subtasks
    \item \textbf{Belief-Desire-Intention (BDI) orchestration} \cite{rao1995bdi}, where routing policies encode institutional-level intentions (safety-first mandate, RBAC compliance)
\end{itemize}

This design contrasts with fully decentralized approaches (e.g., multi-agent reinforcement learning \cite{busoniu2008marl}) by prioritizing deterministic safety guarantees and explainable routing decisions over emergent coordination, a critical requirement for safety-critical healthcare applications \cite{topol2019deepmedicine}.

By introducing the Aika Meta-Agent as the fifth component of the framework, the system achieves a synthesis of specialized expertise (via the four Safety Agent Suite agents) and unified user experience (via centralized orchestration with personality adaptation). This architectural layering enables the framework to scale from individual student support to institution-wide analytics while maintaining the safety-first invariants that define its clinical validity.

\begin{figure}[h]
    \centering
    \resizebox{0.95\textwidth}{!}{%
    \begin{tikzpicture}[
        node distance=2.3cm,
        actor/.style={rectangle, rounded corners=4pt, draw=ugmBlue, very thick, fill=white, align=center, minimum width=2.8cm, minimum height=1.1cm},
        agent/.style={rectangle, rounded corners=4pt, draw=ugmBlue, thick, fill=ugmBlue!7, align=center, minimum width=3.0cm, minimum height=1.1cm, font=\footnotesize},
        datastore/.style={rectangle, draw=ugmBlue, thick, fill=ugmBlue!10, align=center, minimum width=3.0cm, minimum height=1.1cm, font=\footnotesize},
        arrow/.style={-Latex, thick, ugmBlue},
        dashedarrow/.style={-Latex, thick, ugmBlue, dashed}
    ]
        \node[actor] (user) {Student};
        \node[agent, right=of user] (sta) {Safety Triage Agent};
        \node[agent, right=of sta] (sca) {Support Coach Agent};
        \node[actor, right=of sca] (studentReturn) {Student Response};

        \draw[arrow] (user) -- node[above, font=\footnotesize]{Message} (sta);
        \draw[arrow] (sta) -- node[above, font=\footnotesize]{Safe prompt} (sca);
        \draw[arrow] (sca) -- node[above, font=\footnotesize]{Coaching reply} (studentReturn);

        \node[agent, below=2.0cm of sta] (sda) {Service Desk Agent};
        \node[actor, right=of sda] (admin) {Counselor / Staff};
        \draw[arrow] (sta) -- node[left, font=\footnotesize]{Escalation} (sda);
        \draw[arrow] (sda) -- node[above, font=\footnotesize]{Case alert} (admin);
        \draw[dashedarrow] (admin) -- node[below, font=\footnotesize]{Resolution update} (sda);

        \node[datastore, below=2.0cm of sca] (progress) {Progress Logs};
        \draw[arrow] (sca) -- node[right, font=\footnotesize]{Module completion} (progress);

        \node[agent, below=1.6cm of progress] (ia) {Insights Agent};
        \node[datastore, right=of ia] (analytics) {Aggregated Reports};
        \draw[arrow] (progress) -- node[right, font=\footnotesize]{Anonymised traces} (ia);
        \draw[arrow] (ia) -- node[above, font=\footnotesize]{Weekly summary} (analytics);
        \draw[arrow] (ia) -- node[left, font=\footnotesize]{Trend alert} (admin);
    \end{tikzpicture}}
    \caption{Data flow between the Safety Agent Suite and its users. Solid arrows show operational data paths; dashed arrows show supervisory feedback.}
    \label{fig:dfd}
\end{figure}


\section{Security and Privacy Threat Model}
\label{sec:threat_model}

Protecting student data and maintaining institutional trust require an explicit articulation of adversaries, assets, and mitigations. Table~\ref{tab:threat_model} summarises the threat model adopted for the Safety Agent Suite, drawing on STRIDE/LINDDUN heuristics and privacy-by-design obligations.\cite{FIND_CITATION_PLACEHOLDER}

\begin{table}[h]
    \centering
    \caption{Threat model overview.}
    \label{tab:threat_model}
    \begin{tabular}{p{3.0cm}p{3.5cm}p{3.5cm}p{3.5cm}}
        \toprule
        \textbf{Actor / Threat} & \textbf{Targeted Asset} & \textbf{Likely Impact} & \textbf{Mitigations / Controls} \\
        \midrule
        Compromised student account & Conversation logs, risk flags & Exposure of sensitive disclosures; spoofed escalations & MFA and device attestation (frontend); STA confidence thresholds with human verification; audit trail review (Chapter~\ref{sec:rq1}). \\
        Malicious insider (staff) & Case notes, progress logs & Unauthorised browsing or data exfiltration & Role-based access control, immutable audit logs, case access alerts, quarterly review. \\
        External attacker (API abuse) & Backend endpoints, tooling & Prompt injection, denial of service, data tampering & API gateway with rate limiting, input sanitation, LangGraph guardrails, automated anomaly detection on latency/tool-failure metrics. \\
        Analytics linkage attack & Aggregated insights & Re-identification through small cohorts & Minimum cohort size $k\geq50$, differential privacy noise (placeholder $\epsilon=1.0$), suppression of rare categories (Chapter~\ref{sec:rq4}). \\
        Model supply-chain risks & LLM outputs / prompts & Hallucinated or unsafe responses & Structured prompts, refusal and escalation policies, prompt/response logging with red-team testing cadence. \\
        Infrastructure failure & Agent orchestration state & Service outage, message loss & Stateless frontend, checkpointed LangGraph state, automated failover for database replicas, latency SLOs monitored (Section~\ref{chap:technical_architecture}). \\
        \bottomrule
    \end{tabular}
\end{table}

These mitigations align with institutional policies and inform the evaluation metrics in Chapter~IV (e.g., STA sensitivity to avoid under- or over-escalation) and the privacy discussion in Chapter~V. Residual risks---such as novel jailbreak prompts or emergent privacy attacks---are addressed through scheduled red-team exercises, update audits for commercial APIs, and continuous monitoring of anomaly indicators.


%%%%%%%%%%%%%%%%%%%%%%%%%%%%%%%%%%%%%%%%%%%%%%%%%%%%%%%
%%% SECTION 3.4 - TECHNICAL ARCHITECTURE %%%
%%%%%%%%%%%%%%%%%%%%%%%%%%%%%%%%%%%%%%%%%%%%%%%%%%%%%%%

\section{Technical Architecture}
\label{chap:technical_architecture}

This section details the "how" of the system, providing the engineering blueprint for the agentic AI framework. The architecture is designed following a modern, service-oriented pattern, which decouples the primary components of the system into distinct, independently deployable services. This approach enhances maintainability, scalability, and promotes a clean separation of concerns \cite{newman2021buildingmicroservices, richards2020softwarearchitecture}. The framework consists of three core services: a unified frontend application, a backend service that houses the agentic core, and a data persistence service for all storage needs.

\subsection{Overall System Architecture}
\label{sec:overall_system_architecture}

The overall technical architecture is visualized in Figure \ref{fig:system_architecture_diagram}. It is a monolithic frontend-backend structure composed of three primary services that work in concert to deliver the full functionality of the framework to both students and administrators.

\begin{enumerate}
    \item \textbf{Frontend Service (UGM-AICare Web Application):} This is a comprehensive web application built using the \textbf{Next.js} framework. It serves two distinct user-facing roles from a single codebase:
        \begin{itemize}
            \item \textbf{The User Portal:} This is the interface for students. It provides access to features such as a journaling system, a user dashboard for tracking progress, and the `/aika` chat interface for direct interaction with the Support Coach Agent (SCA) and Safety Triage Agent (STA). It also handles features like appointment scheduling with counselors.
            \item \textbf{The Admin Dashboard:} This is a secure, role-protected area of the application for university staff and counselors. Its responsibilities include rendering the analytics and insights provided by the Insights Agent (IA), displaying real-time alerts for flagged conversations, and providing a case management system to act on escalations from the STA and SDA.
        \end{itemize}
    \item \textbf{Backend Service (The Agentic Core):} This service is the "brain" of the entire operation, built using the \textbf{FastAPI} Python framework. It exposes a \textbf{REST API} through which the unified Next.js frontend communicates. The backend is responsible for handling all business logic, including processing incoming chat messages from the User Portal, orchestrating the agents within the LangGraph state machine, making calls to the Google Gemini API, and interacting with the database. The asynchronous capabilities of FastAPI are critical for efficiently managing multiple concurrent conversations and long-running agentic tasks.
    \item \textbf{Data Persistence Service:} A \textbf{PostgreSQL} relational database serves as the single source of truth for the system. It is responsible for storing all persistent data, including user information (anonymized), conversation logs, clinical case data, and generated reports.
\end{enumerate}

\subsection{Backend Service: The Agentic Core}
\label{sec:backend_service}

The backend service is the central nervous system and cognitive engine of the entire framework. It is a Python-based application responsible for executing all business logic, orchestrating the agentic workflows, and serving as the intermediary between the user-facing application and the data persistence layer. To meet the demanding requirements of a real-time, AI-powered conversational system, the backend is built upon a modern, high-performance technology stack.

\subsubsection{API Framework: FastAPI}
\label{sec:api_framework}

The foundation of the backend service is the \textbf{FastAPI} framework. This choice was made after careful consideration of several alternatives, based on its specific suitability for building high-performance, API-driven services that interact with machine learning models \cite{ramirez2023fastapi, tiangolo2022fastapi}. The primary justifications for its selection are:

\begin{itemize}
    \item \textbf{Asynchronous Support:} FastAPI is built on top of ASGI (Asynchronous Server Gateway Interface), allowing it to handle requests asynchronously. This is a critical requirement for this framework, as interactions with the Google Gemini API are I/O-bound operations. Asynchronous handling ensures that the server can manage multiple concurrent user conversations and long-running agentic tasks without blocking, leading to a highly responsive and scalable system.
    \item \textbf{High Performance:} Leveraging Starlette for web routing and Pydantic for data validation, FastAPI is one of the fastest Python web frameworks available, delivering performance on par with NodeJS and Go applications \cite{ramirez2023fastapi, tiangolo2022fastapi}. This is essential for minimizing latency in the real-time chat interface.
    \item \textbf{Data Validation and Serialization:} FastAPI uses Pydantic type hints to enforce rigorous data validation for all incoming and outgoing API requests. This not only reduces the likelihood of data-related bugs but also automatically serializes data to and from JSON, streamlining the development process.
    \item \textbf{Automatic Interactive Documentation:} The framework automatically generates interactive API documentation (via Swagger UI and ReDoc) based on the Pydantic models. This creates a reliable, always-up-to-date contract for the frontend team and simplifies the testing and debugging process.
\end{itemize}

The backend exposes a RESTful API for all communication with the frontend service. The design follows standard REST principles, using conventional HTTP methods to perform operations on resources. A summary of key endpoints is provided in Table \ref{tab:api_endpoints}.

\begin{table}[h]
    \centering
    \caption{Key Endpoints of the Backend REST API.}
    \label{tab:api_endpoints}
    \begin{tabular}{lll}
        \toprule
        \textbf{Method} & \textbf{Endpoint} & \textbf{Description} \\
        \midrule
        \texttt{POST} & \texttt{/api/chat/message} & Submits a user message for processing by the agentic core. \\
        \texttt{GET} & \texttt{/api/insights/latest} & Fetches the latest strategic report from the Insights Agent. \\
        \texttt{POST} & \texttt{/api/appointments} & Creates a new appointment with a counselor via the SDA. \\
        \texttt{GET} & \texttt{/api/admin/cases} & Retrieves all flagged cases for the admin dashboard. \\
        \bottomrule
    \end{tabular}
\end{table}

\subsubsection{Agent Orchestration: LangGraph}

To manage the complex, cyclical, and stateful interactions between the agents, the framework employs \textbf{LangGraph}. LangGraph extends the linear "chain" paradigm of LangChain by modeling agentic workflows as a state graph, which is essential for building robust multi-agent systems \cite{mathew2025largelanguagemodelagents, barua2024llmagentsreview}.

The core of the orchestration is a central \textbf{State Graph}, where the application's state is explicitly defined and passed between nodes. This state object, implemented as a Pydantic class, contains all relevant information for a given workflow, such as the full \texttt{conversation\_history}, the \texttt{current\_risk\_level} as determined by the STA, and the \texttt{active\_case\_id}. 

The workflow is structured as follows:
\begin{itemize}
    \item \textbf{Nodes:} Each of the five framework components (Aika Meta-Agent for routing, plus the four specialist agents: STA, SCA, SDA, IA) and their associated tools are implemented as nodes in the graph. A node is a function that receives the current state, performs its task (e.g., makes an LLM call, queries the database), and returns a dictionary of updates to be merged back into the state.
    \item \textbf{Edges:} The flow of control between nodes is managed by edges. Crucially, the framework uses \textbf{conditional edges} to implement the agentic logic. After a node executes, a routing function inspects the updated state to decide which node to call next. For example, after the STA node classifies a message, a conditional edge checks the \texttt{current\_risk\_level} in the state. If the level is `CRITICAL`, the edge routes the workflow to the SDA node to create a case; otherwise, it routes to the SCA node to continue the conversation. This structure is visualized in Figure \ref{fig:langgraph_conceptual}.
\end{itemize}

This stateful, cyclical approach allows for sophisticated agentic behaviors, such as retrying failed tool calls, handing off tasks between agents, and maintaining a durable memory of the interaction, which are critical for the reliability and safety of the system.

\begin{figure}[h]
    \centering
    \begin{tikzpicture}[
        state/.style={rectangle, rounded corners=4pt, draw=ugmBlue, thick, align=center, fill=ugmBlue!6, minimum width=3.0cm, minimum height=1.1cm, font=\footnotesize},
        decision/.style={diamond, draw=ugmGold!80!black, thick, fill=ugmGold!20, aspect=2, align=center, font=\footnotesize},
        arrow/.style={-Latex, thick, ugmBlue},
        riskarrow/.style={-Latex, thick, ugmGold}
    ]
        \node[state] (entry) {User Message\\(turn $t$)};
        \node[state, right=2.6cm of entry] (sta) {Safety Triage Agent\\(risk assessment)};
        \node[decision, right=2.5cm of sta] (risk) {Risk Level?};
        \node[state, above right=1.6cm and 2.1cm of risk] (sca) {Support Coach Agent\\(empathetic response)};
        \node[state, below right=1.6cm and 2.1cm of risk] (sda) {Service Desk Agent\\(case actions)};
        \node[state, right=3.1cm of sca] (userout) {Response to Student};
        \node[state, right=3.1cm of sda] (adminout) {Admin Dashboard\\Escalation Record};

        \draw[arrow] (entry) -- node[above, font=\footnotesize]{contextual state} (sta);
        \draw[arrow] (sta) -- node[above, font=\footnotesize]{risk score $R_t$} (risk);
        \draw[riskarrow] (risk) -- node[above, font=\footnotesize]{Low/Moderate} (sca);
        \draw[riskarrow] (risk) -- node[below, font=\footnotesize]{Critical} (sda);
        \draw[arrow] (sca) -- node[above, font=\footnotesize]{coaching reply} (userout);
        \draw[arrow] (sda) -- node[above, font=\footnotesize]{case ticket, alert} (adminout);
        \draw[arrow, looseness=1.1, out=170, in=190] (sca.west) to node[above, font=\footnotesize]{next turn} (sta.north);
        \draw[arrow, looseness=1.1, out=-170, in=-190] (sda.west) to node[below, font=\footnotesize]{status updates} (sta.south);
        \draw[arrow, dashed, very thick, color=ugmBlue!70] (sca.south) -- node[right, font=\footnotesize]{booking request} (sda.north);
    \end{tikzpicture}
    \caption{Conceptual LangGraph state machine showing how conversation turns pass through the Safety Triage Agent before branching to the Support Coach or Service Desk agents, with feedback loops preserving state. The Aika Meta-Agent orchestrates entry into this workflow based on user role and intent classification (see Figure~\ref{fig:aika_orchestration}).}
    \label{fig:langgraph_conceptual}
\end{figure}

\subsubsection{Asynchronous Task Scheduling}

To facilitate the proactive, long-term analysis performed by the Insights Agent (IA), the framework requires a mechanism for scheduling periodic tasks. Instead of relying on an external workflow orchestration tool like n8n, a task scheduler is integrated directly into the FastAPI backend service.

For this purpose, the \textbf{APScheduler} (Advanced Python Scheduler) library is utilized. This choice was made for the following reasons:
\begin{itemize}
    \item \textbf{Integration and Simplicity:} As a Python library, APScheduler integrates seamlessly into the FastAPI application's event loop. This avoids the operational complexity and additional infrastructure requirements of deploying and maintaining a separate workflow management service.
    \item \textbf{Sufficient Functionality:} For the primary requirement of running the IA's analysis on a fixed schedule (e.g., weekly), APScheduler's cron-style triggering is perfectly suited and provides a lightweight yet robust solution.
\end{itemize}
The scheduler is configured to trigger the IA's main analysis function at a predefined interval. This function then executes its NLP pipeline, generates the strategic report, and pushes the results to the database and relevant stakeholders, thus closing the strategic oversight loop of the framework without manual intervention.

\subsection{Frontend Service: The UGM-AICare Web Application}

The frontend service is the primary human-computer interface for the entire framework, serving both students and administrative staff. It is engineered as a monolithic frontend application using the \textbf{Next.js} React framework. This choice was deliberate, allowing for the development and maintenance of two distinct user experiences, the public-facing User Portal and the secure Admin Dashboard within a single, cohesive codebase. This approach simplifies dependency management and ensures a consistent design language across the platform while leveraging Next.js's powerful features for routing and role-based access control \cite{vercel2024nextjsdocs, granicz2022modernreact}.

The selection of Next.js is justified by several key architectural advantages that directly support the project's requirements:

\begin{itemize}
    \item \textbf{Hybrid Rendering Strategies:} Next.js provides the flexibility to employ different rendering strategies on a per-page basis. For the dynamic, data-heavy Admin Dashboard, \textbf{Server-Side Rendering (SSR)} can be utilized to ensure that staff always see the most up-to-date case information and analytics. For the public-facing User Portal, a combination of SSR for dynamic content (like the user's dashboard) and \textbf{Static Site Generation (SSG)} for informational pages ensures both data freshness and optimal performance.
    \item \textbf{Component-Based Architecture:} Built upon React, Next.js facilitates a modular and reusable component-based architecture. This allows for the creation of discrete UI components (e.g., the chat window, dashboard widgets, journaling entries) that can be developed, tested, and maintained in isolation, significantly improving the scalability and maintainability of the codebase.
    \item \textbf{Integrated API Routes:} Next.js includes a built-in capability to create API routes within the same project. While the primary business logic resides in the separate FastAPI backend, this feature is leveraged to handle server-side frontend tasks, such as proxying requests to the backend API, securely managing session tokens, and hiding sensitive API keys from the client-side browser.
\end{itemize}

The application is functionally divided into two main areas:

\subsubsection{The User Portal}
This is the student-facing portion of the application, designed to be an accessible and engaging entry point to the university's mental health resources. Its key functional components include:
\begin{itemize}
    \item \textbf{The `/aika` Conversational Interface:} A real-time chat component that serves as the primary interaction point with the Support Coach Agent (SCA) and the underlying Safety Triage Agent (STA). It is responsible for managing the state of the conversation and rendering responses from the backend.
    \item \textbf{Journaling System:} A private, secure feature allowing students to write and review personal journal entries, a common practice in CBT-based therapies.
    \item \textbf{User Dashboard:} A personalized space where students can track their progress through coaching modules, revisit completed exercises, and see reminders for upcoming check-ins.
    \item \textbf{Appointment Scheduling:} An interface that communicates with the Service Desk Agent (SDA) via the backend API to allow students to view available slots and book appointments with human counselors.
\end{itemize}

\subsubsection{The Admin Dashboard}
This is a secure, authentication-protected area of the application designed for counselors and administrative staff. It functions as the central control and oversight panel for the entire agentic framework. Key features include:
\begin{itemize}
    \item \textbf{Insights Visualization:} Renders the reports and data visualizations generated by the Insights Agent (IA), providing staff with a clear, actionable overview of student well-being trends.
    \item \textbf{Real-Time Case Management:} Displays alerts for conversations flagged as "critical" by the STA. It provides an interface for counselors to review the flagged conversation, manage the case status, and document actions taken, directly interacting with the workflows managed by the SDA.
    \item \textbf{System Configuration:} Provides an interface for administrators to configure certain parameters of the agentic system, such as the email list for IA reports or the thresholds for proactive interventions.
\end{itemize}

All dynamic data and actions within both the User Portal and the Admin Dashboard are handled through asynchronous requests to the backend REST API, ensuring a clean and complete separation between the presentation layer (frontend) and the business logic and agentic core (backend).

\subsection{Data Persistence Layer: PostgreSQL}

The data persistence layer is the architectural component responsible for the storage, retrieval, and management of all long-term data within the framework. For this system, \textbf{PostgreSQL}, a powerful, open-source object-relational database system, was selected as the data persistence service. This decision was based on its robustness, feature set, and suitability for an application that handles structured, relational, and sensitive data \cite{stonebraker2018postgresql, juba2021postgresql}.

The choice of a relational database model, and PostgreSQL specifically, is justified by the following key factors:

\begin{itemize}
    \item \textbf{Data Integrity and ACID Compliance:} The nature of the application, which involves managing user interactions, clinical case escalations, and appointments, requires strong guarantees of data integrity. PostgreSQL's full compliance with ACID (Atomicity, Consistency, Isolation, Durability) properties ensures that all transactions are processed reliably. This is a non-negotiable requirement for a system where a missed escalation or a lost conversation log could have significant consequences.
    \item \textbf{Structured and Relational Data Model:} The data generated by the framework is inherently relational. There are clear, defined relationships between users, their conversation sessions, the messages within those sessions, and the clinical cases that may arise from them. A relational schema allows for the enforcement of these relationships at the database level through foreign key constraints, ensuring a consistent and logical data model.
    \item \textbf{Scalability and Concurrency Control:} PostgreSQL is renowned for its robust implementation of Multi-Version Concurrency Control (MVCC), which allows for high concurrency by enabling read operations to occur without blocking write operations. This is critical for the system's architecture, as the Insights Agent (IA) will perform large-scale read queries for analytics, while the real-time agents (STA, SCA, SDA) will be continuously writing new data from user interactions.
    \item \textbf{Extensibility and Advanced Features:} PostgreSQL supports a rich set of data types, advanced indexing capabilities, and powerful query optimization. This provides the flexibility to handle complex analytical queries from the IA efficiently and to extend the database schema in the future without requiring a migration to a different database system.
\end{itemize}

The backend service is the sole component with direct credentials to access the database. All interactions from the frontend are proxied through the backend's REST API, which enforces business logic and authorization before any database transaction is executed. This centralized access model is a critical security measure that prevents direct, unauthorized access to the data persistence layer.

The detailed logical structure of the database, including the table schemas and their relationships, will be presented in the \textbf{Database Design} section, which includes a full Entity-Relationship Diagram (ERD).


%%%%%%%%%%%%%%%%%%%%%%%%%%%%%%%%%%%%%%%%%%%%%%%%%%%%%%%%
%%% SECTION 3.5 - PRIVACY AND ETHICAL SAFEGUARDS %%%
%%%%%%%%%%%%%%%%%%%%%%%%%%%%%%%%%%%%%%%%%%%%%%%%%%%%%%%%

\section{Privacy and Ethical Safeguards}
\label{sec:privacy_ethics}

The design of the multi-agent framework incorporates privacy and ethical considerations as core architectural principles, not as afterthoughts. This section outlines the key safeguards built into the agent architecture to protect user privacy and ensure ethical operation in safety-critical conversational contexts.

\subsection{User Anonymization and Privacy by Design}

The framework adheres to the principle of \textbf{Privacy by Design (PbD)} \cite{cavoukian2011privacybydesign}, embedding privacy protections directly into the agent architecture. All user interactions are anonymized through non-identifiable UUIDs, ensuring that conversational data cannot be linked back to real-world identities without explicit administrative access to segregated identity tables.

The Aika Meta-Agent enforces role-based access control at the orchestration layer, ensuring that each specialist agent only accesses the conversation context necessary for its function. For example, the Insights Agent operates exclusively on anonymized message logs and is programmatically restricted from accessing case management data or user profile information.

\subsection{PII Redaction and Data Minimization}

Before any user message is persisted to the conversation history, the backend system performs automated PII redaction to identify and remove common personal identifiers such as email addresses, phone numbers, and proper names. This pre-persistence anonymization ensures that even if an agent retrieves historical context, it cannot inadvertently process sensitive personal information.

The agent framework follows the principle of \textbf{data minimization}: each agent is designed to operate with the minimum necessary information. The Safety Triage Agent, for instance, analyzes only the current message and immediate conversation context for risk detection, without requiring access to longitudinal user profiles or administrative metadata.

\subsection{Ethical Safeguards in Safety-Critical Decisions}

Given the high-stakes nature of mental health triage, the Safety Triage Agent is designed with explicit ethical safeguards:

\begin{itemize}
    \item \textbf{Conservative Risk Classification:} The agent employs a "safety-first" bias, erring on the side of escalation when ambiguous risk indicators are detected. This prevents false negatives in critical situations.
    \item \textbf{Human-in-the-Loop for Critical Cases:} All cases flagged as "critical" by the STA trigger immediate notifications to human counselors. The agent does not make autonomous decisions about crisis intervention; it serves as a detection and escalation mechanism only.
    \item \textbf{Transparency in Agent Responses:} The Support Coach Agent explicitly discloses its non-human nature and limitations in its initial greeting, ensuring users have informed consent about the conversational context.
\end{itemize}

\subsection{Scope Limitation: Focus on Agent Architecture}

It is important to note that while the full UGM-AICare implementation includes comprehensive database schema design, user interface components, and deployment infrastructure, \textbf{the evaluation and validation of these system components is beyond the scope of this thesis}. This research focuses specifically on the design, implementation, and performance evaluation of the multi-agent architecture itself: the BDI-based specialist agents, the Aika orchestration layer, and their collective behavior in safety-critical conversational scenarios.

The thesis evaluates agent performance through controlled scenario-based testing rather than real-world user deployment, as the latter would require extensive ethics approval, medical supervision, and longitudinal user studies that exceed the timeline and scope of bachelor's-level research. The database, UI, and deployment infrastructure serve as implementation context to demonstrate the framework's feasibility, but are not subjects of formal evaluation in this work.

%%%%%%%%%%%%%%%%%%%%%%%%%%%%%%%%%%%%%%%%%%%%%%%%%
%%% SECTION 3.6 - SECURITY AND PRIVACY BY DESIGN %%%
%%%%%%%%%%%%%%%%%%%%%%%%%%%%%%%%%%%%%%%%%%%%%%%%%

\section{Security and Privacy by Design}
\label{sec:security_privacy_design}

In safety-critical health applications, security and privacy must be foundational design principles, not retrofitted features. This section outlines the key security mechanisms built into the multi-agent framework to protect user data, prevent unauthorized access, and ensure system integrity.

\subsection{Authentication and Access Control}

The UGM-AICare implementation uses JWT-based (JSON Web Token) authentication to secure API endpoints and verify user identity. Each user session is associated with a cryptographically signed token that expires after a defined period, preventing unauthorized session hijacking.

At the agent layer, access control is enforced through the Aika Meta-Agent's routing logic. Each specialist agent operates with restricted permissions defined at the application layer, preventing privilege escalation or unauthorized data access. For example, the Insights Agent is restricted to read-only access on anonymized conversation logs and cannot modify case records or user profiles.

\subsection{Data Encryption and Secure Communication}

All data transmission between the frontend application and backend API occurs over HTTPS (TLS 1.3), ensuring end-to-end encryption of user messages during transit. At rest, sensitive data fields (such as case notes created by human counselors) are encrypted using AES-256 encryption, with decryption keys managed through environment variables that are never committed to version control.

The LangGraph orchestration layer ensures that inter-agent communication occurs within the backend application context, preventing message interception or tampering. Agent-to-agent state transitions are validated through typed state schemas, ensuring that malformed or malicious state objects cannot compromise the workflow.

\subsection{Audit Logging and Traceability}

Every agent invocation, risk classification decision, and case escalation is logged with timestamps and contextual metadata. This audit trail serves dual purposes:

\begin{itemize}
    \item \textbf{Clinical Accountability:} Human counselors can review the exact sequence of agent decisions that led to a case escalation, ensuring transparency in the triage process.
    \item \textbf{Security Monitoring:} Unusual patterns (e.g., excessive API calls, repeated failed authentication attempts) can be detected through log analysis, enabling proactive threat detection.
\end{itemize}

The audit logs are stored in a separate, access-controlled database with retention policies aligned with institutional data governance requirements.

\subsection{Threat Model and Mitigation Strategies}

The framework's threat model considers several attack vectors relevant to conversational AI in healthcare contexts:

\begin{itemize}
    \item \textbf{Prompt Injection Attacks:} Malicious users could attempt to manipulate agent behavior through carefully crafted input prompts. Mitigation: All user inputs are sanitized and validated before being passed to LLM inference. The agent system prompts are designed to explicitly ignore instructions embedded in user messages.
    
    \item \textbf{Data Exfiltration:} An attacker could attempt to extract sensitive conversation data through API manipulation. Mitigation: Rate limiting, request validation, and strict role-based access control prevent unauthorized bulk data queries.
    
    \item \textbf{Model Manipulation:} An attacker could attempt to poison the training data or fine-tuning datasets to bias agent behavior. Mitigation: This thesis uses pre-trained foundation models (Gemini) without custom fine-tuning. In production scenarios, model provenance tracking and adversarial robustness testing would be required.
\end{itemize}

%%%%%%%%%%%%%%%%%%%%%%%%%%%%%%%%%%%%%%%%%%%%%%%%%
%%% REMOVED: USER EXPERIENCE (UX) DESIGN %%%
%%% REMOVED: USER INTERFACE (UI) DESIGN %%%
%%%%%%%%%%%%%%%%%%%%%%%%%%%%%%%%%%%%%%%%%%%%%%%%%

% The full UGM-AICare implementation includes comprehensive UX personas,
% user stories, journey maps, and UI wireframes. However, the evaluation
% of user experience and interface design is beyond the scope of this thesis,
% which focuses exclusively on the multi-agent architecture. These design
% artifacts are available in the project repository for reference but are
% not evaluated as part of this research.

%%%%%%%%%%%%%%%%%%%%%%%%%%%%%%%%%%%%%%%%%%%%%%%%%
%%% SECTION 3.7 - ETHICAL CONSIDERATIONS %%%
%%%%%%%%%%%%%%%%%%%%%%%%%%%%%%%%%%%%%%%%%%%%%%%%%

\section{Ethical Considerations and Research Limitations}
\label{sec:ethical_considerations}

The development of an AI-driven framework for mental health support necessitates thorough examination of ethical implications and transparent acknowledgment of research limitations. This section addresses the ethical design choices and defines the boundaries of the study's findings.

\subsection{Informed Consent and Transparency}

The UGM-AICare framework is designed with the principle that users must have clear understanding of the system's capabilities and limitations. The Support Coach Agent explicitly discloses its non-human nature in initial interactions, ensuring users engage with informed consent about the conversational context. This transparency is critical in healthcare applications where users may form therapeutic relationships with AI systems.

\subsection{Human-in-the-Loop for Safety}

The framework is explicitly designed as a tool that assists, but does not replace, human counselors. Every critical risk escalation from the Safety Triage Agent (STA) creates a case that requires mandatory review and action by a qualified human professional. The system automates the detection and reporting, but the final clinical judgment and intervention remain firmly in human hands.

This human oversight is not merely procedural, it addresses the fundamental ethical limitation of LLMs in safety-critical contexts. While models like Gemini 2.5 Flash demonstrate strong performance in text understanding, they can still misinterpret nuanced emotional states or linguistic cues. The human-in-the-loop design ensures that no automated risk classification leads directly to intervention without expert clinical validation.

\subsection{AI as Support Tool, Not Replacement for Therapy}

It is ethically imperative to clearly define the system's role. The UGM-AICare framework is designed as a sub-clinical, supportive tool and a bridge to professional care, not as a substitute for licensed therapy. The Support Coach Agent is programmed to explicitly state this boundary and to encourage users to seek professional help for serious or persistent issues, facilitated through the Service Desk Agent's appointment booking functionality.

\subsection{Data Privacy and Purpose Limitation}

As detailed in Section~\ref{sec:privacy_ethics}, the system architecture protects user anonymity through UUID-based identifiers and pre-persistence PII redaction. The principle of \textbf{purpose limitation} is strictly enforced: data collected is used exclusively for providing in-the-moment support and generating aggregated, anonymized insights to improve university services. It is never used for academic assessment, disciplinary action, or any purpose outside its stated mission.

\subsection{Research Limitations}

This study, as a work of Design Science Research focused on artifact creation and evaluation, is subject to several important limitations:

\begin{itemize}
    \item \textbf{Methodological Limitation - Scenario-Based Evaluation:} The evaluation of this framework (detailed in Chapter 4) is based on controlled scenario testing with synthetic conversational data, not real-world user deployment. This thesis validates the \textit{technical feasibility} of the agentic workflows and the \textit{architectural integrity} of the multi-agent design. It does \textbf{not} measure long-term psychological outcomes or therapeutic efficacy on actual students. Such claims would require extensive ethics approval, medical supervision, and longitudinal clinical trials that exceed the scope of bachelor's-level research.
    
    \item \textbf{Technical Limitation - Inherent Risks of LLMs:} The framework relies on Google Gemini 2.5 Flash API. Like all LLMs, it is subject to inherent limitations including potential biases from training data and the possibility of generating factually incorrect or nonsensical responses ("hallucinations"). While the system's use of structured tools, typed state schemas, and explicit agent prompts is designed to mitigate these risks, they cannot be eliminated entirely.
    
    \item \textbf{Data Limitation - Simulated Evaluation Data:} The evaluation is conducted using synthetically generated maternal health scenarios and simulated conversational patterns, not real user data. This is necessary to protect privacy during the development phase and to enable controlled testing without requiring human subjects approval. However, it means that agent performance has not been validated on the specific linguistic diversity, cultural contexts, and edge cases of a live Indonesian student population.
    
    \item \textbf{Scope Limitation - Agent Architecture Focus:} This thesis evaluates the multi-agent architecture—the BDI-based specialist agents, Aika orchestration layer, and their collective behavior in safety-critical conversations. The full UGM-AICare implementation includes database design, user interface components, blockchain token systems, and deployment infrastructure, but \textbf{these system components are not subjects of formal evaluation in this work}. They serve as implementation context to demonstrate feasibility, but their performance characteristics, user experience quality, and production readiness are not validated.
\end{itemize}

\subsection{Ethical Safeguards and Human Oversight}

Technology alone is insufficient to guarantee ethical operation. Therefore, the system is designed with procedural safeguards that ensure human oversight for all critical functions.

\begin{itemize}
    \item \textbf{Human-in-the-Loop for Safety:} The framework is explicitly designed to be a tool that assists, but does not replace, human counselors. Every critical risk escalation from the Safety Triage Agent (STA) creates a case that requires mandatory review and action by a qualified human professional. The system automates the detection and reporting, but the final clinical judgment and intervention remain firmly in human hands.
    \item \textbf{Purpose Limitation:} The data collected through the chat interface is used for the sole and explicit purposes of providing in-the-moment support, managing clinical escalations, and generating anonymized, aggregated statistics for improving the university's support services. The data is not used for any other purpose, such as academic assessment or disciplinary action. This principle is enforced through the technical separation of the anonymized analytical data from any administrative records.
\end{itemize}

These ethical safeguards are not merely compliance measures. They represent the framework's foundational commitment to responsible AI deployment in safety-critical healthcare contexts.