\chapter{System Design and Architecture}

% Bab ini menjelaskan metode atau cara yang digunakan dalam penelitian ini untuk 
% mencapai maksud dan tujuan seperti yang tertulis dalam sub-bab 1.3 [jika diinginkan, kalian dapat menuliskan Kembali tujuan penelitian yang ingin dicapai di sini].

\section{Research Methodology: Design Science Research (DSR)}

% Purpose: To formally establish the academic methodology underpinning your constructive research, directly addressing the need for a research method.

% Elaboration Points:

% Provide a formal definition of Design Science Research, citing key literature (e.g., Hevner et al., 2004). Explain that DSR's goal is to create innovative artifacts that solve real-world problems.

% Outline the stages of DSR and map them to your thesis structure:

% Problem Identification & Motivation (Addressed in Chapter 1).

% Definition of Objectives for a Solution (Addressed in Chapter 1).

% Design & Development (The focus of this chapter, Chapter 3, and the next, Chapter 4).

% Demonstration (The focus of the testing in Chapter 4).

% Evaluation & Communication (Addressed in Chapter 4 and Chapter 5, and through this thesis itself).

% Include a flowchart diagram illustrating your research process from start to finish.


\section{System Overview and Conceptual Design}

Purpose: To provide a high-level, 10,000-foot view of the system's purpose and its main components, making it understandable before diving into technical specifics.

Elaboration Points:

Present a concise paragraph describing the conceptual model: "The proposed framework is an ecosystem of three collaborative, intelligent agents that work in concert to transform an institution's mental health support from a reactive to a proactive model..."

Include a high-level Context Diagram showing the main entities (Students, University Staff/Counselors) and systems (UGM-AICare User App, Agentic AI Backend, Admin Dashboard) and how they interact.

\section{Functional Architecture: The Agentic Core}
Purpose: To detail the "what" of the system. This section explains the specific roles and functions of each agent. This is the heart of your functional design.

Elaboration Points: Create a subsection for each of the three agents. For each, define its:

\subsection{The Analytics Agent}

Goal: To autonomously identify mental health trends from anonymized user data.

Perception (Inputs): Anonymized conversation logs from the PostgreSQL database.

Processing Logic: Describe the NLP tasks it performs: topic modeling, sentiment analysis, and summarization.

Action (Outputs): A structured weekly report (in JSON format) containing key insights (e.g., top 5 trending stress topics, overall sentiment score).

\subsection{The Intervention Agent}

Goal: To automate targeted, proactive outreach campaigns.

Perception (Inputs): The structured report from the Analytics Agent and a set of predefined "campaign rules."

Processing Logic: A rule-based engine that maps insights to actions (e.g., IF 'exam stress' > threshold THEN trigger 'time-management-workshop' campaign).

Action (Outputs): A signal sent to the orchestration layer (n8n) with target audience segment and message content.

\subsection{The Triage Agent}

Goal: To efficiently route a student to the most appropriate level of support in real-time.

Perception (Inputs): A user's live conversation with the chatbot.

Processing Logic: A real-time text classification model to determine the conversation's severity level (e.g., Level 1: Casual, Level 2: Moderate, Level 3: Red Flag).

Action (Outputs): A structured recommendation (e.g., suggest a self-help module, suggest booking a counselor, provide an emergency hotline).

Include a detailed Data Flow Diagram (DFD) showing how data moves between these three agents and the database.

\section{Technical Architecture: The Hybrid System}
Purpose: To detail the "how" of the system—the engineering blueprint, including justification for key technology choices.

Elaboration Points:

\subsection{Overall System Architecture Diagram}
A detailed diagram showing the interplay between all technologies: Next.js (Admin Dashboard), FastAPI (Backend), PostgreSQL, LangChain (as a library), and n8n (as a separate, orchestrated service). Show the communication protocols (e.g., REST API, direct DB connection).

\subsection{The "Brain": FastAPI + LangChain Service}
Justify the choice of FastAPI (for performance, async support) and LangChain (for LLM orchestration). Detail the API design, defining key endpoints (e.g., /api/agents/generate-report) and their request/response schemas.

\subsection{The "Nervous System": n8n Workflows} 
Justify the choice of n8n for robust workflow automation. Provide screenshots or diagrams of the primary n8n workflows (e.g., the "Weekly Report Generation" workflow triggered by a Cron job).

\section{Database Design}
Purpose: To define the data persistence layer of the system.

Elaboration Points:

Present a clean Entity-Relationship Diagram (ERD).

% Add table of key columns and data types for the most important table
\begin{table}[h]
    \centering
    \caption{Key Columns and Data Types for \texttt{conversation\_logs} Table}
    \label{tab:conversation-logs}
    \begin{tabular}{|l|l|l|}
        \hline
        \textbf{Column Name} & \textbf{Data Type} & \textbf{Description} \\
        \hline
        id & SERIAL PRIMARY KEY & Unique identifier \\
        user\_id & UUID & Reference to user (anonymized) \\
        timestamp & TIMESTAMP WITH TIME ZONE & Time of message \\
        message & TEXT & User or agent message content \\
        sender & VARCHAR(16) & 'user' or 'agent' \\
        sentiment\_score & FLOAT & Sentiment analysis result \\
        topic & VARCHAR(64) & NLP-inferred topic label \\
        \hline
    \end{tabular}
\end{table}

\section{User Interface (UI) Design}
Purpose: To show the design of the human interface for the system's administrative users.

Elaboration Points:

Define the primary user persona for the dashboard (e.g., "Dr. Astuti, Head of Counseling Services").

Present wireframes or high-fidelity mockups for the key screens of the Admin Dashboard (e.g., the main analytics view, the report history page).

\section{Security and Privacy by Design}
Purpose: To demonstrate that critical security and privacy considerations are integral to the architecture.

Elaboration Points:

Detail the Data Anonymization Pipeline: How is Personally Identifiable Information (PII) identified and redacted from chat logs before they are stored for analysis?

Describe the Role-Based Access Control (RBAC) mechanism for the admin dashboard.

Mention standard security practices like data encryption in transit (TLS) and at rest.

\section{Alur Tugas Akhir}

Menguraikan prosedur yang akan digunakan dan jadwal atau alur penyelesaian setiap 
tahap. Alur penelian ini dapat disajikan dalam bentuk diagram. Diagram dapat disusun dengan aturan yang baik semisal menggunakan \textit{flowchart}. Aturan dan tutorial pembuatan \textit{flowchart} dapat dilihat di \textcolor{blue}{http://ugm.id/flowcharttutorial}. Setelah menggambarkannya, penulis wajib menjelaskan langkah-langkah setiap alur tugas akhir dalam sub bab tersendiri sesuai dengan kebutuhan.

\section{Etika, Masalah, dan Keterbatasan Penelitian (Opsional))}

Bagian ini membahas pertimbangan etis penelitian dan [potensi] masalah serta
keterbatasannya. Jika menyangkut penelitian dengan makhluk hidup, maka dibutuhkan adanya \textit{ethical clearance}, di bagian ini hal itu akan dibahas. Demikian juga tentang keterbatasan ataupun masalah yang akan timbul.
