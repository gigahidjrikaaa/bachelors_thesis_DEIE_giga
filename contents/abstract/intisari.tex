% Intisari
\noindent
Layanan kesejahteraan mahasiswa di perguruan tinggi masih banyak bersifat reaktif dan sering terlambat menjangkau mereka yang membutuhkan. Skripsi ini merancang dan mempraktikkan sebuah kerangka \textit{agentic AI} yang berorientasi keselamatan untuk mendukung layanan yang lebih proaktif dan terukur dengan tetap melibatkan manusia. Artifak inti, \textit{Safety Agent Suite}, terdiri dari empat agen yang saling melengkapi: (i) \textbf{Safety Triage Agent} untuk penyaringan risiko dan eskalasi, (ii) \textbf{Support Coach Agent} yang memberikan intervensi singkat berlandaskan CBT, (iii) \textbf{Service Desk Agent} untuk tindak lanjut operasional, dan (iv) \textbf{Insights Agent} untuk analitik agregat yang menjaga privasi guna perbaikan layanan.

\noindent
Kami membangun prototipe fungsional dan melakukan evaluasi berbasis skenario yang menitikberatkan pada kinerja agen dan aspek keselamatan: sensitivitas/spesifisitas triase pada prompt krisis sintetis serta waktu ke eskalasi; keandalan orkestrasi melalui tingkat keberhasilan pemanggilan fungsi dan pola \textit{retry}; latensi ujung-ke-ujung; ketahanan terhadap \textit{prompt injection}; serta kualitas coaching yang dinilai buta menggunakan rubrik kepatuhan CBT dan kewajaran respons. Untuk \textit{Insights Agent}, kami hanya melaporkan pemeriksaan agregat yang sederhana (misalnya kestabilan jumlah topik di atas ambang privasi) tanpa klaim pada level individu. Hasil menunjukkan orkestrasi agen yang layak dengan latensi terkendali dan moda kegagalan yang dapat dipantau di bawah pengawasan manusia. Kami juga membahas pertimbangan etis, rancangan privasi untuk analitik agregat, keterbatasan, dan kebutuhan studi lanjutan.

\noindent\textbf{Kata kunci}: AI agentik; sistem multiagen; triase keselamatan; coaching CBT; human-in-the-loop; analitik agregat; LangGraph

