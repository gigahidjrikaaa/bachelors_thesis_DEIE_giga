% Intisari
\noindent
Layanan kesejahteraan mahasiswa di perguruan tinggi masih banyak bersifat reaktif dan sering terlambat menjangkau mereka yang membutuhkan. Skripsi ini merancang dan mengevaluasi sebuah kerangka multi-agen AI yang berorientasi keselamatan untuk mendukung layanan yang lebih proaktif dan terukur dengan tetap melibatkan manusia. Artifak inti, \textit{Safety Agent Suite}, terdiri dari lima komponen: (i) \textbf{Aika Meta‑Agent} yang menyediakan orkestrasi terpadu dengan penyaringan risiko langsung Tingkat 1 yang terintegrasi, (ii) \textbf{Safety Triage Agent} untuk analisis risiko menyeluruh Tingkat 2 di tingkat percakapan, (iii) \textbf{Therapeutic Coach Agent} yang memberikan intervensi terapeutik singkat berbasis CBT, (iv) \textbf{Case Management Agent} untuk koordinasi kasus klinis dan tindak lanjut operasional, dan (v) \textbf{Insights Agent} untuk analitik agregat yang menjaga privasi. Arsitektur pemantauan risiko dua tingkat (penyaringan per-pesan Aika + analisis percakapan STA) mencapai pengurangan biaya API LLM sebesar 45-60\% sambil meningkatkan akurasi klinis melalui penilaian kontekstual holistik. Sistem multi-agen dibangun dengan LangGraph dan mencakup perlindungan untuk penggunaan alat, redaksi, dan kemampuan audit.

\noindent
Kami membangun prototipe fungsional dalam platform UGM-AICare dan melakukan evaluasi berbasis skenario yang menitikberatkan secara eksklusif pada kinerja arsitektur agen: sensitivitas/spesifisitas triase pada skenario krisis sintetis; keandalan orkestrasi melalui tingkat keberhasilan pemanggilan fungsi dan transisi state; latensi ujung-ke-ujung; ketahanan terhadap \textit{prompt injection}; serta kualitas coaching yang dinilai buta menggunakan rubrik kepatuhan CBT. \textbf{Skripsi ini berfokus secara spesifik pada desain dan evaluasi kerangka multi-agen itu sendiri}—agen spesialis berbasis BDI, lapisan orkestrasi Aika, dan perilaku kolektif mereka dalam konteks percakapan kritis keselamatan. Desain basis data, komponen antarmuka pengguna, dan infrastruktur deployment didokumentasikan sebagai konteks implementasi namun bukan subjek evaluasi formal. Hasil menunjukkan kelayakan orkestrasi agen yang andal dengan latensi terkendali dan moda kegagalan yang dapat dipantau di bawah pengawasan manusia. Kami membahas pertimbangan etis, prinsip \textit{privacy by design}, keterbatasan penelitian, dan kebutuhan studi klinis lapangan di masa depan dengan pengguna riil.

\noindent\textbf{Kata kunci}: Sistem Multi-Agen; Arsitektur BDI; Orkestrasi Agen; Triase Keselamatan; LangGraph; Human-in-the-Loop; Kesejahteraan Mahasiswa; Evaluasi Berbasis Skenario

