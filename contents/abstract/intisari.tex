% Intisari
\noindent
Layanan kesejahteraan mahasiswa di perguruan tinggi masih banyak bersifat reaktif dan sering terlambat menjangkau mereka yang membutuhkan. Skripsi ini merancang dan mengevaluasi sebuah kerangka multi-agen AI yang berorientasi keselamatan untuk mendukung layanan yang lebih proaktif dan terukur dengan tetap melibatkan manusia. Artifak inti, \textit{Safety Agent Suite}, terdiri dari lima komponen: (i) \textbf{Safety Triage Agent} untuk penyaringan risiko dan eskalasi, (ii) \textbf{Support Coach Agent} yang memberikan intervensi singkat berlandaskan CBT, (iii) \textbf{Service Desk Agent} untuk tindak lanjut operasional, (iv) \textbf{Insights Agent} untuk analitik agregat yang menjaga privasi guna perbaikan layanan, dan (v) \textbf{Aika Meta‑Agent} yang mengkoordinasikan keempat agen spesialis tersebut dengan orkestrasi berbasis peran untuk memastikan interaksi yang koheren dan mengutamakan keselamatan.

\noindent
Kami membangun prototipe fungsional dalam platform UGM-AICare dan melakukan evaluasi berbasis skenario yang menitikberatkan secara eksklusif pada kinerja arsitektur agen: sensitivitas/spesifisitas triase pada skenario krisis sintetis; keandalan orkestrasi melalui tingkat keberhasilan pemanggilan fungsi dan transisi state; latensi ujung-ke-ujung; ketahanan terhadap \textit{prompt injection}; serta kualitas coaching yang dinilai buta menggunakan rubrik kepatuhan CBT. \textbf{Skripsi ini berfokus secara spesifik pada desain dan evaluasi kerangka multi-agen itu sendiri}—agen spesialis berbasis BDI, lapisan orkestrasi Aika, dan perilaku kolektif mereka dalam konteks percakapan kritis keselamatan. Desain basis data, komponen antarmuka pengguna, dan infrastruktur deployment didokumentasikan sebagai konteks implementasi namun bukan subjek evaluasi formal. Hasil menunjukkan kelayakan orkestrasi agen yang andal dengan latensi terkendali dan moda kegagalan yang dapat dipantau di bawah pengawasan manusia. Kami membahas pertimbangan etis, prinsip \textit{privacy by design}, keterbatasan penelitian, dan kebutuhan studi klinis lapangan di masa depan dengan pengguna riil.

\noindent\textbf{Kata kunci}: Sistem Multi-Agen; Arsitektur BDI; Orkestrasi Agen; Triase Keselamatan; LangGraph; Human-in-the-Loop; Kesejahteraan Mahasiswa; Evaluasi Berbasis Skenario

