% Intisari
\noindent
Institusi Pendidikan Tinggi menghadapi peningkatan permintaan dukungan kesejahteraan mahasiswa namun masih bergantung pada kerangka kerja konseling reaktif yang seringkali gagal menjangkau mahasiswa sebelum krisis memuncak. Skripsi ini mengusulkan dan mengevaluasi kerangka kerja AI agentic proaktif yang dirancang untuk menjembatani kesenjangan \textit{insight-to-action} dengan memungkinkan intervensi dini dan manajemen sumber daya berbasis data. Kami memperkenalkan \textit{Safety Agent Suite}, arsitektur multi-agen terpisah yang mendistribusikan tanggung jawab klinis dan operasional kepada agen khusus di bawah pengawasan manusia. Sistem ini mencakup: (i) \textbf{Aika}, orkestrator Meta-Agent yang menyediakan antarmuka pengguna terpadu dan melakukan penyaringan risiko Tingkat 1 segera; (ii) \textbf{Safety Triage Agent (STA)} untuk analisis risiko percakapan Tingkat 2 yang komprehensif; (iii) \textbf{Therapeutic Coach Agent (TCA)} yang memberikan intervensi mikro terapeutik berbasis Cognitive Behavioral Therapy (CBT); (iv) \textbf{Case Management Agent (CMA)} untuk koordinasi operasional; dan (v) \textbf{Insights Agent} untuk analitik manajemen sumber daya yang menjaga privasi. Untuk menyeimbangkan responsivitas dengan kedalaman analisis, kami menggunakan arsitektur pemantauan risiko dua tingkat yang menggabungkan penyaringan segera dengan analisis percakapan mendalam untuk memungkinkan intervensi dini. Sistem multi-agen dibangun dengan LangGraph dan mencakup perlindungan untuk penggunaan alat, redaksi, dan kemampuan audit.

\noindent
Kami membangun prototipe fungsional dalam platform UGM-AICare dan melakukan evaluasi berbasis skenario yang menitikberatkan secara eksklusif pada kinerja arsitektur agen: sensitivitas dan \textit{False Negative Rate} (FNR) triase pada skenario krisis sintetis; keandalan orkestrasi melalui tingkat keberhasilan pemanggilan fungsi dan transisi state; latensi ujung-ke-ujung; verifikasi kepatuhan privasi; serta kualitas coaching melalui rubrik kepatuhan CBT dengan penilaian ahli dan validasi LLM. \textbf{Skripsi ini berfokus secara spesifik pada desain dan evaluasi kerangka multi-agen itu sendiri}—agen spesialis berbasis BDI, lapisan orkestrasi Aika, dan perilaku kolektif mereka dalam konteks percakapan kritis keselamatan. Desain basis data, komponen antarmuka pengguna, dan infrastruktur deployment didokumentasikan sebagai konteks implementasi namun bukan subjek evaluasi formal. Hasil menunjukkan kelayakan teknis keselamatan proaktif, orkestrasi agen yang andal, dan dukungan yang menjaga privasi, mengonfirmasi kapasitas sistem untuk menutup kesenjangan \textit{insight-to-action} di bawah pengawasan manusia. Kami membahas pertimbangan etis, prinsip \textit{privacy by design}, keterbatasan penelitian, dan kebutuhan studi klinis lapangan di masa depan dengan pengguna riil.

\noindent\textbf{Kata kunci}: Sistem Multi-Agen; Arsitektur BDI; Orkestrasi Agen; Triase Keselamatan; LangGraph; Human-in-the-Loop; Kesejahteraan Mahasiswa; Evaluasi Berbasis Skenario

