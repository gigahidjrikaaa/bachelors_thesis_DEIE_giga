\chapter{Introduction}

\section{Background}
\label{sec:background}


Discuss the "organizational challenge": HEIs struggle with scalable, proactive, and efficient mental health support.

Describe the current model as "reactive," waiting for students to seek help, which leads to overwhelmed human resources.

Introduce "Digital Transformation" in the context of higher education.

Introduce "Agentic AI" as a paradigm shift from passive tools to autonomous systems that act to achieve goals.

%-------------------------------------------------

\section{Problem Statement}
\label{sec:problem_statement}

\noindent Berdasarkan latar belakang yang telah diuraikan, penelitian ini difokuskan untuk menjawab pertanyaan-pertanyaan berikut:

\begin{enumerate}
    \item How to design an agentic AI framework to automate trend analysis, intervention campaigns, and initial triage for university mental health services?
    \item How can a robust and modular technical architecture (using FastAPI, LangChain, n8n) implement this framework?
    \item What is the potential impact of this framework on operational efficiency and institutional decision-making processes?
\end{enumerate}

%-------------------------------------------------	

\section{Objectives}
\label{sec:objectives}

\noindent Sejalan dengan rumusan masalah di atas, tujuan dari penelitian ini adalah sebagai berikut:

\begin{enumerate}
    \item To design the conceptual and technical framework for the agentic AI system.
    \item To implement a functional proof-of-concept prototype.
    \item To evaluate the prototype's capabilities against predefined functional scenarios.
\end{enumerate}

%-------------------------------------------------	

\section{Scope and Limitations}
\label{sec:scope_and_limitations}

\begin{enumerate}
    \item The research is focused on the design, prototype implementation, and scenario-based validation of the framework.
    \item It is not a full-scale deployment across the university.
    \item It does not measure the long-term impact on student psychological outcomes, but rather the system's functional capabilities and potential organizational impact.
\end{enumerate}

% \noindent Agar penelitian ini lebih terarah dan mendalam, ruang lingkupnya dibatasi sebagai berikut:

% \begin{enumerate}
%     \item \textbf{Objek Penelitian:} Fokus pada desain, pengembangan prototipe, dan evaluasi platform terintegrasi yang menggabungkan \textit{hybrid conversational AI}, \textit{gamification}, dan \textit{blockchain} untuk mahasiswa. Studi kasus dapat difokuskan pada mahasiswa UGM atau lingkungan universitas serupa di Indonesia.
%     \item \textbf{Teknologi:} Implementasi \textit{hybrid conversational AI} akan menggunakan kombinasi \textit{Pre-trained LLM} Gemini dengan RAG berbasis dokumen kesehatan mental UGM. Implementasi \textit{blockchain} akan menggunakan EDUChain (L2 Ethereum) untuk fitur \textit{achievement badges minting}. Mekanisme \textit{gamification} akan mencakup \textit{badges}, \textit{points}, dan sistem \textit{levelling}.
%     \item \textbf{Metode Penelitian:} Menggunakan pendekatan \textit{Design Science Research} (DSR) yang meliputi tahap desain artifak (platform), pengembangan prototipe, dan evaluasi. Evaluasi akan menggunakan metode campuran (\textit{mixed methods}), menggabungkan analisis data log penggunaan platform, survei, dan/atau wawancara dengan kelompok pengguna terbatas.
%     \item \textbf{Waktu dan Tempat Penelitian:} Penelitian dilaksanakan dalam periode [Sebutkan periode, misal: Semester Gasal 2024/2025 hingga Semester Genap 2024/2025] di lingkungan [Sebutkan, misal: Laboratorium Departemen Teknik Elektro dan Teknologi Informasi UGM dan melibatkan partisipan mahasiswa UGM secara daring].
%     \item \textbf{Populasi dan Sampel:} Populasi adalah mahasiswa aktif [Sebutkan jenjang, misal: S1] di [Sebutkan Fakultas/Universitas]. Sampel untuk evaluasi adalah sejumlah [Sebutkan jumlah, misal: 30-50] mahasiswa yang dipilih melalui [Sebutkan metode sampling, misal: purposive sampling atau voluntary sampling].
%     \item \textbf{Variabel dan Indikator:} Variabel independen adalah penggunaan platform dan fitur-fiturnya. Variabel dependen meliputi metrik \textit{user engagement} (misal: frekuensi login, durasi sesi, jumlah interaksi AI, progres gamifikasi) dan indikator kesejahteraan (misal: skor skala PSS - Perceived Stress Scale, skor skala koneksi sosial, feedback kualitatif).
%     \item \textbf{Keterbatasan Penelitian:} Penelitian ini tidak mencakup implementasi skala penuh di seluruh universitas. Evaluasi aspek kesejahteraan bersifat subjektif berdasarkan persepsi pengguna. Aspek keamanan \textit{blockchain} dievaluasi pada level prototipe dan tidak mencakup audit keamanan ekstensif. Model AI mungkin memiliki keterbatasan dalam menangani semua topik atau kondisi kesehatan mental yang kompleks dan tidak menggantikan konseling profesional.
% \end{enumerate}

%-------------------------------------------------

\section{Contributions}
\label{sec:contributions}

\begin{enumerate}
    \item Academic: A novel framework for applying agentic AI in an institutional (higher education) context.
    \item Practical: A blueprint for UGM to develop a more proactive, data-driven, and efficient mental health support system.
\end{enumerate}

% Penelitian ini diharapkan dapat memberikan manfaat signifikan bagi berbagai pihak:

% \begin{itemize}
%     \item \textbf{Bagi Mahasiswa:} Menyediakan akses terhadap platform digital inovatif yang dapat mendukung kebutuhan informasi, interaksi sosial, motivasi, dan akses ke sumber daya kesejahteraan secara terintegrasi, aman, dan menarik.
%     \item \textbf{Bagi Institusi Pendidikan (UGM dan lainnya):} Memberikan model dan bukti konsep (\textit{proof-of-concept}) mengenai pemanfaatan teknologi AI, \textit{gamification}, dan \textit{blockchain} untuk meningkatkan layanan dukungan mahasiswa, engagement, dan strategi peningkatan kesejahteraan di era digital.
%     \item \textbf{Bagi Komunitas Akademik dan Riset:} Menambah khazanah ilmu pengetahuan di bidang Teknik Informasi, khususnya terkait interaksi manusia-komputer (HCI), AI dalam pendidikan (\textit{AI in Education}), aplikasi \textit{blockchain} non-finansial, dan desain sistem terintegrasi untuk kesejahteraan digital (\textit{digital well-being}). Menjadi dasar bagi penelitian selanjutnya di area ini.
%     \item \textbf{Bagi Pengembang Teknologi:} Memberikan wawasan praktis mengenai tantangan dan strategi dalam mengintegrasikan tiga teknologi kompleks (\textit{conversational AI, gamification, blockchain}) dalam satu platform yang berfokus pada pengguna akhir.
% \end{itemize}

%-------------------------------------------------

\section{Thesis Outline}
\label{sec:thesis_outline}

Penyusunan laporan skripsi ini akan mengikuti sistematika sebagai berikut:

\textbf{Bab I Pendahuluan,} menguraikan latar belakang masalah, justifikasi pentingnya penelitian, rumusan masalah yang akan dijawab, tujuan yang ingin dicapai, batasan ruang lingkup penelitian, manfaat yang diharapkan, serta sistematika penulisan laporan skripsi.

\textbf{Bab II Tinjauan Pustaka dan Dasar Teori,} menyajikan kajian literatur terhadap penelitian-penelitian terdahulu yang relevan di bidang \textit{conversational AI}, \textit{gamification} dalam pendidikan/kesejahteraan, aplikasi \textit{blockchain} terkait, serta \textit{user engagement} dan kesejahteraan mahasiswa. Bab ini juga memaparkan landasan teori yang mendasari konsep dan teknologi yang digunakan dalam penelitian.

\textbf{Bab III Metodologi Penelitian,} menjelaskan secara rinci pendekatan penelitian yang digunakan (\textit{Design Science Research}), tahapan perancangan arsitektur platform, spesifikasi teknis pengembangan prototipe (termasuk pemilihan teknologi AI, \textit{blockchain}, dan \textit{gamification}), metode pengumpulan data, serta desain dan instrumen evaluasi platform.

\textbf{Bab IV Hasil dan Pembahasan,} menyajikan hasil dari implementasi prototipe platform, data yang terkumpul selama tahap evaluasi (baik kuantitatif maupun kualitatif), serta analisis mendalam terhadap hasil tersebut dikaitkan dengan tujuan penelitian dan pertanyaan penelitian. Pembahasan juga mencakup analisis terhadap tantangan teknis dan usabilitas yang ditemukan.

\textbf{Bab V Kesimpulan dan Saran,} merangkum temuan-temuan utama penelitian, menyajikan kesimpulan yang menjawab rumusan masalah dan tujuan penelitian, serta memberikan saran praktis bagi pengembangan platform lebih lanjut dan rekomendasi untuk penelitian di masa mendatang.

% Akhir dari Chapter 1