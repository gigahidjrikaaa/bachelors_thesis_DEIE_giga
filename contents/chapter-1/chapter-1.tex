\chapter{Introduction}
\label{chap:introduction}

\section{Background}
\label{sec:background}


Higher Education Institutions (HEIs) are facing a critical and growing challenge in supporting student well-being \cite{hill2024studentwellbeing, duraku2024overcoming}. A landmark report highlights the escalating prevalence of mental health and substance use issues among student populations, urging institutions to adopt a more comprehensive support model \cite{scherer2021mentalhealth}. This crisis not only jeopardizes students' academic success and personal development but also places an immense, unsustainable strain on the institutions tasked with supporting them. Recent global surveys indicate that nearly 42\% of university students meet the criteria for at least one mental health disorder, while the average counselor-to-student ratio in higher education remains around 1:1,500, well above recommended levels for effective service delivery \cite{lipson2022healthy, gallagher2023counselor}.

The traditional support model, centered around on-campus counseling services, is fundamentally \textbf{reactive}. It relies on students to self-identify their distress and navigate the process of seeking help. This paradigm faces significant operational challenges, including insufficient staffing, long waiting lists, and an inability to provide immediate, 24/7 support, which ultimately limits access for a large portion of the student body \cite{baik2019universities}. Consequently, a critical gap persists between the need for mental health services and their actual provision, leaving many students without timely support \cite{outay2024multiagent}.

To bridge this gap, a paradigm shift from a reactive to a \textbf{proactive} support model is imperative \cite{outay2024multiagent}. The engine for this evolution is \textbf{Digital Transformation}, a process that leverages technology to fundamentally reshape organizational processes and enhance value delivery within HEIs \cite{omirali2025digitaltrust}. Within this context, Artificial Intelligence (AI) has emerged as a key enabling technology, with systematic reviews confirming its significant potential to analyze complex data, automate processes, and deliver personalized interventions at scale within the higher education landscape \cite{pati2025agentic, karunanayake2025nextgen}.

However, most existing AI applications in university mental health remain limited to passive chatbots or predictive dashboards that, while insightful, depend on human operators to interpret and act upon their outputs, a limitation widely recognized as the \textit{insight-to-action gap} \cite{jorno2018actionableinsight, susnjak2022dashboard}. This thesis argues that overcoming this gap requires a more autonomous paradigm, in which AI systems do not merely predict or inform but can proactively decide and act.

This research therefore moves beyond conventional AI applications by proposing the use of \textbf{Agentic AI}. An intelligent agent is an autonomous system capable of perception, decision-making, and proactive action to achieve specific goals \cite{saleem2025multiagent, wooldridge2009introductionmas}, representing a new frontier in educational technology \cite{salutari2024mas}. We propose that a framework built upon a system of collaborative intelligent agents, a \textbf{Multi-Agent System (MAS)}, can create a truly transformative ecosystem. Such a system would not only serve as a support tool for students but, more importantly, would function as a strategic asset for the institution, enabling data-driven decision-making, automating operational workflows, and facilitating a proactive stance on student well-being. 

This framework is prototyped within the \textbf{UGM-AICare Project}, a collaborative university research initiative focused on developing AI-driven mental health and well-being tools for the Universitas Gadjah Mada (UGM) community. The project serves as the practical testbed for validating the proposed agentic system in a real institutional context.


%-------------------------------------------------

\section{Problem Formulation}
\label{sec:problem_formulation}

The inefficiency and reactive nature of current university mental health support systems present a complex problem. To move towards a proactive and scalable model, this research addresses the following core challenges:

\begin{enumerate}
    \item The primary challenge is the \textbf{design of a cohesive, safety-oriented agentic AI framework} capable of automating key institutional processes. This requires a shift from a monolithic chatbot to a multi-agent system where specialized agents handle distinct tasks, including real-time crisis detection, personalized coaching, clinical case management, and privacy-preserving analytics.

    \item The technical challenge of \textbf{orchestrating a heterogeneous multi-agent system} in a robust, scalable, and secure cloud-native architecture. This involves managing stateful, long-running interactions and ensuring reliable communication between agents powered by external, non-deterministic LLMs.

    \item The methodological challenge of \textbf{validating the framework's functional capabilities and potential for impact in the absence of a full-scale clinical trial}. This requires developing meaningful, scenario-based testing protocols that can effectively demonstrate the agentic workflows and their advantages over static systems.
\end{enumerate}

To address these challenges, this thesis proposes and details the \textbf{Safety Agent Suite}, a framework comprised of four specialized, collaborative intelligent agents: a \textbf{Safety Triage Agent (STA)}, a \textbf{Support Coach Agent (SCA)}, a \textbf{Service Desk Agent (SDA)}, and an \textbf{Insights Agent (IA)}.

%-------------------------------------------------	

\section{Objectives}
\label{sec:objectives}

The primary objectives of this thesis are:
\begin{enumerate}
    \item To design an agentic AI framework, grounded in the BDI model of rational agency, that systematically bridges the 'insight-to-action' gap in institutional mental health support.
    \item To implement a functional proof-of-concept prototype, the 'Safety Agent Suite,' demonstrating the orchestration of specialized agents (triage, coaching, service desk, insights) using LangGraph.
    \item To evaluate the prototype's core agentic workflows through scenario-based testing, validating its capacity for proactive intervention and automated administrative action.
\end{enumerate}

%-------------------------------------------------

\section{Research Questions}
\label{sec:research_questions}

To keep the scope concrete and measurable, this thesis addresses the following research questions (RQs):

\begin{enumerate}
    \item \textbf{RQ1 (Safety):} Can the Safety Triage Agent detect crisis intent with high sensitivity while keeping false negatives minimal, and escalate within an acceptable time budget?
    \item \textbf{RQ2 (Reliability):} How reliably does the multi-agent orchestration execute tool calls and complete end-to-end workflows under schema validation, retries, and timeouts?
    \item \textbf{RQ3 (Quality):} Do Support Coach responses meet a basic standard of CBT-informed guidance and appropriateness as rated by human evaluators on a small, blinded set?
    \item \textbf{RQ4 (Insights, minimal):} Can the Insights Agent produce stable, aggregate-only summaries under privacy thresholds without exposing individual data?
\end{enumerate}

These questions directly inform the evaluation in Chapter IV through scenario-based tests and simple, transparent metrics (e.g., sensitivity/specificity, tool-call success rate, latency percentiles, rubric scores), with human oversight preserved for safety-critical cases.

%-------------------------------------------------	

\section{Scope and Limitations}
\label{sec:scope_and_limitations}

To ensure the feasibility and focus of this research, the following boundaries are established:

\begin{enumerate}
    \item This research is focused on the \textbf{design and prototype implementation of the agentic AI framework} (the Safety Agent Suite). Supporting content such as CBT-informed prompts is limited to curated scripts; broader engagement mechanics (e.g., badges, external loyalty systems) are out of scope.

    \item The evaluation of the framework is based on \textbf{functional, scenario-based testing} of the prototype's agentic workflows. It does not measure the long-term psychological impact on students or the real-world operational savings for the institution.

    \item The data utilized for testing the analytics agent will consist of \textbf{anonymized, pre-existing chat logs or simulated data} to ensure user privacy and controlled testing conditions.

    \item This research does not pursue full differential privacy proofs or external ledger integrations. Instead, it implements a lightweight privacy pipeline with PII redaction and aggregate-only reporting thresholds to demonstrate \textbf{privacy-aware insights} within the prototype.
\end{enumerate}

\section{Contributions}
\label{sec:contributions}

This thesis contributes a focused blueprint and evidence base for safety‑oriented agentic support:

\begin{enumerate}
    \item \textbf{Safety pipeline specification}. A concrete guideline for triage and escalation: risk cues and scoring, guardrails and redaction steps, decision thresholds, human‑in‑the‑loop invariants, and service targets such as time‑to‑escalation.
    \item \textbf{Agent orchestration design}. A LangGraph view of the Safety Agent Suite—nodes, edges, and typed state schemas—plus the supporting tool‑use protocol (validated schemas, idempotency, retry/backoff) that keeps workflows predictable.
    \item \textbf{Evaluation assets and findings}. Scenario-based tests (synthetic crisis set, adversarial prompts, blinded coaching rubric) and their results, covering safety sensitivity, orchestration reliability, latency, and coaching quality under human oversight.
\end{enumerate}

\section{Thesis Outline}
\label{sec:thesis_outline}

The structure of this thesis is outlined as follows:

\textbf{Chapter I: Introduction.} This chapter elaborates on the background of the study, the justification for the research's significance, the problem formulation to be addressed, and the specific objectives to be achieved. It also defines the scope and limitations of the research, outlines the expected contributions, and presents the overall organizational structure of the thesis report.

\textbf{Chapter II: Literature Review and Theoretical Framework.} This chapter surveys prior work on agentic and conversational AI for mental health, safety-critical triage systems, human-in-the-loop design, and privacy-aware analytics. It establishes the theoretical foundation that underpins the core concepts and technologies utilized in this research.

\textbf{Chapter III: System Design and Architecture.} This chapter outlines the methodology and technical blueprint for the system. It explains the adoption of Design Science Research and presents the system's high-level conceptual architecture, focusing on the four components of the \textbf{Safety Agent Suite}. It details the underlying cloud-native technical architecture, justifying the chosen technology stack, including the use of \textbf{LangGraph} for agent orchestration and a \textbf{FastAPI} backend for the core application logic. It also describes the database structure, user interface design, and integrated security and privacy measures like differential privacy.

\textbf{Chapter IV: Implementation and Evaluation.} This chapter describes the development and testing of the system prototype. This chapter details the technical environment used for implementation and demonstrates the functional prototype that was built. It then explains the testing process used to evaluate the system's performance against its design requirements. The chapter concludes by presenting the results from these tests and providing an analysis of the findings.

\textbf{Chapter V: Conclusion and Future Work.} This chapter summarizes the study's findings and contributions. This chapter revisits the initial research problems and presents the main conclusions drawn from the research. It concludes by offering recommendations for both the future development of the system and for subsequent research in this area.

% Akhir dari Chapter 1
