\chapter{Introduction}
\label{chap:introduction}

\section{Background}
\label{sec:background}

Higher Education Institutions (HEIs) are facing a critical and growing challenge in supporting student well-being \cite{hill2024studentwellbeing, duraku2024overcoming}. A landmark report highlights the escalating prevalence of mental health and substance use issues among student populations, urging institutions to adopt a more comprehensive support model \cite{scherer2021mentalhealth}. This crisis not only jeopardizes students' academic success and personal development but also places an immense, unsustainable strain on the institutions tasked with supporting them \cite{baik2019universities}.

The traditional support model, centered around on-campus counseling services, is fundamentally \textbf{reactive}. It relies on students to self-identify their distress and navigate the process of seeking help. This paradigm faces significant operational challenges, including insufficient staffing, long waiting lists, and an inability to provide immediate, 24/7 support, which ultimately limits access for a large portion of the student body \cite{baik2019universities}. Consequently, a critical gap persists between the need for mental health services and their actual provision, leaving many students without timely support \cite{outay2024multiagent}.

To bridge this gap, a paradigm shift from a reactive to a \textbf{proactive} support model is imperative \cite{outay2024multiagent}. The engine for this evolution is \textbf{Digital Transformation}, a process that leverages technology to fundamentally reshape organizational processes and enhance value delivery within HEIs \cite{omirali2025digitaltrust}. Within this context, Artificial Intelligence (AI) has emerged as a key enabling technology, with systematic reviews confirming its significant potential to analyze complex data, automate processes, and deliver personalized interventions at scale within the higher education landscape \cite{pati2025agentic, karunanayake2025nextgen}.

This research moves beyond conventional AI applications by proposing the use of \textbf{Agentic AI}. An intelligent agent is an autonomous system capable of perception, decision-making, and proactive action to achieve specific goals \cite{saleem2025multiagent}, representing a new frontier in educational technology \cite{salutari2024mas}. We propose that a framework built upon a system of collaborative intelligent agents, a Multi-Agent System (MAS), a concept already explored for smart campus management \cite{salutari2024mas}, can create a truly transformative ecosystem. Such a system would not only serve as a support tool for students but, more importantly, would function as a strategic asset for the institution, enabling data-driven decision-making, automating operational workflows, and facilitating a proactive stance on student well-being. This thesis details the design, development, and evaluation of such a framework, prototyped within the UGM-AICare project.


%-------------------------------------------------

\section{Problem Formulation}
\label{sec:problem_formulation}

The inefficiency and reactive nature of current university mental health support systems present a complex problem. To move towards a proactive and scalable model, this research addresses the following core challenges:

\begin{enumerate}
    \item The primary challenge is the \textbf{design of a cohesive, safety-oriented agentic AI framework} capable of automating key institutional processes. This requires a shift from a monolithic chatbot to a multi-agent system where specialized agents handle distinct tasks, including real-time crisis detection, personalized coaching, clinical case management, and privacy-preserving analytics.

    \item A significant technical challenge lies in the \textbf{implementation of a cloud-native architecture} to realize this framework. This involves orchestrating multiple AI agents using \textbf{LangGraph}, powered by a robust Large Language Model (Google Gemini), and integrating them within a secure FastAPI backend with an internal task scheduler.

    \item Finally, there is a need to \textbf{evaluate the potential impact} of such a framework on institutional operational efficiency and its ability to support data-driven, safety-first decision-making. This will be validated through a proof-of-concept prototype and scenario-based testing focused on the agentic workflows.
\end{enumerate}

To address these challenges, this thesis proposes and details the \textbf{Safety Agent Suite}, a framework comprised of four specialized, collaborative intelligent agents: a \textbf{Safety Triage Agent (STA)}, a \textbf{Support Coach Agent (SCA)}, a \textbf{Service Desk Agent (SDA)}, and an \textbf{Isingts Agent (IA)}.

%-------------------------------------------------	

\section{Objectives}
\label{sec:objectives}

The primary objectives of this thesis are:
\begin{enumerate}
    \item To design the conceptual and technical framework for the agentic AI system.
    \item To implement a functional proof-of-concept prototype.
    \item To evaluate the prototype's capabilities against predefined functional scenarios.
\end{enumerate}

%-------------------------------------------------	

\section{Scope and Limitations}
\label{sec:scope_and_limitations}

To ensure the feasibility and focus of this research, the following boundaries are established:

\begin{enumerate}
    \item This research is focused on the \textbf{design and prototype implementation of the agentic AI framework} (the Safety Agent Suite). The integration of content based on Cognitive Behavioral Therapy (CBT) and a blockchain-based gamification system serve as proof-of-concept features to validate the framework's capabilities, not as primary research areas themselves.

    \item The evaluation of the framework is based on \textbf{functional, scenario-based testing} of the prototype's agentic workflows. It does not measure the long-term psychological impact on students or the real-world operational savings for the institution.

    \item The data utilized for testing the analytics agent will consist of \textbf{anonymized, pre-existing chat logs or simulated data} to ensure user privacy and controlled testing conditions.

    \item This research will not provide an exhaustive analysis of differential privacy algorithms or blockchain scalability. It will, however, demonstrate their integration into the system to fulfill the architectural requirements of \textbf{privacy-first analytics} and \textbf{novel user engagement}.
\end{enumerate}

% \noindent Agar penelitian ini lebih terarah dan mendalam, ruang lingkupnya dibatasi sebagai berikut:

% \begin{enumerate}
%     \item \textbf{Objek Penelitian:} Fokus pada desain, pengembangan prototipe, dan evaluasi platform terintegrasi yang menggabungkan \textit{hybrid conversational AI}, \textit{gamification}, dan \textit{blockchain} untuk mahasiswa. Studi kasus dapat difokuskan pada mahasiswa UGM atau lingkungan universitas serupa di Indonesia.
%     \item \textbf{Teknologi:} Implementasi \textit{hybrid conversational AI} akan menggunakan kombinasi \textit{Pre-trained LLM} Gemini dengan RAG berbasis dokumen kesehatan mental UGM. Implementasi \textit{blockchain} akan menggunakan EDUChain (L2 Ethereum) untuk fitur \textit{achievement badges minting}. Mekanisme \textit{gamification} akan mencakup \textit{badges}, \textit{points}, dan sistem \textit{levelling}.
%     \item \textbf{Metode Penelitian:} Menggunakan pendekatan \textit{Design Science Research} (DSR) yang meliputi tahap desain artifak (platform), pengembangan prototipe, dan evaluasi. Evaluasi akan menggunakan metode campuran (\textit{mixed methods}), menggabungkan analisis data log penggunaan platform, survei, dan/atau wawancara dengan kelompok pengguna terbatas.
%     \item \textbf{Waktu dan Tempat Penelitian:} Penelitian dilaksanakan dalam periode [Sebutkan periode, misal: Semester Gasal 2024/2025 hingga Semester Genap 2024/2025] di lingkungan [Sebutkan, misal: Laboratorium Departemen Teknik Elektro dan Teknologi Informasi UGM dan melibatkan partisipan mahasiswa UGM secara daring].
%     \item \textbf{Populasi dan Sampel:} Populasi adalah mahasiswa aktif [Sebutkan jenjang, misal: S1] di [Sebutkan Fakultas/Universitas]. Sampel untuk evaluasi adalah sejumlah [Sebutkan jumlah, misal: 30-50] mahasiswa yang dipilih melalui [Sebutkan metode sampling, misal: purposive sampling atau voluntary sampling].
%     \item \textbf{Variabel dan Indikator:} Variabel independen adalah penggunaan platform dan fitur-fiturnya. Variabel dependen meliputi metrik \textit{user engagement} (misal: frekuensi login, durasi sesi, jumlah interaksi AI, progres gamifikasi) dan indikator kesejahteraan (misal: skor skala PSS - Perceived Stress Scale, skor skala koneksi sosial, feedback kualitatif).
%     \item \textbf{Keterbatasan Penelitian:} Penelitian ini tidak mencakup implementasi skala penuh di seluruh universitas. Evaluasi aspek kesejahteraan bersifat subjektif berdasarkan persepsi pengguna. Aspek keamanan \textit{blockchain} dievaluasi pada level prototipe dan tidak mencakup audit keamanan ekstensif. Model AI mungkin memiliki keterbatasan dalam menangani semua topik atau kondisi kesehatan mental yang kompleks dan tidak menggantikan konseling profesional.
% \end{enumerate}

%-------------------------------------------------

\section{Contributions}
\label{sec:contributions}

\begin{enumerate}
    \item Academic: A novel framework for applying agentic AI in an institutional (higher education) context.
    \item Practical: A blueprint for UGM to develop a more proactive, data-driven, and efficient mental health support system.
\end{enumerate}

% Penelitian ini diharapkan dapat memberikan manfaat signifikan bagi berbagai pihak:

% \begin{itemize}
%     \item \textbf{Bagi Mahasiswa:} Menyediakan akses terhadap platform digital inovatif yang dapat mendukung kebutuhan informasi, interaksi sosial, motivasi, dan akses ke sumber daya kesejahteraan secara terintegrasi, aman, dan menarik.
%     \item \textbf{Bagi Institusi Pendidikan (UGM dan lainnya):} Memberikan model dan bukti konsep (\textit{proof-of-concept}) mengenai pemanfaatan teknologi AI, \textit{gamification}, dan \textit{blockchain} untuk meningkatkan layanan dukungan mahasiswa, engagement, dan strategi peningkatan kesejahteraan di era digital.
%     \item \textbf{Bagi Komunitas Akademik dan Riset:} Menambah khazanah ilmu pengetahuan di bidang Teknik Informasi, khususnya terkait interaksi manusia-komputer (HCI), AI dalam pendidikan (\textit{AI in Education}), aplikasi \textit{blockchain} non-finansial, dan desain sistem terintegrasi untuk kesejahteraan digital (\textit{digital well-being}). Menjadi dasar bagi penelitian selanjutnya di area ini.
%     \item \textbf{Bagi Pengembang Teknologi:} Memberikan wawasan praktis mengenai tantangan dan strategi dalam mengintegrasikan tiga teknologi kompleks (\textit{conversational AI, gamification, blockchain}) dalam satu platform yang berfokus pada pengguna akhir.
% \end{itemize}

%-------------------------------------------------

\section{Thesis Outline}
\label{sec:thesis_outline}

The structure of this thesis is outlined as follows:

\textbf{Chapter I: Introduction.} This chapter elaborates on the background of the study, the justification for the research's significance, the problem formulation to be addressed, and the specific objectives to be achieved. It also defines the scope and limitations of the research, outlines the expected contributions, and presents the overall organizational structure of the thesis report.

\textbf{Chapter II: Literature Review and Theoretical Framework.} This chapter presents a comprehensive review of relevant prior research in the fields of conversational AI, the application of gamification in educational and well-being contexts, related blockchain applications, and studies on user engagement and student welfare. Furthermore, this chapter establishes the theoretical foundation that underpins the core concepts and technologies utilized in this research.

\textbf{Chapter III: System Design and Architecture.} This chapter outlines the methodology and technical blueprint for the system. It explains the adoption of Design Science Research and presents the system's high-level conceptual architecture, focusing on the four components of the \textbf{Safety Agent Suite}. It details the underlying cloud-native technical architecture, justifying the chosen technology stack, including the use of \textbf{LangGraph} for orchestration and a \textbf{FastAPI} backend. It also describes the database structure, user interface design, and integrated security and privacy measures like differential privacy.

\textbf{Chapter IV: Implementation and Evaluation.} This chapter describes the development and testing of the system prototype. This chapter details the technical environment used for implementation and demonstrates the functional prototype that was built. It then explains the testing process used to evaluate the system's performance against its design requirements. The chapter concludes by presenting the results from these tests and providing an analysis of the findings.

\textbf{Chapter V: Conclusion and Future Work.} This chapter summarizes the study's findings and contributions. This chapter revisits the initial research problems and presents the main conclusions drawn from the research. It concludes by offering recommendations for both the future development of the system and for subsequent research in this area.

% Akhir dari Chapter 1
