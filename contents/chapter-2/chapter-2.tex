\chapter{Literature Review and Theoretical Background}

This chapter establishes the academic context for the research. It begins by surveying the existing literature on AI applications in mental health and student support to identify the limitations of current approaches. It then details the theoretical framework and enabling technologies that provide the foundation for the proposed solution. Finally, it synthesizes these areas to formally identify the research gap this thesis addresses.

% ==================================================================================
% ==================================================================================
% SECTION 1: Literature Review
% ==================================================================================
% ==================================================================================

\section{Literature Review: The Landscape of AI in University Mental Health Support}
\label{sec:literature_review}

This review surveys existing research at the intersection of artificial intelligence, institutional support systems, and student mental health. The aim is to contextualize the present work by examining the evolution and limitations of current approaches, thereby setting the stage for the introduction of a more advanced, agentic framework.

\subsection{Conversational Agents for Mental Health Support}

The application of conversational agents in mental health has evolved significantly, from early experiments in simulating dialogue to sophisticated, evidence-based therapeutic tools. This evolution reveals both the immense potential of these technologies and the persistent operational limitations that motivate the current research.

\subsubsection{Evolution from Rule-Based Systems to LLM-Powered Agents}
The concept of using a computer program for therapeutic dialogue dates back to Weizenbaum's ELIZA (1966), a system that used simple keyword matching and canned response templates to mimic a Rogerian psychotherapist \cite{shum2018conversationalsystems,alamin2024historychatbots}. While a landmark in human-computer interaction, ELIZA and subsequent rule-based systems lacked any true semantic understanding, memory, or capacity for evidence-based intervention. Their primary limitation was their inability to move beyond superficial pattern recognition, leading to brittle and often nonsensical conversations when faced with inputs outside their predefined rules \cite{shum2018conversationalsystems}.

The advent of Large Language Models (LLMs) has catalyzed a paradigm shift. Modern conversational agents, powered by Transformer architectures, can generate fluent, empathetic, and context-aware responses. These models are pre-trained on vast text corpora, enabling them to understand linguistic nuance and generate human-like text. This has allowed for the development of agents that can engage in more meaningful, multi-turn conversations, moving beyond simple question-answering to provide more substantive support \cite{alamin2024historychatbots}.

\subsubsection{Therapeutic Applications and Efficacy}
Contemporary mental health chatbots leverage LLMs to deliver a range of evidence-based interventions. A primary application is the delivery of psychoeducation and structured exercises from therapeutic modalities like Cognitive Behavioral Therapy (CBT). Systems such as Woebot have been the subject of randomized controlled trials (RCTs), which have demonstrated their efficacy in reducing symptoms of depression and anxiety among university students by delivering daily, brief, conversational CBT exercises \cite{woebotRCT2024,eltahawy2024robotsdottherapy}. Other platforms, like Tess, have shown similar positive outcomes by providing on-demand emotional support and coping strategies.

These tools offer several key advantages:
\begin{itemize}
    \item \textbf{Accessibility and Scalability:} They are available 24/7, overcoming the time and resource constraints of traditional human-led services.
    \item \textbf{Anonymity:} They provide a non-judgmental and anonymous space for users to disclose their feelings, which can lower the barrier for individuals who fear stigma \cite{kang2025chatbotstigma}.
\end{itemize}

\subsubsection{The Dominant Reactive Paradigm and Its Limitations}
Despite their technological sophistication and therapeutic potential, the fundamental operational model of these applications remains overwhelmingly \textbf{reactive and user-initiated}. They are designed as standalone tools that depend on the student to possess the self-awareness to recognize their distress, the motivation to seek help, and the knowledge of the tool's existence.

This paradigm fails to account for significant, well-documented barriers to help-seeking. Research shows that many individuals, particularly young adults, do not seek professional help for mental health issues due to factors including self-stigma, fear of judgment, and a desire for self-reliance \cite{corrigan2009stigmahelpseeking,patel2022helpseekingcollege}. Furthermore, the very symptoms of mental health conditions, such as the anhedonia and executive dysfunction associated with depression, can severely impair an individual's ability to initiate action and seek support \cite{liu2023distresshelpseeking}.

A systematic review of mental health chatbots for university students concluded that while these tools are promising, their primary limitation is their passive nature; they do not and cannot initiate contact or intervene based on a student's changing needs unless the student opens the app \cite{adhikari2023chatbotsreview}. This leaves the most vulnerable students—those who are not actively seeking help—unsupported, creating a critical gap in the continuum of care that this thesis aims to address.


\subsection{Data Analytics for Proactive Student Support}

Parallel to the development of conversational AI, the field of higher education has seen a rise in the use of data analytics to support student success. This section reviews the evolution of these analytical approaches, from established learning analytics to the more nascent field of well-being analytics, and identifies the key limitations that motivate the design of the Analytics and Intervention Agents.

\subsubsection{Learning Analytics for Academic Intervention}
The domain of \textbf{Learning Analytics} is well-established and focuses on the "measurement, collection, analysis and reporting of data about learners and their contexts, for purposes of understanding and optimising learning and the environments in which it occurs" \cite{siemens2011learninganalytics}. Typically, these systems analyze data from institutional sources such as the Learning Management System (LMS), student information systems, and library databases. By modeling variables like assignment submission times, forum participation, and grades, institutions can build predictive models to identify students at high risk of academic failure or dropout \cite{banihashem2022predictivedropout}. These systems have proven effective in enabling timely academic interventions, such as targeted tutoring or advisor outreach, thereby improving student retention and success rates.

\subsubsection{The Challenge of Well-being Analytics}
More recently, researchers have attempted to extend the principles of learning analytics to the more complex and sensitive domain of student well-being. The goal is to create early-warning systems by identifying behavioral proxies for mental distress. Studies have explored the use of non-academic data sources, such as campus card usage for building access, meal plan data, and social event attendance, to find correlations with well-being outcomes \cite{paolucci2024wellbeinganalytics}. For example, a sudden decrease in social activity or irregular campus attendance could be interpreted as a potential indicator of withdrawal or depression.

However, this approach is fraught with significant theoretical and practical challenges. Firstly, the "signal-to-noise" ratio is extremely low; the link between such indirect behavioral data and a student's internal mental state is often weak, correlational, and highly prone to misinterpretation \cite{masiello2024privacyethics}. A student may miss meals for many reasons other than depression. Secondly, these methods raise profound ethical questions regarding student privacy and surveillance, as they involve monitoring non-academic aspects of student life, often without explicit, ongoing consent for this specific purpose \cite{masiello2024privacyethics,paolucci2024wellbeinganalytics}.

A more direct, and arguably more ethical, source of data is the language students use when interacting with university services. The text from chat logs, when properly anonymized, provides a direct window into student concerns. The application of sentiment analysis and topic modeling to this textual data can yield far more reliable insights into the specific stressors affecting the student population at any given time. This approach, which is central to the design of the Analytics Agent, shifts the focus from inferring mental state from indirect behaviors to directly analyzing the expressed concerns of the student body \cite{paolucci2024wellbeinganalytics}.

\subsubsection{The Insight-to-Action Gap}
Whether based on academic, behavioral, or textual data, a critical limitation plagues nearly all current analytical systems in higher education: the \textbf{insight-to-action gap} \cite{jorno2018actionableinsight}. The output of these systems is almost universally a dashboard, a report, or an alert delivered to a human administrator (e.g., a counselor, dean, or advisor) \cite{susnjak2022dashboard}. This administrator must then manually interpret the data, decide on an appropriate intervention strategy, and execute it.

This manual process creates a severe bottleneck that fundamentally limits the scalability, speed, and personalization of any proactive effort \cite{kaliisa2023hypedashboards}. An administrator may be able to respond to a handful of individual alerts, but they cannot manually orchestrate a personalized outreach campaign to hundreds of students who may be exhibiting early signs of exam-related stress identified by a topic model. The manual-execution step prevents the institution from fully capitalizing on the proactive insights generated by its analytical systems. It is this specific gap that the proposed \textbf{Safety Coaching Agent} and \textbf{Safety Traige Agent} is designed to close by automating the link between data-driven insight and scalable, targeted outreach.

% ============================================================
% ============================================================
% SECTION 2: Theoretical Background
% ============================================================
% ============================================================

\section{Theoretical Background}
\label{sec:theoretical_background}

To address the limitations of reactive, disconnected support systems, a new architectural approach is required. This section details the theoretical framework and enabling technologies that provide the foundation for the proposed agentic AI system. These concepts are presented as the necessary components to build a proactive, integrated, and autonomous solution.

%%%%%%%%%%%%%%%%%%%%%%%%%%%%%%%%%%%%%%%%%%%%%%%%%%%%%%%%%%%%%%
% SubSection: Foundational Principles of the Framework
%%%%%%%%%%%%%%%%%%%%%%%%%%%%%%%%%%%%%%%%%%%%%%%%%%%%%%%%%%%%%%
\subsection{Foundational Principles of the Framework}
\label{subsec:foundational_principles}

Beyond the technical architecture, the proposed framework is grounded in several key strategic and ethical principles that justify its design and purpose. These concepts from service design, management science, and data ethics provide the theoretical motivation for shifting how institutional support is delivered.

\subsubsection{Proactive vs. Reactive Support Models}
The traditional approach to institutional support, particularly in mental health, is predominantly \textbf{reactive}. This model, common in service design, operates on a "break-fix" basis, where the service delivery is initiated only after a user (in this case, a student) self-identifies a problem and actively seeks a solution \cite{freeman2025competitionalgorithms}. This places the onus of initiation entirely on the individual, creating significant barriers to access such as stigma, lack of awareness, or the inability to act during a crisis. In contrast, a \textbf{proactive support model} aims to anticipate needs and intervene before a problem escalates. Drawing from principles in preventative healthcare and proactive customer relationship management, this model uses data to identify patterns and risk factors, enabling the institution to offer timely, relevant support to at-risk cohorts \cite{williams2022datadrivenhe, lyon2020datadrivenuniversity}. This thesis is an explicit attempt to architect a system that facilitates this strategic shift from a reactive to a proactive support paradigm.

\subsubsection{Data-Driven Decision-Making in Higher Education}
The concept of \textbf{Data-Driven Decision-Making (DDDM)} posits that strategic decisions should be based on objective data analysis and interpretation rather than solely on intuition or tradition \cite{williams2022datadrivenhe, lyon2020datadrivenuniversity}. In higher education, this has manifested as the field of learning analytics, where student data is used to improve learning outcomes and retention. This framework extends that principle to student well-being. The \textbf{Insights Agent} is the core enabler of DDDM for the university's support services. By autonomously processing anonymized interaction data to identify trends, sentiment shifts, and emerging topics of concern, it provides administrators with actionable, empirical evidence. This allows the institution to move beyond anecdotal evidence and allocate resources—such as workshops, counselors, or targeted information campaigns—to where they are most needed, thereby optimizing the efficiency and impact of its support ecosystem \cite{popoola2025privacyawareframework}.

\subsubsection{Privacy by Design (PbD)}
Given the highly sensitive nature of mental health data, the framework's architecture is guided by the principles of \textbf{Privacy by Design (PbD)}. PbD is an internationally recognized framework, formalized in ISO 31700, which dictates that privacy should be the default, embedded into the design and architecture of systems from the outset rather than being an add-on feature \cite{guarda2024certificationsprivacy,atabey2024ethicsinedtech}. Key principles include being proactive not reactive, making privacy the default setting, and providing end-to-end security. A direct implementation of PbD within this framework is the Data Anonymization Pipeline. This process ensures that Personally Identifiable Information (PII) is identified and redacted from all chat logs before they are stored for analysis. Furthermore, access to the administrative dashboard is controlled by a strict Role-Based Access Control (RBAC) mechanism, ensuring that only authorized personnel can view sensitive data. These measures, combined with standard security practices like data encryption, embed privacy and security directly into the system's architecture from the outset \cite{popoola2025privacyawareframework, guarda2024certificationsprivacy}. This demonstrates a commitment to building a system that is not only effective but also fundamentally ethical and secure.

%%%%%%%%%%%%%%%%%%%%%%%%%%%%%%%%%%%%%%%%%%%%%%%%%%%%%%%%%%%%%
% SubSection: Agentic AI, Multi-Agent Systems, and LLMs
%%%%%%%%%%%%%%%%%%%%%%%%%%%%%%%%%%%%%%%%%%%%%%%%%%%%%%%%%%%%%
\subsection{Agentic AI and Multi-Agent Systems (MAS)}
\label{subsec:agentic_ai}

The paradigm of Artificial Intelligence (AI) has evolved significantly from systems that perform singular, reactive tasks to those that exhibit autonomous, proactive, and social behaviors. A cornerstone of this evolution is the concept of an \textbf{intelligent agent}. An agent is not merely a program; it is a persistent computational entity with
a degree of autonomy, situated within an environment, which it can both perceive and upon which it can act to achieve a set of goals or design objectives \cite{wooldridge1995intelligentagents}. The defining characteristic of an agent is its \textbf{autonomy}—its capacity to operate independently, making decisions and initiating actions without direct, constant human intervention. This is distinct from traditional objects, which are defined by their methods and attributes but do not exhibit control over their own behavior \cite{wooldridge2009introductionmas}.

To operationalize this concept, this thesis formally introduces a framework built upon four distinct, specialized intelligent agents that form the \textbf{Safety Agent Suite}. Each agent is designed to address a specific challenge outlined in Chapter 1, and together they form the core of the proposed proactive support system. These agents are:
\begin{itemize}
    \item The \textbf{Safety Triage Agent (STA)}, responsible for real-time risk assessment and crisis intervention.
    \item The \textbf{Support Coach Agent (SCA)}, responsible for delivering personalized, evidence-based coaching.
    \item The \textbf{Service Desk Agent (SDA)}, responsible for managing clinical case workflows and administrative tasks.
    \item The \textbf{Insights Agent (IA)}, responsible for privacy-preserving data analysis and trend identification.
\end{itemize}

The theoretical underpinnings of these agents' architecture and behavior are drawn from established models of rational agency and multi-agent systems, as detailed below.

Fundamentally, an agent's operation is defined by a continuous cycle of perception, reasoning (or deliberation), and action. It perceives its environment through virtual \textbf{sensors} (e.g., data feeds, API calls, database queries) and influences that environment through its \textbf{actuators} (e.g., sending emails, generating reports, invoking other services) \cite{yan2024explainablebdi}. A prominent and highly relevant architecture for designing such goal- oriented agents is the \textbf{Belief-Desire-Intention (BDI)} model \cite{rao1995bdi, yan2024explainablebdi}. This model provides a framework for rational agency that mirrors human practical reasoning:

\begin{itemize}
    \item \textbf{Beliefs:} This represents the informational state of the agent—its knowledge about the environment, which may be incomplete or incorrect. For the \textbf{Insights Agent}, beliefs correspond to the current understanding of student well-being trends derived from anonymized data.
    \item \textbf{Desires:} These are the motivational states of the agent, representing the objectives or goals it is designed to achieve. Desires can be seen as the potential tasks the agent could undertake, such as the \textbf{Support Coach Agent's} overarching goal to "deliver personalized coaching."
    \item \textbf{Intentions:} This represents the agent's commitment to a specific plan or course of action. An intention is a desire that the agent has chosen to actively pursue. For instance, the \textbf{Safety Triage Agent}, upon identifying a high-severity conversation, forms an intention to immediately route the user to emergency resources.
\end{itemize}

The BDI framework allows for the design of agents that are not merely reactive but are proactive and deliberative, capable of reasoning about how to best achieve their goals given their current beliefs about the world \cite{wooldridge2009introductionmas, rao1995bdi}.

To formally ground the proposed framework in this established model, the roles and logic of each of the four agents are mapped to the BDI components in Table \ref{tab:bdi_mapping}. This mapping clarifies how each agent perceives its environment, formulates its objectives, and decides on a concrete course of action, allowing for the design of agents that are not merely reactive but are proactive and deliberative, capable of reasoning about how to best achieve their goals given their current beliefs about the world.

% This table maps the proposed intelligent agents to the Belief-Desire-Intention (BDI) model of rational agency.
\begin{table}[H] % Using [H] from the 'float' package to place the table HERE
    \centering
    \caption{Mapping of the Agentic Framework to the BDI Model.}
    \label{tab:bdi_mapping}
    \small % Reducing font size to make the table more compact
    \renewcommand{\arraystretch}{1.4} % Adjusting row spacing
    \begin{tabular}{lp{0.3\textwidth}p{0.25\textwidth}p{0.25\textwidth}}
        \toprule
        \textbf{Agent} & 
        \textbf{Beliefs} \textit{(Informational State)} & 
        \textbf{Desires} \textit{(Motivational Goals)} & 
        \textbf{Intentions} \textit{(Committed Plans)} \\
        \midrule

        \textbf{Safety Triage Agent (STA)} &
        \begin{itemize} \itemsep0em
            \item The user's live conversation history.
            \item A classification model for conversation severity.
            \item A directory of emergency resources.
        \end{itemize} &
        \begin{itemize} \itemsep0em
            \item To assess a user's immediate risk level accurately.
            \item To provide the most appropriate support for the user's current state.
        \end{itemize} &
        \begin{itemize} \itemsep0em
            \item Upon detecting a high-severity utterance, form an intention to immediately escalate.
            \item To retrieve and display emergency contact information without delay.
        \end{itemize} \\
        \midrule

        \textbf{Support Coach Agent (SCA)} &
        \begin{itemize} \itemsep0em
             \item User's stated goals and conversation history.
             \item A library of evidence-based interventions (e.g., CBT exercises).
        \end{itemize} &
        \begin{itemize} \itemsep0em
            \item To deliver personalized coaching.
            \item To guide the user through therapeutic exercises.
        \end{itemize} &
        \begin{itemize} \itemsep0em
            \item Based on user input, form an intention to deliver a specific CBT exercise.
            \item To provide empathetic and supportive responses.
        \end{itemize} \\
        \midrule

        \textbf{Service Desk Agent (SDA)} &
        \begin{itemize} \itemsep0em
            \item Status of clinical cases.
            \item Availability of counselors.
            \item User requests for appointments.
        \end{itemize} &
        \begin{itemize} \itemsep0em
            \item To manage clinical case workflows.
            \item To schedule appointments efficiently.
        \end{itemize} &
        \begin{itemize} \itemsep0em
            \item Upon a user request, form an intention to find an available appointment slot.
            \item To create and update case notes.
        \end{itemize} \\
        \midrule
        
        \textbf{Insights Agent (IA)} & 
        \begin{itemize} \itemsep0em
            \item Access to the anonymized conversation log database.
            \item The timestamp of the last generated report.
            \item A model of known topics.
        \end{itemize} &
        \begin{itemize} \itemsep0em
            \item To identify emerging trends in student concerns.
            \item To quantify shifts in overall student sentiment.
        \end{itemize} &
        \begin{itemize} \itemsep0em
            \item Upon its weekly trigger, form an intention to generate a new summary report.
            \item To formulate and execute a database query for the past week's data.
        \end{itemize} \\
        \bottomrule
    \end{tabular}
\end{table}

When multiple agents, each with its own goals and capabilities, co-exist and interact within a shared environment, they form a \textbf{Multi-Agent System (MAS)}. An MAS is a system in which the overall intelligent behavior and functionality are a product of the collective, emergent dynamics of its constituent agents \cite{burguillo2017multiagentsystems, petrova2025agenticweb}. The power of an MAS lies in its ability to solve problems that would be difficult or impossible for a monolithic system or a single agent to handle. This is achieved through social interaction, primarily:

\begin{itemize}
    \item \textbf{Coordination and Cooperation:} Agents must coordinate their actions to avoid interference and cooperate to achieve common goals. In this thesis, the \textbf{Insights}, \textbf{Support Coach}, \textbf{Safety Triage}, and \textbf{Service Desk} agents must cooperate: the Insights Agent provides the data-driven insights (beliefs) that the Support Coach Agent uses to form its outreach plans (intentions), while the Safety Triage Agent handles immediate, real-time needs that may fall outside the other agents' scopes, and the Service Desk Agent manages the administrative follow-up.
    \item \textbf{Negotiation:} When agents have conflicting goals or must compete for limited resources, they must be able to negotiate to find a mutually acceptable compromise \cite{paurobally2002rationalagents, agerri2006unifiedcommunication}.
    \item \textbf{Communication:} Effective interaction requires a shared Agent Communication Language (ACL), such as FIPA-ACL or KQML, which defines the syntax and semantics for messages, allowing agents to perform actions like requesting information, making proposals, and accepting or rejecting tasks \cite{fornara2003interaction, williams2025multicommunication}.
\end{itemize}

Therefore, this thesis leverages the MAS paradigm by designing a framework composed of four specialized, collaborative agents. Their individual, goal-directed behaviors, orchestrated within a hybrid architecture, work in concert to achieve the overarching systemic objective: transforming institutional mental health support from a reactive model to a proactive, data-driven ecosystem.

%%%%%%%%%%%%%%%%%%%%%%%%%%%%%%%%%%%%%%%%%%%%%%%%%%%%%%%%%%%%%%
% SubSection: Large Language Models (LLMs)
%%%%%%%%%%%%%%%%%%%%%%%%%%%%%%%%%%%%%%%%%%%%%%%%%%%%%%%%%%%%%%
\subsection{Large Language Models (LLMs)}
\label{subsec:llms}

Large Language Models (LLMs) are a class of deep learning models that have demonstrated remarkable capabilities in understanding and generating human-like text. The architectural foundation for virtually all modern LLMs, including the Gemini models used in this research, is the \textbf{Transformer architecture}, first introduced by Vaswani et al. \cite{vaswani2017attention}. The Transformer's key innovation is the \textbf{self-attention mechanism}, which allows the model to dynamically weigh the importance of different words in an input sequence when processing and generating language. This enables the model to capture complex, long-range dependencies and contextual relationships far more effectively than its predecessors, such as Recurrent Neural Networks (RNNs) \cite{liu2023transformersurvey, vaswani2017comparisonrnn}.

The core operation of a Transformer-based model involves processing input text through a series of encoding and/or decoding layers. The process can be conceptualized as follows:
\begin{enumerate}
    \item \textbf{Tokenization and Embedding:} Input text is first broken down into smaller units called tokens. Each token is then mapped to a high-dimensional vector, or an "embedding," that represents its semantic meaning.
    \item \textbf{Positional Encoding:} Since the self-attention mechanism does not inherently process sequential order, a positional encoding vector is added to each token embedding to provide the model with information about the word's position in the sequence.
    \item \textbf{Self-Attention Layers:} The sequence of embeddings passes through multiple self-attention layers. In each layer, the model calculates attention scores for every token relative to all other tokens in the sequence, effectively learning which parts of the input are most relevant for understanding the context of each specific token.
    \item \textbf{Feed-Forward Networks:} Each attention layer is followed by a feed-forward neural network that applies further transformations to each token's representation.
    \item \textbf{Output Generation:} The model's final output is a probability distribution over its entire vocabulary for the next token in the sequence. The model then typically selects the most likely token (or samples from the distribution) and appends it to the input, repeating the process autoregressively to generate coherent text \cite{liu2023transformersurvey}.
\end{enumerate}

\begin{figure}[htbp]
  \centering
  % \includegraphics[width=\textwidth]{placeholder_diagram_transformer.png} % Ganti dengan file gambar Anda
  \fbox{\parbox[c][8cm][c]{0.9\textwidth}{\centering \textbf{Placeholder for Diagram: Simplified Transformer Architecture for Generative LLMs} \\ \vspace{1cm} This diagram should illustrate the flow of information: \\ 1. Input Text -> Tokenizer \\ 2. Tokens -> Embedding Layer + Positional Encoding \\ 3. Embedded Tokens -> A stack of 'N' Decoder Blocks \\ 4. Inside a Decoder Block: Multi-Head Self-Attention -> Feed-Forward Network \\ 5. Final Output -> Linear Layer -> Softmax -> Probability of Next Token}}
  \caption{A simplified view of the decoder-only Transformer architecture used in generative LLMs. The model processes input embeddings through multiple layers of self-attention and feed-forward networks to predict the next token in a sequence.}
  \label{fig:transformer_architecture}
\end{figure}

This research utilizes a cloud-based API model strategy, leveraging the Gemini 2.5 family of models to balance performance, privacy, and capability. The Gemini models represent Google's state-of-the-art, natively multimodal foundation models, available in various sizes (e.g., Gemini Pro). Unlike models trained solely on text, Gemini was pre-trained from the ground up on multiple data modalities, giving it more sophisticated reasoning capabilities \cite{google2025gemini2_5}. In this framework, a powerful model like Gemini 2.5 Pro is accessed via a secure API for all agentic tasks \cite{gemini_api_docs}, from the real-time conversation handling of the Safety Triage Agent to the complex, non-sensitive tasks, such as the weekly trend analysis performed by the Insights Agent.

\subsubsection{Cloud-Based API Models: The Gemini 2.5 Family}

The framework integrates a state-of-the-art, proprietary model accessed via a cloud API. The Gemini family, specifically the flagship \textbf{Gemini 2.5 Flash} model, serves this role, providing a level of reasoning and multimodal understanding that is critical for handling the most complex tasks and ensuring system robustness. While a detailed architectural schematic is not public, in line with the proprietary nature of frontier AI models, its capabilities have been extensively documented by Google through official developer guides and announcements \cite{google2025gemini2_5,gemini_api_docs}.

Gemini 2.5 builds upon the efficient \textbf{Mixture-of-Experts (MoE) Transformer} architecture of its predecessors. In an MoE architecture, the model is composed of numerous smaller "expert" neural networks. For any given input, a routing mechanism activates only a sparse subset of these experts. This allows the model to have a very large total parameter count—enabling vast knowledge and capability—while keeping the computational cost for any single inference relatively low \cite{google2025gemini2_5}.

The strategic role of Gemini 2.5 in this framework is defined by its next-generation capabilities:
\begin{itemize}
    \item \textbf{Native Multimodality with Expressive Audio:} A significant architectural leap in Gemini 2.5 is its native handling of audio \cite{googleblogaudio2025}. Unlike models that first transcribe audio to text, Gemini 2.5 processes audio streams directly. This allows it to understand not just the words, but also the nuances of human speech such as tone, pitch, and prosody, which is invaluable for a mental health application where user sentiment is key.
    \item \textbf{Controllable Reasoning and "Thinking Time":} Gemini 2.5 introduces a "thinking budget," a mechanism that allows developers to control the trade-off between response latency and reasoning depth \cite{google2025gemini2_5}. For high-frequency tasks performed by the Safety Triage Agent, a lower budget can ensure speed. For complex analytical tasks required by the Insights Agent, a higher budget can be allocated to allow for more thorough reasoning, providing granular control over both cost and quality.
    \item \textbf{Advanced Agentic Capabilities and Tool Use:} The model is explicitly designed to power advanced agents. It features more reliable and sophisticated function calling, enabling seamless integration with external tools and APIs \cite{google2025gemini2_5}. This is essential for the Service Desk Agent to execute multi-step plans, such as scheduling an appointment based on a user's request.
    \item \textbf{High-Fidelity Reasoning:} As a frontier model, Gemini 2.5 serves as the high-capability engine for all requests, ensuring service continuity and the highest quality output.
\end{itemize}

By integrating Gemini 2.5 via its API, the agentic framework gains access to state-of-the-art reasoning power on demand, ensuring that it can handle a wide spectrum of tasks with both efficiency and exceptional quality.


%%%%%%%%%%%%%%%%%%%%%%%%%%%%%%%%%%%%%%%%%%%%%%%%%%%%%%%%%%%%%%
% SubSection: LLM Orchestration Frameworks (LangChain)
%%%%%%%%%%%%%%%%%%%%%%%%%%%%%%%%%%%%%%%%%%%%%%%%%%%%%%%%%%%%%%
\subsection{LLM Orchestration Frameworks}
\label{subsec:llm_orchestration}

While LLMs provide powerful reasoning capabilities, they are inherently stateless and lack direct access to external data or tools. An LLM, in isolation, cannot query a database, call an API, or access a private document. To build sophisticated, stateful applications that overcome these limitations, an orchestration framework is required.

\subsubsection{LangChain: The Building Blocks of LLM Applications}
\label{subsubsec:langchain_building_blocks}

\textbf{LangChain} is an open-source framework designed specifically for this purpose, providing the essential "glue" to connect LLMs with external resources and compose them into complex applications \cite{barua2024llmagentsreview,yu2025agentworkflow}. The core philosophy of LangChain is to provide modular components that can be "chained" together to create complex workflows. The most recent and fundamental abstraction in LangChain is the \textbf{LangChain Expression Language (LCEL)}. LCEL provides a declarative, composable syntax for building chains, where the pipe (`|`) operator streams the output of one component into the input of the next. Every component in an LCEL chain is a "Runnable," a standardized interface that supports synchronous, asynchronous, batch, and streaming invocations, making it highly versatile for production environments \cite{yu2025agentworkflow, pospech2025metagraph}.

A simple LCEL chain can be represented as:
$$ \text{Chain} = \text{PromptTemplate} \ | \ \text{LLM} \ | \ \text{OutputParser} $$
In this sequence, user input is first formatted by a `PromptTemplate`, the result is passed to the `LLM` for processing, and the LLM's raw output is then transformed into a structured format (e.g., JSON) by an `OutputParser`.

For this thesis, the most critical application of LangChain is its ability to create \textbf{agents}. A LangChain agent uses an LLM not just for text processing, but as a reasoning engine to make decisions. This is often based on a framework known as \textbf{ReAct (Reasoning and Acting)}, which enables the LLM to synergize reasoning and action \cite{yao2022react, barua2024llmagentsreview}. The agent is given access to a set of \textbf{Tools}—which are simply functions that can interact with the outside world (e.g., a database query function, a file reader, a web search API). The agent's operational loop, managed by an \textbf{Agent Executor}, can be formalized as an iterative process.

Let $G$ be the initial goal and $H_t$ be the history of actions and observations up to step $t$. The process at each step $t$ is:
\begin{enumerate}
    \item \textbf{Reasoning (Thought Generation):} The agent generates a thought $th_t$ and a subsequent action $a_t$ by sampling from the LLM's conditional probability distribution, given the goal and the history so far.
    $$ (th_t, a_t) \sim p(th, a | G, H_{t-1}; \theta_{LLM}) $$
    The prompt to the LLM contains the goal and the trajectory of previous thoughts, actions, and observations, guiding its next decision.

    \item \textbf{Action Execution:} The Agent Executor parses $a_t$ to identify the chosen tool and its input, then executes it to produce an observation, $o_t$.
    $$ o_t = \text{ExecuteTool}(a_t) $$

    \item \textbf{History Augmentation:} The new observation is appended to the history, forming the context for the next iteration.
    $$ H_t = H_{t-1} \oplus (a_t, o_t) $$
    This loop continues until the LLM determines the goal $G$ is met and generates a final answer.
\end{enumerate}
This iterative loop is what transforms a passive LLM into a proactive, problem-solving agent. For example, the \textbf{Insights Agent} in this framework, when tasked with "summarizing student stress trends," would use this loop to formulate a SQL query (Thought and Action), execute it (Observation), and then use the results to generate a final summary. This orchestration is fundamental to enabling the autonomous capabilities central to this thesis.

% --- DESCRIPTION FOR A DIAGRAM ---
% You could add a diagram here with the following description:
% \begin{figure}[htbp]
%   \centering
%   % \includegraphics[width=\textwidth]{path/to/your/diagram.png}
%   \caption{Conceptual Diagram of a LangChain Agent based on the ReAct Framework. The diagram shows a central box labeled "Agent (LLM Reasoning Core)". An arrow labeled "Goal/Input" points to this box. From the agent box, several arrows point outwards to a set of boxes labeled "Tools", such as "Database Query", "Web Search", and "Calculator". The agent box has a looping arrow on it labeled "Thought -> Action -> Observation". An arrow labeled "Final Answer" points away from the agent box, showing the result after the loop is complete. This visualizes how the LLM core decides which tool to use to accomplish a goal.}
%   \label{fig:langchain_agent}
% \end{figure}

\subsubsection{LangGraph: Orchestrating Multi-Agent Systems}

While LangChain's standard agent executors are powerful, they are often designed for linear, sequential execution paths. For a sophisticated multi-agent system like the \textbf{Safety Agent Suite}, where agents must collaborate, hand off tasks, and operate in a cyclical, stateful manner, a more robust orchestration mechanism is required. This is the role of \textbf{LangGraph}, an extension of LangChain designed for building durable, stateful, multi-agent applications by modeling them as cyclical graphs \cite{yang2025aiagentprotocols,rauch2025modularagents}.

The core concept of LangGraph is to represent the agentic workflow as a \textbf{state graph}. This is a directed graph where nodes represent functions or LLM calls (the "work" to be done) and edges represent the conditional logic that directs the flow of execution from one node to another. A central \textbf{State} object is passed between nodes, allowing each agent or tool to read the current state, perform its function, and then update the state with its results. This creates a persistent, auditable record of the agent's operations \cite{mathew2025largelanguagemodelagents,pospech2025metagraph}.

A LangGraph workflow can be defined by the following components:
\begin{itemize}
    \item \textbf{State Graph:} The overall structure, $G = (N, E)$, where $N$ is a set of nodes and $E$ is a set of directed edges. The graph's state is explicitly defined by a state object that is passed and updated throughout the execution.
    \item \textbf{Nodes:} Each node represents an agent or a tool. When called, a node receives the current state object as input and returns a dictionary of updates to be applied to the state. For example, the `Safety Triage Agent` node would take the user's message from the state, process it, and return an update specifying the assessed risk level.
    \item \textbf{Edges:} Edges connect the nodes and control the flow of the application. LangGraph supports \textbf{conditional edges}, which are crucial for agentic behavior. After a node executes, a routing function is called to inspect the current state and decide which node to move to next \cite{yu2025agentworkflow,pospech2025metagraph}. For example, after the `Safety Triage Agent` runs, a conditional edge might route the workflow to the `Service Desk Agent` if the risk is moderate, or directly to an "escalate" tool if the risk is critical.
\end{itemize}

This cyclical, stateful approach provides several key advantages for this framework:
\begin{enumerate}
    \item \textbf{Explicit Multi-Agent Collaboration:} LangGraph allows for the explicit definition of workflows where different agents are called in sequence or in parallel, and their outputs are used to inform the next step \cite{tran2025multiagentcollaboration, mathew2025largelanguagemodelagents}. This is essential for the \textbf{Safety Agent Suite}, where the `Insights Agent`'s output must trigger the `Support Coach Agent`.
    \item \textbf{State Management and Durability:} Because the state is explicitly managed, the agent's "memory" of the conversation and its previous actions is robust. The graph's execution can be paused, resumed, and inspected, which is vital for long-running, interactive coaching sessions.
    \item \textbf{Flexibility and Control:} Unlike the more constrained loops of standard agent executors, LangGraph allows for the creation of arbitrary cycles. An agent can loop, retry a tool call if it fails, or route to a human-in-the-loop for verification, providing a much higher degree of control and reliability for a safety-critical application \cite{tang2025autoagent, aquinodeazevedo2025ragtomultiagent}.
\end{enumerate}

By using LangGraph to orchestrate the \textbf{Safety Agent Suite}, this framework moves beyond simple, linear agentic loops and implements a true multi-agent system capable of complex, stateful, and collaborative problem-solving \cite{yang2025aiagentprotocols, tran2025multiagentcollaboration}.

% ============================================================
% ============================================================
% Section: Synthesis and Identification of the Research Gap
% ============================================================
% ============================================================

\section{Synthesis and Identification of the Research Gap}

The preceding review of the literature and theoretical landscape reveals a critical disconnect. On one hand, the field has produced increasingly sophisticated but fundamentally \textbf{reactive} conversational agents for mental health. On the other, it has developed proactive institutional analytics that remain bottlenecked by a reliance on \textbf{manual intervention}. The failure of the existing literature is not in the individual components, but in the lack of integration between them.

This creates a significant and unaddressed research gap: the need for an \textbf{integrated, autonomous, and proactive framework} that can systemically bridge the chasm from data-driven insight to automated, personalized intervention and administrative action. Current systems are not designed as a cohesive ecosystem. The analytical tools do not automatically trigger the intervention tools, the conversational agents do not seamlessly hand off tasks to administrative agents, and the user-facing support does not operate with an awareness of the broader institutional context provided by analytics.

The central argument of this thesis is that the next frontier in institutional mental health support lies not in the incremental improvement of any single component, but in the \textbf{synergistic integration of multiple specialized agents} into a single, closed-loop system. Such a system, architected as a Multi-Agent System (MAS), is capable of emergent behaviors that are more than the sum of its parts.

Therefore, this research directly addresses the identified gap by proposing and prototyping a novel agentic AI framework, the \textbf{Safety Agent Suite}, where:
\begin{itemize}
    \item An \textbf{Insights Agent (IA)} autonomously identifies trends, moving beyond the static dashboards of current well-being analytics and creating actionable intelligence.
    \item A \textbf{Support Coach Agent (SCA)} and a \textbf{Safety Triage Agent (STA)} act on this intelligence and on real-time user needs, providing both proactive, personalized coaching and immediate, context-aware crisis support. They function as the intelligent front-door to the support ecosystem, overcoming the limitations of purely reactive chatbots.
    \item A \textbf{Service Desk Agent (SDA)} closes the "insight-to-action" loop on an administrative level, automating the workflows for clinical case management and resource allocation that currently render proactive models inefficient and unscalable.
\end{itemize}

By designing and evaluating a system where these agents work in concert, orchestrated by LangGraph, this thesis pioneers a holistic solution that is fundamentally more proactive, scalable, and efficient than the disparate tools described in the current literature.
