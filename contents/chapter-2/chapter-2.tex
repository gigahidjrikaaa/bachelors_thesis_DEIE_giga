\chapter{Tinjauan Pustaka dan Dasar Teori}

% Bab ini mengulas fondasi konseptual dan empiris yang melandasi penelitian mengenai platform terintegrasi berbasis AI, gamifikasi, dan blockchain untuk keterlibatan dan kesejahteraan mahasiswa. Bagian pertama menyajikan tinjauan pustaka terhadap karya-karya terdahulu yang relevan, mengidentifikasi kontribusi serta keterbatasan mereka, dan memposisikan penelitian ini dalam lanskap keilmuan saat ini. Bagian kedua memaparkan dasar-dasar teori mengenai teknologi inti yang digunakan (AI Konversasional Hibrida, Gamifikasi, Blockchain) serta konsep keterlibatan pengguna dan kesejahteraan mahasiswa dalam konteks digital. Terakhir, bab ini menganalisis secara komparatif pendekatan-pendekatan yang ada dan justifikasi pemilihan metode untuk penelitian ini.

\section{Tinjauan Pustaka}
\label{sec:tinjauan_pustaka_revised}

Penelitian mengenai pemanfaatan teknologi untuk mendukung mahasiswa bukanlah hal baru. Berbagai solusi digital telah diusulkan dan dievaluasi untuk meningkatkan aspek-aspek tertentu dari pengalaman belajar dan kehidupan mahasiswa. Namun, integrasi spesifik antara AI konversasional dengan kapabilitas empatik, mekanisme gamifikasi yang memotivasi, dan infrastruktur blockchain yang menjamin keamanan dan transparansi, khususnya untuk meningkatkan keterlibatan (\textit{engagement}) dan kesejahteraan (\textit{well-being}) mahasiswa secara simultan, masih merupakan area yang relatif baru dan terus berkembang, menyisakan banyak ruang untuk eksplorasi dan kontribusi ilmiah. Tinjauan ini akan mengulas beberapa area kunci penelitian terdahulu secara sistematis untuk memetakan lanskap riset saat ini dan mengidentifikasi celah yang relevan.

\subsection{AI Konversasional untuk Dukungan Kesejahteraan dan Keterlibatan Mahasiswa} 
\label{subsec:ai_konversasional_tinjauan_revised}

Pemanfaatan AI konversasional, dalam bentuk \textit{chatbot} atau agen virtual, telah menunjukkan potensi signifikan dalam menyediakan dukungan kepada mahasiswa. Penelitian awal cenderung berfokus pada implementasi \textit{chatbot} untuk tugas-tugas administratif dan penyediaan informasi umum \cite{ai_chatbots_education_advances_2024}. Seiring dengan kemajuan \textit{Natural Language Processing} (NLP) dan \textit{Large Language Models} (LLMs), fokus penelitian bergeser ke arah pengembangan agen konversasional yang lebih canggih. Tinjauan sistematis mengenai desain agen konversasional empatik, khususnya dalam konteks dukungan kesehatan mental, mengidentifikasi bahwa arsitektur hibrida seringkali lebih unggul dalam mencapai akurasi dan nuansa respons \cite{empathetic_conversational_agents_mental_health_2024}. Studi kasus di Indonesia juga mengonfirmasi penerimaan positif mahasiswa terhadap \textit{chatbot} untuk dukungan kesehatan mental awal, dengan catatan penting mengenai kebutuhan personalisasi dan penanganan privasi \cite{chatbot_student_mental_health_unuja}.

Dalam domain keterlibatan akademik, sistem tutor cerdas (\textit{Intelligent Tutoring Systems} - ITS) dan agen konversasional telah dieksplorasi untuk menyediakan pendampingan belajar personal. Tinjauan oleh Rizvi dkk. (2023) membahas bagaimana ITS berbasis AI dapat meningkatkan pengalaman belajar dan keterlibatan mahasiswa melalui jalur pembelajaran yang adaptif dan umpan balik instan \cite{rizvi_ai_tutoring_review_2023}. Kontribusi utama dari studi-studi ini adalah demonstrasi efektivitas AI dalam personalisasi pembelajaran. Namun, banyak dari sistem ini belum secara eksplisit mengintegrasikan dukungan kesejahteraan emosional atau aspek sosial dari keterlibatan. Selain itu, potensi dampak negatif seperti kelelahan digital akibat interaksi berkepanjangan dengan sistem AI, serta isu etika terkait pengumpulan dan penggunaan data mahasiswa, tetap menjadi perhatian yang memerlukan mitigasi cermat dalam desain platform \cite{ai_effects_student_wellbeing_2025}.

% PLACEHOLDER UNTUK GAMBAR/DIAGRAM
% \begin{figure}[htbp]
%   \centering
%   % \includegraphics[width=0.8\textwidth]{placeholder_diagram_ai_support_models.png} % Ganti dengan file gambar Anda
%   \fbox{\parbox[c][10cm][c]{0.8\textwidth}{\centering Gambar Placeholder: Diagram Taksonomi atau Perbandingan Model AI Konversasional untuk Dukungan Mahasiswa (misal: Rule-based vs. Retrieval vs. Generative vs. Hybrid)}}
%   \caption{Perbandingan Pendekatan dalam Pengembangan AI Konversasional untuk Dukungan Mahasiswa.}
%   \label{fig:ai_support_models}
% \end{figure}

\subsection{Gamifikasi untuk Peningkatan Keterlibatan dan Motivasi Mahasiswa}
\label{subsec:gamifikasi_engagement_tinjauan_revised}

Gamifikasi telah diakui sebagai strategi yang menjanjikan untuk meningkatkan motivasi dan keterlibatan mahasiswa di berbagai aktivitas akademik maupun non-akademik \cite{gamification_motivation_engagement_chemistry_2021, gamification_in_education_boosting_2025}. Tinjauan literatur sistematis oleh Rossi dkk. (2023) mengonfirmasi potensi ini, meskipun menekankan variabilitas implementasi dan tantangan dalam analisis empiris yang seragam \cite{gamification_higher_ed_review_2023}.

Efektivitas gamifikasi sering dikaitkan dengan kemampuannya memenuhi kebutuhan psikologis dasar seperti yang dijelaskan dalam \textit{Self-Determination Theory} (SDT) \cite{sdt_gamification_review_2015, sdt_gamification_duolingo_2025}. Namun, implementasi yang dangkal atau terlalu berfokus pada imbalan ekstrinsik berisiko menciptakan keterlibatan superfisial dan bahkan dapat merusak motivasi intrinsik jangka panjang \cite{challenges_gamification_higher_ed_2022}. Studi oleh Hanus dan Fox (2015), sebuah penelitian longitudinal, memberikan wawasan penting mengenai dampak jangka panjang gamifikasi terhadap motivasi intrinsik, perbandingan sosial, dan performa akademik, menyoroti kompleksitas interaksi ini dari waktu ke waktu \cite{hanus_longitudinal_gamification_2015}.

Lebih lanjut, respons mahasiswa terhadap gamifikasi dapat bervariasi berdasarkan karakteristik individual. Penelitian mengenai tipe pemain (\textit{player types}) menyarankan bahwa personalisasi desain gamifikasi, dengan menyesuaikan elemen permainan dengan preferensi pengguna yang berbeda (misalnya, \textit{achievers, explorers, socializers, philanthropists}), dapat meningkatkan efektivitasnya \cite{taskin_player_types_gamification_2022}. Hal ini menunjukkan bahwa pendekatan "satu ukuran untuk semua" dalam gamifikasi mungkin kurang optimal. Mengintegrasikan pemahaman tentang tipe pengguna ke dalam desain platform gamifikasi menjadi salah satu kontribusi potensial, meskipun tantangan teknis untuk personalisasi adaptif tetap ada.

% PLACEHOLDER UNTUK TABEL
% \begin{table}[htbp]
%   \centering
%   \caption{Ringkasan Studi Kunci Terkait Gamifikasi dalam Pendidikan Tinggi (Versi Diperbarui).}
%   \label{tab:gamifikasi_studi_kunci_revised}
%   \begin{tabular}{|p{0.2\textwidth}|p{0.3\textwidth}|p{0.25\textwidth}|p{0.2\textwidth}|}
%     \hline
%     \textbf{Referensi} & \textbf{Fokus/Masalah} & \textbf{Kontribusi Utama} & \textbf{Keterbatasan Utama} \\
%     \hline
%     \cite{gamification_higher_ed_review_2023} & Tinjauan sistematis gamifikasi di PT & Pemetaan lanskap riset, identifikasi potensi & Variabilitas implementasi, kurangnya analisis empiris seragam \\
%     \hline
%     \cite{hanus_longitudinal_gamification_2015} & Dampak longitudinal gamifikasi di kelas & Wawasan efek jangka panjang pada motivasi, performa & Konteks kelas spesifik, mungkin perlu replikasi \\
%     \hline
%     \cite{challenges_gamification_higher_ed_2022} & Tantangan dan risiko gamifikasi & Identifikasi potensi superfisialitas, penekanan berlebih pada kompetisi & Bersifat tinjauan naratif, perlu studi empiris lebih lanjut \\
%     \hline
%     \cite{taskin_player_types_gamification_2022} & Tipe pemain dan personalisasi gamifikasi & Menyoroti pentingnya desain adaptif & Lebih fokus pada kerangka tipe pemain, kurang evaluasi implementasi personalisasi \\
%     \hline
%   \end{tabular}
% \end{table}

\subsection{Aplikasi Teknologi Blockchain dalam Pendidikan dan Sistem Terintegrasi}
\label{subsec:blockchain_pendidikan_tinjauan_revised}

Teknologi Blockchain menawarkan potensi transformatif untuk sektor pendidikan, terutama dalam hal peningkatan keamanan data, transparansi, dan verifikasi \cite{blockchain_security_privacy_education_2020}. Aplikasi awal banyak berfokus pada penerbitan dan verifikasi kredensial akademik yang aman dan tahan pemalsuan \cite{blockchain_education_transforming_2024}. Namun, potensi blockchain melampaui sekadar ijazah digital.

Dalam konteks platform terintegrasi yang diusulkan, blockchain dapat memainkan peran krusial dalam sistem penghargaan gamifikasi. Penggunaan \textit{smart contract} dapat mengotomatisasi distribusi penghargaan (misalnya, token atau lencana digital) berdasarkan pencapaian yang terverifikasi, menciptakan sistem yang lebih transparan dan akuntabel \cite{integrated_metaverse_blockchain_ai_education_2025}. Meskipun tinjauan spesifik mengenai integrasi blockchain dalam sistem *reward* gamifikasi pendidikan masih terbatas, penelitian mengenai tantangan umum blockchain dalam gamifikasi mulai muncul, menyoroti aspek seperti pengalaman pengguna dan skalabilitas \cite{xi_blockchain_gamification_education_2020}.

Aspek penting lainnya adalah manajemen identitas dan privasi data. Konsep \textit{Self-Sovereign Identity} (SSI) yang didukung oleh blockchain berpotensi memberdayakan mahasiswa dengan kontrol lebih besar atas data pribadi mereka \cite{survey_blockchain_privacy_2024}. Ini sangat relevan mengingat platform yang diusulkan akan mengumpulkan data interaksi AI dan progres gamifikasi. Namun, tantangan implementasi SSI yang user-friendly dan integrasinya dengan sistem universitas yang ada masih signifikan. Selain itu, keseimbangan antara transparansi inheren blockchain (terutama pada blockchain publik) dan kebutuhan privasi pengguna memerlukan desain arsitektur yang hati-hati, mungkin melibatkan solusi \textit{off-chain storage} atau teknik peningkatan privasi lainnya \cite{survey_blockchain_privacy_2024}.

% PLACEHOLDER UNTUK GAMBAR/DIAGRAM
% \begin{figure}[htbp]
%   \centering
%   % \includegraphics[width=0.7\textwidth]{placeholder_diagram_blockchain_applications_education.png} % Ganti dengan file gambar Anda
%   \fbox{\parbox[c][8cm][c]{0.7\textwidth}{\centering Gambar Placeholder: Diagram Aplikasi Blockchain di Pendidikan (Kredensial, Reward Gamifikasi, SSI, Manajemen Data)}}
%   \caption{Potensi Aplikasi Teknologi Blockchain dalam Ekosistem Pendidikan.}
%   \label{fig:blockchain_applications_education_revised}
% \end{figure}

\subsection{Platform Terintegrasi untuk Keterlibatan dan Kesejahteraan Mahasiswa}
\label{subsec:platform_terintegrasi_tinjauan}

Upaya untuk meningkatkan keterlibatan dan kesejahteraan mahasiswa seringkali melibatkan berbagai intervensi atau platform digital yang berdiri sendiri. Namun, terdapat argumen kuat yang mendukung pendekatan terintegrasi. Sistem dukungan mahasiswa yang terintegrasi, yang mengkoordinasikan berbagai layanan dan sumber daya, telah terbukti dapat meningkatkan hasil akademik dan non-akademik mahasiswa dengan mengatasi berbagai penghalang secara holistik \cite{child_trends_integrated_student_support_2019}. Dalam konteks digital, platform kesejahteraan digital (\textit{digital well-being platforms}) mulai banyak dikembangkan, meskipun tinjauan sistematis seperti yang dilakukan oleh Borges dkk. (2021) menunjukkan bahwa penelitian di area ini, khususnya di pendidikan tinggi, masih terus berkembang dan memerlukan lebih banyak bukti empiris mengenai efektivitas jangka panjang dan desain yang optimal \cite{borges_digital_wellbeing_higher_ed_2021}. Pengukuran keterlibatan pengguna (\textit{user engagement}) pada platform digital semacam ini juga merupakan aspek penting, di mana tinjauan sistematis oleh Ng dkk. (2022) dapat memberikan panduan mengenai konseptualisasi dan metrik pengukuran yang relevan \cite{ng_user_engagement_systematic_review_2022}.

\subsection{Sintesis dan Celah Penelitian}
\label{subsec:sintesis_celah_detail_revised}

Dari tinjauan pustaka yang lebih mendalam ini, semakin jelas bahwa meskipun terdapat kemajuan signifikan dalam pemanfaatan AI konversasional untuk dukungan \cite{empathetic_conversational_agents_mental_health_2024, rizvi_ai_tutoring_review_2023}, penerapan gamifikasi untuk motivasi \cite{gamification_higher_ed_review_2023, hanus_longitudinal_gamification_2015}, dan eksplorasi blockchain untuk keamanan dan transparansi di sektor pendidikan \cite{blockchain_education_transforming_2024, xi_blockchain_gamification_education_2020}, beberapa celah penelitian utama tetap ada dan memotivasi urgensi penelitian ini:
\begin{enumerate}
    \item \textbf{Integrasi Sinergis Multiteknologi yang Belum Teruji:} Fokus utama tetap pada kurangnya studi yang secara komprehensif merancang, mengimplementasikan, dan mengevaluasi platform yang mengintegrasikan *ketiga* teknologi ini (AI hibrida, gamifikasi yang dipersonalisasi, dan blockchain) secara sinergis untuk tujuan ganda: peningkatan keterlibatan *dan* kesejahteraan mahasiswa. Sebagian besar penelitian masih bersifat parsial atau konseptual \cite{integrated_metaverse_blockchain_ai_education_2025}.
    \item \textbf{Personalisasi Gamifikasi Berbasis Data dan Tipe Pengguna:} Meskipun pentingnya personalisasi gamifikasi telah diidentifikasi \cite{taskin_player_types_gamification_2022}, implementasi praktis dan evaluasi platform gamifikasi yang secara dinamis beradaptasi dengan tipe pengguna atau data perilaku mahasiswa (yang mungkin difasilitasi oleh AI dan dicatat secara aman oleh blockchain) masih minim.
    \item \textbf{Studi Longitudinal dan Dampak Jangka Panjang:} Kebutuhan akan studi longitudinal untuk memahami efek jangka panjang dari intervensi teknologi, baik gamifikasi \cite{hanus_longitudinal_gamification_2015} maupun platform kesejahteraan digital \cite{borges_digital_wellbeing_higher_ed_2021}, masih sangat besar. Penelitian ini, meskipun mungkin terbatas dalam durasinya, dapat memberikan dasar untuk investigasi semacam itu.
    \item \textbf{Validasi dalam Konteks Budaya Spesifik (Indonesia):} Kebutuhan akan penelitian yang memvalidasi efektivitas dan penerimaan platform teknologi canggih ini dalam konteks budaya dan sistem pendidikan tinggi di Indonesia tetap krusial \cite{chatbot_student_mental_health_unuja}.
    \item \textbf{Desain Etis dan Tata Kelola Platform Terintegrasi:} Mengelola data sensitif dari interaksi AI, memastikan keadilan dalam sistem gamifikasi berbasis blockchain, dan menjaga privasi pengguna dalam ekosistem terintegrasi memunculkan tantangan etika dan tata kelola yang kompleks dan memerlukan eksplorasi lebih lanjut \cite{challenges_gamification_higher_ed_2022, survey_blockchain_privacy_2024}.
\end{enumerate}
Penelitian ini diajukan untuk secara langsung menjawab celah-celah ini, dengan fokus pada perancangan artefak inovatif berupa platform terintegrasi dan evaluasi empiris awal terhadap potensi dampaknya.

% ==================================================================================
% Bagian selanjutnya dari Bab 2 (Dasar Teori, Analisis Perbandingan Metode, dll.)
% akan mengikuti struktur templat yang Anda berikan. 
% Anda perlu mengembangkannya dengan detail yang sesuai.
% ==================================================================================

\section{Dasar Teori}
\label{sec:dasar_teori_revised}

% (Konten Dasar Teori seperti draf sebelumnya, namun pastikan untuk mengembangkannya secara detail)
Bagian ini memaparkan landasan konseptual dan teoritis yang relevan dengan komponen-komponen utama platform yang diusulkan. Pemahaman mendalam terhadap teori ini esensial untuk perancangan sistem yang efektif dan evaluasi yang valid. Sumber utama bagian ini adalah buku referensi, artikel tinjauan (\textit{review articles}), dan publikasi ilmiah fundamental di bidang terkait.

\subsection{AI Konversasional Hibrida (\textit{Hybrid Conversational AI})}
\label{subsec:teori_ai_detail_revised}
% (Uraikan lebih lanjut: Definisi, evolusi, arsitektur hibrida, RAG, NLP, empati, tantangan)
AI Konversasional adalah cabang kecerdasan buatan yang berfokus pada penciptaan sistem yang mampu berinteraksi dengan manusia menggunakan bahasa alami \cite{ai_chatbots_education_advances_2024}. Pendekatan hibrida berupaya menggabungkan keunggulan berbagai teknik (misalnya, LLM dengan basis pengetahuan melalui Retrieval-Augmented Generation/RAG, atau LLM yang dikombinasikan dengan alur dialog terstruktur) untuk mencapai keseimbangan antara fleksibilitas percakapan, akurasi faktual, kontrol atas respons, dan kemampuan menyimulasikan empati \cite{empathetic_conversational_agents_mental_health_2024, hybrid_ai_healthcare_2025}.

\subsection{Gamifikasi dan Teori Motivasi}
\label{subsec:teori_gamifikasi_detail_revised}
% (Uraikan lebih lanjut: Definisi, elemen, teori motivasi (SDT, Flow, dll.), kerangka desain, contoh, etika)
Gamifikasi melibatkan penerapan elemen-elemen desain permainan (misalnya, poin, lencana, papan peringkat, narasi, tantangan) dalam konteks non-permainan untuk mendorong keterlibatan, motivasi, dan perubahan perilaku positif \cite{gamification_in_education_boosting_2025, gamification_higher_ed_review_2023}. Dasar teoritisnya seringkali mengacu pada teori motivasi seperti \textit{Self-Determination Theory} (SDT), yang menekankan pentingnya otonomi, kompetensi, dan keterhubungan \cite{sdt_gamification_review_2015}.

\subsection{Teknologi Blockchain dan Aplikasinya yang Relevan}
\label{subsec:teori_blockchain_detail_revised}
% (Uraikan lebih lanjut: Konsep dasar, karakteristik, jenis, konsensus, smart contract, aplikasi (SSI, Verifiable Credentials, tokenisasi), tantangan)
Blockchain adalah sebuah teknologi buku besar terdistribusi yang memungkinkan pencatatan transaksi atau data secara aman, transparan (atau pseudo-anonim), dan tidak dapat diubah (\textit{immutable}) tanpa memerlukan otoritas pusat \cite{blockchain_security_privacy_education_2020, blockchain_education_transforming_2024}.

\subsection{Keterlibatan Pengguna (\textit{User Engagement}) dan Kesejahteraan Mahasiswa (\textit{Student Well-being}) dalam Konteks Digital}
\label{subsec:teori_engagement_wellbeing_detail_revised}
% (Uraikan lebih lanjut: Definisi operasional, dimensi, model, metode pengukuran, faktor pengaruh, konsep digital well-being)
Keterlibatan pengguna (\textit{user engagement}) adalah konstruk psikologis yang menggambarkan kualitas pengalaman pengguna saat berinteraksi dengan suatu produk atau layanan digital, melibatkan aspek kognitif, afektif, dan perilaku \cite{ng_user_engagement_systematic_review_2022}. Kesejahteraan (\textit{well-being}) mahasiswa adalah kondisi positif yang komprehensif \cite{ai_effects_student_wellbeing_2025}, di mana dalam era digital, konsep \textit{digital well-being} menjadi krusial \cite{borges_digital_wellbeing_higher_ed_2021}.


\section{Analisis Perbandingan Metode}
\label{sec:analisis_metode_revised}

Di dalam tinjauan pustaka hasil akhirnya adalah analisis secara kualitatif atau pun secara kuantitatif kelebihan dan kekurangan metode jika dikaitkan dengan masalah, batasan-batasan masalah dan solusi yang dinginkan. Analisis kuantitatif tidak wajib teapi mempunyai nilai tambah di dalam tugas akhir saudara. Bagian ini menjelaskan kenapa metode tersebut dipilih dan uraikan dengan lebih jelas metode pelaksanaan tugas akhir yang ingin Anda lakukan. 
% (Kembangkan dengan analisis pendekatan metodologis alternatif dan justifikasi DSR)

\section{Pertanyaan Tugas Akhir (Jika Perlu)}
\label{sec:pertanyaan_ta_revised}

Pertanyaan tugas akhir bersifat opsional dan dapat ditambahkan untuk menekankan hal-hal yang hendak diketahui dari tugas akhir berdasar pada tujuan tugas akhir. Pertanyaan tugas akhir dikenal dengan RQ (\textit{Research Question}) dan harus memiliki keterkaitan dengan RO (\textit{Research Objective}). Satu RO dapat memiliki satu atau lebih dari satu RQ.
