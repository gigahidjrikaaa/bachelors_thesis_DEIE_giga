\chapter{Literature Review and Theoretical Background}

% Bab ini mengulas fondasi konseptual dan empiris yang melandasi penelitian mengenai platform terintegrasi berbasis AI, gamifikasi, dan blockchain untuk keterlibatan dan kesejahteraan mahasiswa. Bagian pertama menyajikan tinjauan pustaka terhadap karya-karya terdahulu yang relevan, mengidentifikasi kontribusi serta keterbatasan mereka, dan memposisikan penelitian ini dalam lanskap keilmuan saat ini. Bagian kedua memaparkan dasar-dasar teori mengenai teknologi inti yang digunakan (AI Konversasional Hibrida, Gamifikasi, Blockchain) serta konsep keterlibatan pengguna dan kesejahteraan mahasiswa dalam konteks digital. Terakhir, bab ini menganalisis secara komparatif pendekatan-pendekatan yang ada dan justifikasi pemilihan metode untuk penelitian ini.

This chapter provides the theoretical foundations and academic context for the research. The first section, Theoretical Background, explains the core concepts and technologies that constitute the proposed framework. The second section, Literature Review, surveys existing work in related fields to identify the research gap that this thesis aims to address.

\section{Theoretical Background}
\label{sec:theoretical_background}

This section describes the foundational principles and technologies upon which the agentic AI framework is built, including Agentic AI, Large Language Models (LLMs), LLM orchestration, and workflow automation platforms.

\subsection{Agentic AI and Multi-Agent Systems (MAS)}
\label{subsec:agentic_ai}

Artificial Intelligence (AI) has evolved beyond systems that perform simple classification or prediction tasks. A key advancement in this evolution is the concept of an \textbf{intelligent agent}. An agent is defined as an autonomous entity that perceives its environment through sensors and acts upon that environment through actuators to achieve specific goals \cite{FIND_CITATION_PLEASE}. The core characteristic of an agent is its autonomy; it can operate independently to make decisions and take actions without direct human intervention.

When multiple agents operate within the same environment and interact with one another, they form a \textbf{Multi-Agent System (MAS)} \cite{FIND_CITATION_PLEASE}. In an MAS, each agent typically has a specialized role and a set of goals. The overall system's intelligent behavior emerges from the collaboration, coordination, and negotiation among these individual agents. This thesis leverages the MAS paradigm by designing a framework composed of three specialized agents (Analytics, Intervention, and Triage), each with a distinct goal, that work collaboratively to achieve the overarching objective of proactive institutional support.

\subsection{Large Language Models (LLMs)}
\label{subsec:llms}

Large Language Models (LLMs) are a class of deep learning models, most commonly based on the transformer architecture, that are trained on vast amounts of text data \cite{FIND_CITATION_PLEASE}. Their primary capability is to understand, generate, summarize, and translate human language with remarkable fluency. These models function by predicting the next word in a sequence, allowing them to perform a wide range of natural language processing (NLP) tasks.

This research utilizes a \textbf{hybrid LLM strategy} to balance performance, privacy, and capability:
\begin{itemize}
    \item \textbf{Locally-Hosted Open Models (e.g., Gemma):} These are models with open weights that can be deployed on institutional hardware. The primary advantage is data privacy and control, as sensitive data does not need to be sent to third-party vendors. They also offer low latency, which is crucial for real-time applications like the Triage Agent \cite{FIND_CITATION_PLEASE}.
    \item \textbf{Cloud-Based API Models (e.g., Gemini):} These are state-of-the-art, proprietary models accessed via an API. They often provide superior performance on highly complex tasks. In this framework, such a model is used as a robust fallback mechanism to ensure service continuity if the local model fails, thereby enhancing system reliability \cite{FIND_CITATION_PLEASE}.
\end{itemize}

\subsection{LLM Orchestration Frameworks (LangChain)}
\label{subsec:langchain}

An LLM, in isolation, is a powerful text processor but lacks the ability to perform complex, multi-step tasks or interact with external systems. This limitation is addressed by LLM orchestration frameworks like \textbf{LangChain} \cite{FIND_CITATION_PLEASE}. LangChain is a software development framework that provides modular components for building applications powered by LLMs.

Key components of LangChain utilized in this research include:
\begin{itemize}
    \item \textbf{Chains:} Sequences of calls, either to an LLM or another utility, that allow for complex, multi-step logic.
    \item \textbf{Agents and Tools:} LangChain allows the creation of agents that use an LLM as a reasoning engine to decide which "tools" to use. A tool can be any function, such as a database query, a calculation, or an API call.
    \item \textbf{Retrieval-Augmented Generation (RAG):} This technique allows an LLM to access and incorporate information from external knowledge bases before generating a response, which is crucial for providing context-aware answers.
\end{itemize}
In this thesis, LangChain serves as the core of the "Brain" within the FastAPI backend, enabling the development of the sophisticated logic required by the three agents.

\subsection{Workflow Automation Platforms (n8n)}
\label{subsec:n8n}

Modern software systems rarely exist in isolation. The ability to integrate disparate systems and automate data flows is critical for operational efficiency. \textbf{Workflow Automation Platforms} are designed for this purpose \cite{FIND_CITATION_PLEASE}. These platforms provide a visual interface to connect various applications, databases, and APIs into automated sequences, or "workflows."

This project utilizes \textbf{n8n}, an open-source workflow automation tool. Unlike a general-purpose programming language, n8n excels at tasks involving scheduling, event triggers, and system-to-system communication. In our architecture, n8n functions as the "Nervous System," responsible for:
\begin{itemize}
    \item **Scheduled Triggers:** Running tasks at specific times (e.g., triggering the Analytics Agent every Sunday via a Cron job).
    \item **API Integration:** Calling the API endpoints exposed by our FastAPI backend and communicating with external services (e.g., an email server).
    \item **Visual Workflow Management:** Providing a clear, visual representation of the automated processes, which simplifies debugging and modification.
\end{itemize}

% PLACEHOLDER UNTUK GAMBAR/DIAGRAM
% \begin{figure}[htbp]
%   \centering
%   % \includegraphics[width=0.8\textwidth]{placeholder_diagram_ai_support_models.png} % Ganti dengan file gambar Anda
%   \fbox{\parbox[c][10cm][c]{0.8\textwidth}{\centering Gambar Placeholder: Diagram Taksonomi atau Perbandingan Model AI Konversasional untuk Dukungan Mahasiswa (misal: Rule-based vs. Retrieval vs. Generative vs. Hybrid)}}
%   \caption{Perbandingan Pendekatan dalam Pengembangan AI Konversasional untuk Dukungan Mahasiswa.}
%   \label{fig:ai_support_models}
% \end{figure}

% PLACEHOLDER UNTUK TABEL
% \begin{table}[htbp]
%   \centering
%   \caption{Ringkasan Studi Kunci Terkait Gamifikasi dalam Pendidikan Tinggi (Versi Diperbarui).}
%   \label{tab:gamifikasi_studi_kunci_revised}
%   \begin{tabular}{|p{0.2\textwidth}|p{0.3\textwidth}|p{0.25\textwidth}|p{0.2\textwidth}|}
%     \hline
%     \textbf{Referensi} & \textbf{Fokus/Masalah} & \textbf{Kontribusi Utama} & \textbf{Keterbatasan Utama} \\
%     \hline
%     \cite{gamification_higher_ed_review_2023} & Tinjauan sistematis gamifikasi di PT & Pemetaan lanskap riset, identifikasi potensi & Variabilitas implementasi, kurangnya analisis empiris seragam \\
%     \hline
%     \cite{hanus_longitudinal_gamification_2015} & Dampak longitudinal gamifikasi di kelas & Wawasan efek jangka panjang pada motivasi, performa & Konteks kelas spesifik, mungkin perlu replikasi \\
%     \hline
%     \cite{challenges_gamification_higher_ed_2022} & Tantangan dan risiko gamifikasi & Identifikasi potensi superfisialitas, penekanan berlebih pada kompetisi & Bersifat tinjauan naratif, perlu studi empiris lebih lanjut \\
%     \hline
%     \cite{taskin_player_types_gamification_2022} & Tipe pemain dan personalisasi gamifikasi & Menyoroti pentingnya desain adaptif & Lebih fokus pada kerangka tipe pemain, kurang evaluasi implementasi personalisasi \\
%     \hline
%   \end{tabular}
% \end{table}

% ==================================================================================
% Bagian selanjutnya dari Bab 2 (Dasar Teori, Analisis Perbandingan Metode, dll.)
% akan mengikuti struktur templat yang Anda berikan. 
% Anda perlu mengembangkannya dengan detail yang sesuai.
% ==================================================================================

\section{Literature Review}
\label{sec:literature_review}

% (Konten Dasar Teori seperti draf sebelumnya, namun pastikan untuk mengembangkannya secara detail)
Bagian ini memaparkan landasan konseptual dan teoritis yang relevan dengan komponen-komponen utama platform yang diusulkan. Pemahaman mendalam terhadap teori ini esensial untuk perancangan sistem yang efektif dan evaluasi yang valid. Sumber utama bagian ini adalah buku referensi, artikel tinjauan (\textit{review articles}), dan publikasi ilmiah fundamental di bidang terkait.


\section{Analisis Perbandingan Metode}
\label{sec:analisis_metode_revised}

Di dalam tinjauan pustaka hasil akhirnya adalah analisis secara kualitatif atau pun secara kuantitatif kelebihan dan kekurangan metode jika dikaitkan dengan masalah, batasan-batasan masalah dan solusi yang dinginkan. Analisis kuantitatif tidak wajib teapi mempunyai nilai tambah di dalam tugas akhir saudara. Bagian ini menjelaskan kenapa metode tersebut dipilih dan uraikan dengan lebih jelas metode pelaksanaan tugas akhir yang ingin Anda lakukan. 
% (Kembangkan dengan analisis pendekatan metodologis alternatif dan justifikasi DSR)

\section{Pertanyaan Tugas Akhir (Jika Perlu)}
\label{sec:pertanyaan_ta_revised}

Pertanyaan tugas akhir bersifat opsional dan dapat ditambahkan untuk menekankan hal-hal yang hendak diketahui dari tugas akhir berdasar pada tujuan tugas akhir. Pertanyaan tugas akhir dikenal dengan RQ (\textit{Research Question}) dan harus memiliki keterkaitan dengan RO (\textit{Research Objective}). Satu RO dapat memiliki satu atau lebih dari satu RQ.
