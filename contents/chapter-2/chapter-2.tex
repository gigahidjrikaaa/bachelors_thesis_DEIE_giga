\chapter{Tinjauan Pustaka dan Dasar Teori}
\label{ch:tinjpust_dasar_teori}

\section{Tinjauan Pustaka}
\label{sec:tinjauan_pustaka}

\noindent Penelitian mengenai pemanfaatan teknologi untuk mendukung mahasiswa bukanlah hal baru. Namun, integrasi spesifik antara AI konversasional, gamifikasi, dan blockchain untuk meningkatkan keterlibatan (\textit{engagement}) dan kesejahteraan (\textit{well-being}) secara simultan masih merupakan area yang berkembang dan menyisakan banyak ruang untuk eksplorasi. Tinjauan ini akan mengulas beberapa area kunci penelitian terdahulu:

\subsection{AI Konversasional untuk Dukungan Mahasiswa} 
\label{subsec:ai_konversasional}

\noindent Penggunaan \textit{chatbot} dan agen konversasional di lingkungan pendidikan tinggi telah diteliti secara ekstensif, awalnya banyak difokuskan pada fungsi administratif atau penyediaan informasi FAQ \cite{ai_chatbots_education_advances_2024}. Seiring kemajuan \textit{Natural Language Processing} (NLP) dan \textit{Large Language Models} (LLMs), fokus penelitian bergeser ke arah eksplorasi potensi AI dalam memberikan dukungan yang lebih personal dan empatik kepada mahasiswa. Sebuah tinjauan sistematis dalam konteks kesehatan mental, misalnya, mengkaji berbagai desain agen konversasional empatik dan menyoroti bahwa arsitektur hibrida (menggabungkan beberapa pendekatan AI) seringkali mencapai akurasi dan nuansa respons yang lebih tinggi dibandingkan sistem berbasis aturan atau \textit{machine learning} murni \cite{empathetic_conversational_agents_mental_health_2024}. Hal ini relevan dengan kebutuhan mahasiswa yang kompleks. Studi kasus di universitas di Indonesia juga menunjukkan bahwa \textit{chatbot} dinilai efektif oleh mahasiswa untuk menangani isu emosional ringan hingga sedang karena aksesibilitasnya, meskipun terdapat catatan mengenai keterbatasan dalam personalisasi, penanganan kasus kompleks, dan kekhawatiran privasi \cite{chatbot_student_mental_health_unuja}. Meskipun potensi AI dalam mendukung kesejahteraan tampak signifikan \cite{ai_effects_student_wellbeing_2025}, implementasinya harus secara cermat mempertimbangkan dan memitigasi risiko terkait privasi data, potensi berkurangnya interaksi antarmanusia, dan ketergantungan berlebih pada teknologi \cite{ai_effects_student_wellbeing_2025, empathetic_conversational_agents_mental_health_2024}.

\subsection{Gamifikasi untuk Keterlibatan dan Motivasi}
\label{subsec:gamifikasi_engagement}

\noindent Gamifikasi, atau aplikasi elemen desain permainan dalam konteks non-permainan, telah diakui sebagai strategi potensial untuk meningkatkan motivasi dan keterlibatan (engagement) mahasiswa di berbagai aktivitas akademik maupun non-akademik \cite{gamification_motivation_engagement_chemistry_2021, gamification_in_education_boosting_2025}. Tinjauan literatur sistematis mengonfirmasi potensi ini dalam meningkatkan pengalaman belajar di pendidikan tinggi, meskipun implementasinya sangat beragam dan analisis empiris yang konsisten masih menjadi tantangan \cite{gamification_higher_ed_review_2023}. Beberapa studi menunjukkan korelasi positif yang signifikan antara sikap positif terhadap gamifikasi dan tingkat keterlibatan mahasiswa, di mana faktor seperti tingkat konsentrasi mahasiswa dapat berperan sebagai moderator \cite{gamification_engagement_moderating_concentration_2024}. Efektivitas gamifikasi seringkali dijelaskan melalui kerangka teori motivasi seperti \textit{Self-Determination Theory} (SDT), yang menggarisbawahi pentingnya pemenuhan kebutuhan psikologis dasar (otonomi, kompetensi, keterhubungan) untuk menumbuhkan motivasi intrinsik \cite{sdt_gamification_review_2015, sdt_gamification_duolingo_2025}. Elemen-elemen seperti poin, lencana, papan peringkat, progres, dan tantangan, jika dirancang dengan baik, dapat mendukung pemenuhan kebutuhan tersebut \cite{gamification_elements_performance_2023}. Namun, perlu dicatat bahwa desain gamifikasi yang kurang matang atau hanya berfokus pada hadiah ekstrinsik dapat gagal mempertahankan keterlibatan jangka panjang dan bahkan berpotensi mengurangi motivasi intrinsik yang sudah ada \cite{gamification_in_education_boosting_2025, gamification_higher_ed_review_2023}.

\subsection{Aplikasi Blockchain dalam Konteks Pendidikan}
\label{subsec:blockchain_pendidikan}

\noindent Teknologi Blockchain, sebagai bentuk dari \textit{Distributed Ledger Technology} (DLT), menawarkan pendekatan baru untuk manajemen data dan transaksi digital dengan fitur utama desentralisasi, transparansi (terkontrol), dan ketahanan terhadap manipulasi (\textit{immutability}) \cite{blockchain_security_privacy_education_2020}. Dalam sektor pendidikan, potensinya telah dieksplorasi terutama untuk meningkatkan integritas dan verifikasi kredensial akademik (ijazah, transkrip, sertifikat) serta menyederhanakan proses administratif \cite{blockchain_education_transforming_2024}. Studi-studi di Indonesia juga telah mengkaji penggunaan blockchain untuk meningkatkan keamanan sistem informasi pendidikan \cite{blockchain_security_education_rizky_2021}. Di luar kredensial, blockchain juga memiliki potensi untuk mendukung sistem penghargaan (\textit{reward system}) yang transparan dan akuntabel, misalnya dalam konteks gamifikasi, di mana pencapaian atau poin dapat dicatat secara \textit{immutable} \cite{integrated_metaverse_blockchain_ai_education_2025}. Selain itu, konsep \textit{Self-Sovereign Identity} (SSI) yang difasilitasi blockchain dapat memberikan kontrol lebih besar kepada individu (mahasiswa) atas data pribadi mereka \cite{survey_blockchain_privacy_2024}. Meskipun demikian, tantangan terkait skalabilitas, biaya implementasi, interoperabilitas antar sistem, serta isu privasi pada blockchain publik (karena sifat transparansinya) masih perlu dipertimbangkan secara matang dalam setiap implementasi \cite{survey_blockchain_privacy_2024, blockchain_security_privacy_education_2020}.

\subsection{Sintesis dan Celah Penelitian}
\label{subsec:sintesis_celah}

\noindent Berdasarkan tinjauan terhadap ketiga area teknologi tersebut, tampak jelas potensi masing-masing dalam mendukung aspek berbeda dari pengalaman mahasiswa: AI konversasional untuk dukungan informasional dan emosional \cite{empathetic_conversational_agents_mental_health_2024, chatbot_student_mental_health_unuja}, gamifikasi untuk mendorong motivasi dan keterlibatan aktif \cite{gamification_higher_ed_review_2023, gamification_engagement_moderating_concentration_2024}, serta blockchain untuk menyediakan lapisan dasar kepercayaan, keamanan, dan transparansi \cite{blockchain_education_transforming_2024, blockchain_security_privacy_education_2020}.

\hspace{1.25cm} Namun, penelitian yang secara komprehensif mengintegrasikan ketiga teknologi ini—\textit{hybrid conversational AI}, \textit{gamification}, dan \textit{blockchain}—dalam sebuah platform tunggal yang dirancang secara spesifik untuk meningkatkan keterlibatan *dan* kesejahteraan mahasiswa secara bersamaan, masih sangat langka. Kebanyakan studi terdahulu cenderung fokus pada satu atau dua teknologi saja, atau mengaplikasikannya untuk tujuan yang berbeda (misal, blockchain hanya untuk ijazah, AI hanya untuk layanan administratif). Meskipun terdapat visi mengenai platform terintegrasi di masa depan (misalnya, dalam konteks Metaverse \cite{integrated_metaverse_blockchain_ai_education_2025}), desain praktis dan evaluasi empiris terhadap sinergi spesifik dari ketiga teknologi ini untuk domain dukungan mahasiswa masih menjadi celah penelitian (\textit{research gap}) yang signifikan. Penelitian ini diajukan untuk mengisi celah tersebut dengan fokus pada perancangan, pengembangan prototipe, dan evaluasi dampak platform terintegrasi ini.


\section{Dasar Teori}

Berisi teori-teori yang menjadi dasar solusi atau produk hasil skripsi. Dasar teori pada umumnya diperoleh melalui buku referensi, publikasi tugas akhir, dan informasi web yang dapat dipertanggungjawabkan. Hindari penggunaan dasar teori melalui tautan wikipedia, surat kabar, atau portal berita.

\subsection{Pengenalan Aplikasi Permainan}

Proses pembuatan \textit{game} dimulai dari pembuatan \textit{game design document} dimana 
dokumen ini akan menjadi landasan pengembangan game tersebut serta menginformasikan gambaran keseluruhan game yang akan dibuat \cite{ferdiana2012agile}. \textcolor{red}{\textit{Catatan: apapun yang diambil dari tulisan orang lain harus disitasi seperti dicontohkan \cite{ferdiana2012agile}.}}

\begin{figure}[h]
	\centering
	\includegraphics[width=12cm]{contents/chapter-2/gambar-buatan-sendiri.png}
	\caption[Contoh gambar]{Contoh gambar \cite{lukito2016}}
	\label{Fig:gambar-buatan-sendiri}
\end{figure}



\textit{Game design document} adalah sebuah bagian penting dalam pembuatan game baik itu elemen-elemen penyusunnya maupun proses pengembangannya. Game design yang telah dibuat, dijabarkan satu persatu mengenai tahapan dalam pembuatan game dan hasilnya disatukan dalam bentuk dokumentasi \textit{game design document} yang digunakan oleh \textit{developer} sebagai buku petunjuk bagaimana membuat \textit{game} \cite{lukito2016}.

Dalam buku \textit{Game Design Essentials} disebutkan \textit{game design document} merupakan metode yang menghubungkan elemen-elemen penyusun \textit{game}, baik itu \textit{art, sound, program, 
gameplay} sehingga semuanya terdokumentasi menjadi satu dan menjadi acuan bagi para \textit{developer} dalam membuat \textit{game} \cite{wibirama2013dual}. 

\subsection{Dasar Teori Lainnya}

\section{Analisis Perbandingan Metode}

Di dalam tinjauan pustaka hasil akhirnya adalah analisis secara kualitatif atau pun secara kuantitatif kelebihan dan kekurangan metode jika dikaitkan dengan masalah, batasan-batasan masalah dan solusi yang dinginkan. Analisis kuantitatif tidak wajib teapi mempunyai nilai tambah di dalam tugas akhir saudara. Bagian ini menjelaskan kenapa metode tersebut dipilih dan uraikan dengan lebih jelas metode pelaksanaan tugas akhir yang ingin Anda lakukan. 

\section{Pertanyaan Tugas Akhir (Jika Perlu)}

Pertanyaan tugas akhir bersifat opsional dan dapat ditambahkan untuk menekankan hal-hal yang hendak diketahui dari tugas akhir berdasar pada tujuan tugas akhir. Pertanyaan tugas akhir dikenal dengan RQ (\textit{Research Question}) dan harus memiliki keterkaitan dengan RO (\textit{Research Objective}). Satu RO dapat memiliki satu atau lebih dari satu RQ. 

